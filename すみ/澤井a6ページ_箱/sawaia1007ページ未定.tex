%\documentclass[10pt,a4j]{utjarticle}
\documentclass[b5j,twoside,twocolumn]{utarticle}
%\documentclass[b5j,twoside]{utarticle}
%\documentclass[b5j,twoside,twocolumn]{utbook}
\setlength{\columnsep}{2zw}
\usepackage{bxpapersize}
\usepackage{pxrubrica}
\rubysetup{<hj>}
\usepackage{endnotes}
\usepackage{multicol}
\usepackage{plext}
\renewcommand{\theendnote}{[後注\arabic{endnote}]}
\renewcommand{\thefootnote}{\arabic{footnote}}
\usepackage{pxftnright}
\usepackage{fancyhdr}
\setlength{\topmargin}{5mm} % ページ上部余白の設定(182mm x 257mmから計算)。
\addtolength{\topmargin}{-1in} % 初期設定の1インチ分を引いておく。
\setlength{\oddsidemargin}{21mm} % 同、奇数ページ左。
\addtolength{\oddsidemargin}{-1in}
\setlength{\evensidemargin}{17mm} % 同、偶数ページ左。
\addtolength{\evensidemargin}{-1in}
\setlength{\footskip}{-5mm}
%\setlength{\marginparwidth}{23mm}
%\setlength{\marginparsep}{5mm}
\setlength{\textwidth}{225mm} % 文書領域の幅(上下)。縦書と横書でパラメータ(width / height)の向きが変わる。
%\setlength{\textheight}{150mm} % 文書領域の幅(左右)
\makeatletter
\def\@cite#1#2{\rensuji{[{#1\if@tempswa , #2\fi}]}}%%
\def\@biblabel#1{\rensuji{[#1]}}%%%
\makeatother
\usepackage{enumerate}
\usepackage{braket}
\usepackage{url}
\usepackage[dvipdfmx]{graphicx}
\usepackage{float}
\usepackage{amsmath,amssymb}
\newcommand{\relmiddle}[1]{\mathrel{}\middle#1\mathrel{}}
\usepackage{ascmac}
\usepackage{okumacro}
\usepackage{marginnote}
%\usepackage[top=15truemm,bottom=15truemm,left=20truemm,right=20truemm]{geometry}
\usepackage{cleveref}
\usepackage{plext}
\usepackage{pxrubrica}
\usepackage{amsmath}
\usepackage{fancybox}
\usepackage[dvipdfmx]{graphicx}
\usepackage{cancel}
\setcounter{tocdepth}{3}

%\renewcommand{\labelenumi}{(\Alph{enumi})}
\usepackage {scalefnt}
\makeatletter
\@definecounter{yakuchu}
\@addtoreset{yakuchu}{document}% <--- depende on class file
\def\yakuchu{%
\@ifnextchar[\@xfootnote %]
{\stepcounter{yakuchu}%
\protected@xdef\@thefnmark{\theyakuchu}%
\@footnotemark\@footnotetext}}
\def\yakuchutext{%
\@ifnextchar [\@xfootnotenext %]
{\protected@xdef\@thefnmark{\theyakuchu}%
\@footnotetext}}
\def\yakuchumark{%
\@ifnextchar[\@xfootnotemark %]
{\stepcounter{yakuchu}%
\protected@xdef\@thefnmark{\theyakuchu}%
\@footnotemark}}
\makeatother

\usepackage{atbegshi,etoolbox}

\newcounter{newfoot}
\patchcmd{\footnotetext}{\thempfn}{\thenewfoot}{}{}

\newcommand{\evenfootnote}[1]{%
  \ifodd\value{page}%
    \footnotemark%
    \AtBeginShipoutNext{%
      \stepcounter{newfoot}\footnotetext{#1}%
    }%
  \else%
    \stepcounter{newfoot}\footnote{#1}%
  \fi%
}


\pagestyle{fancy}

\title{\tbaselineshift =4.0pt 恋人へのジレンマ------デイヴィドソンによる概念枠批判の検討}
\author{澤井優花}
\date{\vspace{-5mm}}
\setcounter{page}{101}

\begin{document}
\maketitle

\setlength{\footskip}{-2mm}
\lhead[]{【論考】}
\chead[]{}
\rhead[恋人へのジレンマ------デイヴィドソンによる概念枠批判の検討]{}
\lfoot[]{\thepage{}}
\cfoot[]{}
\rfoot[\thepage{}]{}

\tbaselineshift =3.0pt
\let\yakuchu=\endnote
\renewcommand{\footnoterule}{\noindent\rule{100mm}{0.3mm}\vskip2mm}
%\tableofcontents
\thispagestyle{fancy}
\section{はじめに}
あの人はどんな世界を見ているのだろう、ふと疑問に思うことがある。気になるあの子はどんな景色を見ているのだろう、もしかしたら自分とは全く違った世界を見ているのではないか、こんな不安を抱いたことがある者は少なくないのではないか。私もある。例えば恋人。私は恋人が何を考えているのか、どんな世界を見ているのか気になってしまう。


しかし「世界を見る」とはそもそもどのような出来事なのだろう。「世界を見る」には何が必要なのだろうか。例えばアメリカの言語学者ベンヤミン・ウォーフは、まさしく言語が必要であると考えていた。以下は彼の引用である。


\begin{quote}
言語は経験の組織化(organize)を生み出す。われわれは言語を単に表現技術と考えてしまい、言語がまず第一に感覚経験の流れの分節・配列であり、結果として一定の世界秩序を生み出すものであることを見落としがちである。\footnote{B.L. Whorf, ‘The Punctual and Segmentative Aspects of Verbs in Hopi’ in J.B. Caroll (ed.), Language, Thought and Reality, Cambridge, Mass., 1956 p.55}
\end{quote}


ウォーフによればまず秩序のない世界があって、その無文節の世界を分節する役割を果たすのが言語である。よって彼にとって「世界を見る」とは、言語というフィルターを通して存在を分節・配列し、それを解釈する営みなのである。


ウォーフの場合は言語であったが、私たちも「我々には世界を理解するための装置なるものがある」という考えには馴染みがあるのではないだろうか。「あの人は私とみている世界が違うんだ」という表現がある。この表現は、世界を理解するための装置が自分と他者とで異なっているという考えに基づくものであると考えられる。


恋人の話に戻る。では私は恋人が考えていることを理解することはできるのだろうか。もし私達それぞれに世界を理解するための装置、何か中を覗くことの出来る箱のようなものがあるとして、それは皆同じなのだろうか。もし私のもつ装置と、恋人のもつ装置が全く異なるとすれば、私の見ている世界と恋人の見ている世界は全く別のものであるかもしれない。絶望である。それでは私のもつ装置と恋人のもつ装置が全く同じであるとすれば、私と恋人とは全く同じ世界を見ているということになる。こんなに寂しいことはあるだろうか。これもまた絶望である。私にとって、恋人が全く理解できない存在であることも、私と全く同じ世界を見ているつまらない存在であることも絶望なのである。


しかし絶望だと決めつけるのはまだ早い。ドナルド・デイヴィドソンは、このような「世界を理解するための箱」を概念枠と呼び、そのような概念枠はそもそも存在しないのだと批判した哲学者である。本稿の目的は、デイヴィドソンの概念枠批判を検討し、筆者が恋人に対して抱くジレンマを克服することである。

\section{枠組/内容の二元論}
それではさっそくデイヴィドソンの議論を見ていくことにする。コミュニケーションが成り立つ前提となる「世界の分節作用」が言語の第一義的な機能であるとする見方を、デイヴィドソンは枠組/内容の二元論とした。


先にも述べたが、世界の分節作用とは、「言語によって無文節の「存在」が分節されて、存在者の世界が経験的に成立する」という見方のことである。ウォーフの主張もこの見方に属している。


このように、世界を分節化するための装置としての言語を前提し、他方にそのような言語という概念枠によって分節される前の無文節な世界があるとする二元論的な見方を、デイヴィドソンは枠組/内容の二元論と呼びこれを批判した。彼はこのドグマについて以下のように述べる。
\begin{quote}
組織化する体系と、組織化されるものを待ち受けるという、枠組と内容の二元論は理解可能なものでも擁護可能なものでもありえない、と私は言いたい。これはそれ自体、経験主義のドグマ、第三のドグマである。\footnote{ 『真理と解釈』p.201}
\end{quote}


もしこの第三のドグマに従うのであれば、私たちは言語や概念枠による(言語と概念枠は必ずしも同じものではない)分節化のフィルターを通した世界を見ているのだということになる。言語を介して世界と接触できる一方で、私たちは常に言語や概念枠によってとらえられる限りでの世界としか接触できず、世界そのものとは隔てられているとも考えられるのである。デイヴィドソンは概念枠や言語を介さない世界との直接的な接触を回復させるために、この第三のドグマを批判していく。彼の批判がどのようなものであったかを見る前に、なぜ彼はこの枠組/内容の二元論を〈経験主義の〉〈三つ目の〉ドグマと呼んだのか、この事情を確認する。ここにはデイヴィドソンの師匠であったクワインによる、「経験主義の二つのドグマ」が関連している。


\section{\tbaselineshift =4.0pt クワイン------経験主義の二つのドグマ・翻訳の不確定性}
前章では、デイヴィドソンのいう枠組/内容の二元論について大まかな説明を行った。無文節の世界と、それを理解するための装置、つまり概念枠としての言語を想定する見方を、彼は枠組/内容の二元論であるとして批判したのだった。しかし彼がこの二元論を〈経験主義の〉〈三つ目の〉ドグマと呼ぶことには、彼の師匠にあたるクワインの論文「経験主義の二つのドグマ」が大きな影響を与えている。本章においてはそのクワインの議論を整理する。まずクワインは、論理実証主義のもつ二つのドグマを批判し、その代案として全体論的言語観というアイデアを提唱する。そこでクワインは根底的翻訳という思考実験を行い、同じ内容であっても異なる概念枠(言語)を通すことで世界は全く別のものに見えてしまうという結論を導くこととなる。

\subsection{経験主義の二つのドグマ}
クワインは論文「経験主義の二つのドグマ」\footnote{『言語哲学大全Ⅱ 意味と様相(上)』pp.194〜200}において、論理実証主義の二つのドグマを退けた。論理実証主義とは、認識の根拠は経験に依る検証であり、命題の意味はその検証方法に他ならないという思想である。この思想の下では、検証不可能な形而上学は排され、検証可能な文のみがまっとうな文であるとされた。クワインは論理実証主義のもつ二つのドグマを指摘する。


二つのドグマとは「⑴分析的真理と綜合的真理との間に根本的な分割がある。」「⑵還元主義。有意味な言明はどれも直接的経験を指す名辞から何らかの論理的構成物と同値である。」の二つである。まず⑴について、クワインは「意味によって」真という分析性の概念を定義しようとすると循環に陥ってしまうのだということを示すことにより、分析・綜合の区別は曖昧で根拠のないものであると論じる。また⑵については、私たちの世界に対する言明は個々独立にではなく全体として検証されるべきであるという全体論を主張することによってこのドグマを退けた。
\subsection{根底的翻訳と翻訳の不確定性}
このようにして二つのドグマを退けたクワインはのちに代案として「全体論的言語観」\footnote{クワインは分析的言明と綜合的言明の間に境界は無いのだと指摘した。ゆえに体系の別の場所でおもいきった調整を行えば、全体との整合性を保つ場合において個々の言明は何が起ころうとも真とみなされる。このような考え方を全体論的言語観と呼ぶ。言明の翻訳においても全体と整合的であれば、ある言明に対し異なる翻訳が複数ある場合もそれぞれが成り立つとされる。}を提案する。ここでクワインは「根底的翻訳」\footnote{ 『ことばと対象』p.40〜47}という思考実験を行う。完全に未知の言語を、その言語を話す人々の振る舞いだけを元に翻訳しようという思考実験である。実験において翻訳者は未知の言語を話す人々が、刺激に対しどのような言語表現を行っているのかパターンを見出し、現地人の文の刺激意味を自身の文と対応させようとする。しかし文を対応させることが出来たとしても、その文に含まれる語が何を指示しているのかについては複数の翻訳が可能であり、それらの翻訳は一つに定まらない。そして文の対応のさせ方にも複数の仕方が考えられる。ゆえにクワインによれば相対性抜きに「この言葉の意味はなにか」「この言葉の指示対象はなにか」などの問題を語ることはできない。これを翻訳の不確定性と呼ぶ。


これは同じもの、つまり同じ内容も、異なる概念枠(ここでは言語)を通すことによって異なるものに見えてしまうということである。異なる概念枠は互いに翻訳不可能であり、概念枠の持ち主はそれぞれ異なる世界に住むということになる。
クワインの言う「同じ内容を異なる概念枠によって見る」という見方、つまり、「内容/枠」の二元論は、デイヴィドソンによれば脱ドグマ化したはずのクワインにも残る三つ目のドグマなのである。

\section{\tbaselineshift =4.0pt デイヴィドソン------枠/内容の二元論批判}
前章においてデイヴィドソンの「枠組-内容の二元論」批判の経緯について明らかにした。デイヴィドソンの師匠にあたるクワインは「経験主義の二つのドグマ」において、「⑴分析・綜合の区別」「⑵還元主義」を論理実証主義の抱える二つのドグマであるとしてこれらを批判した。しかしながらデイヴィドソンは、これら二つのドグマを批判し脱ドグマ化したはずのクワインが枠組/内容の二元論という三つ目のドグマを捨てきれていないのだと批判する。これが、デイヴィドソンが枠組/内容の二元論に〈経験主義の〉〈三つ目の〉といった形容を付す所以である。



『経験主義の二つのドグマ』で論理実証主義の二つのドグマを批判したのちクワインは根底的翻訳というアイデアを提唱する。クワインがいうには我々は同じ内容に対して異なる概念枠を通してみることによって、それらをまったく異なるものとして理解している。それは経験的内容に直に接しているのではなく、概念枠を介し捉えられる限りでのそれに接しているということなのである。また、クワインのいう「翻訳の不確定性」とは、未知の言語を翻訳する仕方は原理的に決定不可能であるとすることであった。このような考え方は話し手の信念体系をブラックボックス化することに他ならない。これでは、そもそも私たちが他者の考えにアクセスする仕方を持っておらず、自分と他者の信念が一致する・しないなどと議論すること自体が意味をなさないものとなってしまう。


たしかに我々が信念体系を秩序づける枠として思い浮かべるのは「言語」であり、他者の思考にアクセスするには、そのつど解釈を行うしかない。私たちは他者の言語表現と信念とを結びつける。しかしながら言語表現と信念との結合は、同じ表現を使用する全ての場合に妥当するとは限らず、解釈は無限に続くこととなる。デイヴィッドソンによれば、言語が枠として成り立っているという前提は成立しないのである。


デイヴィドソンは「概念枠という考えそのものについて」において内容・枠の二元論を三つ目のドグマであるとして批判し、世界との、概念枠を介さない直接的接触を回復しようと試みる。よって本章ではデイヴィドソンの「概念枠という考えそのものについて」における枠組/内容の二元論批判を明らかにする。
\subsection{概念枠という考えそのものについて}
デイヴィドソンによれば、内容/枠の二元論を支持する者は二つのメタファーに従っているのである。
\begin{quote}
⑴念枠は世界を整理する\\
⑵概念枠は世界に適合する
\end{quote}

デイヴィドソンはどちらのメタファーに従っても「互いに翻訳不可能な異なる概念枠」という考えは成り立たないのだと指摘する\footnote{『真理と解釈』p203.}。


まず⑴について、彼によれば「言語が世界を整理する」とは「言語が世界の中にあるものを整理する」としか理解することができない。クワインのいう互いに翻訳不可能な異なる概念枠という考えが成り立つのならば、世界の中にあるものの異なる整理の仕方があるということになる。しかし世界の中にあるものの異なる整理の仕方があるのは、あらかじめ世界の中にあるものを共有している場合である。もし世界の中にあるものが共有されているのであれば、概念枠は互いに翻訳可能であるはずで、言語相対主義は成り立たない。


そして⑵の「概念枠は世界に適合する」について、デイヴィドソンによれば概念枠が世界に適合するということは、概念枠(言語)の中に世界のありように一致するような文がある、ということとして理解できる。つまり、一方の概念枠においては真である文で、異なる概念枠においては翻訳できないものがあるということになる。
ここでデイヴィドソンはタルスキの真理論を参考にしている。ここでは簡単に説明する。タルスキは、〈実質的に適切な真理の定義〉が満たす条件として「規約T」という制約をあげている。\footnote{『真理と解釈』pp.207〜208}
\begin{itembox}[l]{\textbf{規約T}}
ある形式言語Lについての真理定義から次の形式をした文の全て定理として導出されるならばその定義は真理の定義として実質的に適切である。
\begin{quote}
(T)S is true if and only if P.
\end{quote}
\end{itembox}


Sには対象言語(Lの任意の文)の文が、PにはSのメタ言語における翻訳が代入される。(「Snow is white が真であるのは雪が白い時、その時に限る」はこの代入の例である。英語が対象言語であり、メタ言語における翻訳が「雪が白い」)デイヴィドソンによれば、この規約Tにおいて、既知の言語に翻訳するという概念が本質的な仕方で使用されている。ところが概念相対主義は、「真であるが翻訳できない言語」を主張しており、真理という概念を翻訳抜きに理解できるという前提に基づいている。タルスキに従えば、異なる概念枠において翻訳不可能な文の「真理」を翻訳から独立して語ることは不可能であり、概念枠は世界に適合するという二つ目のメタファーも意味をなさない。
以上によって、言語があらかじめ概念枠として機能しており、言語を共有するもの同士で信念の伝達が可能であるという、枠組/内容の二元論は成り立たないことが明らかとなった。

\section{おわりに}
以上によってデイヴィドソンによる概念枠批判を明らかにした。無文節な世界と、それを理解するための言語があるのだという考え方は、彼によれば枠組/内容の二元論というドグマであった。そしてデイヴィドソンがこのドグマを〈経験主義の〉〈第三の〉ドグマと呼ぶのは、このドグマが、経験主義の二つのドグマから脱ドグマ化したはずのクワインにも残るドグマであったことに由来している。経験主義の二つのドグマを批判したクワインがのちに行った思考実験である根底的翻訳によって、クワインは概念枠が異なれば見る世界も異なるのだと結論付けたのである。このような枠組/内容の二元論という第三のドグマを批判したのが、デイヴィドソンによる「概念枠という考えそのものについて」である。彼によれば、内容/枠の二元論は二つのメタファー「⑴概念枠は世界を整理する」「⑵概念枠は世界に適合する」によって支えられている。しかしながらこの二つのメタファーも、前者は言語相対主義の批判によって、後者はタルスキの真理論を用いることによって否定される。


本稿の目的はデイヴィドソンによる概念枠批判を検討することにより、筆者の恋人に対するジレンマを克服する事であった。そのジレンマとは、私と恋人が見ている世界が異なり、恋人が理解不能な存在であることも絶望であれば、恋人の見る世界が私の見ている世界と同じであっても絶望である、というものであった。デイヴィドソン風の言い方をすれば、私がもつ概念枠と恋人のもつ概念枠が全く異なることも、それらが全く同じであることも同様に絶望なのである。


しかしながらデイヴィドソンによって、この「概念枠」という世界を見るための装置の存在は否定された。概念枠が人によって異なるわけではなく、皆同じものを持っているというわけでもない。そもそもそのような装置は存在しないのである。これは救いだ。そもそも概念枠という考え方が否定されてしまえば、他者が見る世界について悩む必要もないのだ。私が恋人と全く異なる世界を見ているのでも、全く同じ世界を見ているのでもない。私たちはただ世界を見ているのである。

{\small
\section*{参考文献}
\renewcommand{\labelenumi}{\pbox<y>{[\arabic{enumi}]}}
\begin{enumerate}
\item \pbox<z>{B.L. Whorf, ‘The Punctual and Segmentative Aspects of Verbs in Ho} \\ \pbox<z>{pi’ in J.B. Caroll (ed.), Language, Thought and Reality, Cambridge,}\\ \pbox<t>{Mass., 1956}
\item D. デイヴィドソン著、野本和幸他訳、『真理と解釈』、勁草書房、一九九一
\item 飯田隆著、『言語哲学大全Ⅱ 意味と様相(上)』、勁草書房、一九八九
\item  W.V.O.クワイン著、大出晃他訳、『ことばと対象』、勁草書房、一九八四
\item 野本和幸、山田友幸編、『言語哲学を学ぶ人のために』、世界思想社、二〇〇二
\end{enumerate}
}
\end{document}