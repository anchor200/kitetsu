%\documentclass[10pt,a4j]{utjarticle}
\documentclass[b5j,twoside,twocolumn]{utarticle}
%\documentclass[b5j,twoside]{utarticle}
%\documentclass[b5j,twoside,twocolumn]{utbook}
\setlength{\columnsep}{2zw}
\usepackage{bxpapersize}
\usepackage{pxrubrica}
\rubysetup{<hj>}
\usepackage{endnotes}
\usepackage{multicol}
\usepackage{plext}
\renewcommand{\theendnote}{[後注\arabic{endnote}]}
\renewcommand{\thefootnote}{\arabic{footnote}}
\usepackage{pxftnright}
\usepackage{fancyhdr}
\setlength{\topmargin}{5mm} % ページ上部余白の設定(182mm x 257mmから計算)。
\addtolength{\topmargin}{-1in} % 初期設定の1インチ分を引いておく。
\setlength{\oddsidemargin}{21mm} % 同、奇数ページ左。
\addtolength{\oddsidemargin}{-1in}
\setlength{\evensidemargin}{17mm} % 同、偶数ページ左。
\addtolength{\evensidemargin}{-1in}
\setlength{\footskip}{-5mm}
%\setlength{\marginparwidth}{23mm}
%\setlength{\marginparsep}{5mm}
\setlength{\textwidth}{225mm} % 文書領域の幅(上下)。縦書と横書でパラメータ(width / height)の向きが変わる。
%\setlength{\textheight}{150mm} % 文書領域の幅(左右)
\makeatletter
\def\@cite#1#2{\rensuji{[{#1\if@tempswa , #2\fi}]}}%%
\def\@biblabel#1{\rensuji{[#1]}}%%%
\makeatother
\usepackage{enumerate}
\usepackage{braket}
\usepackage{url}
\usepackage[dvipdfmx]{graphicx}
\usepackage{float}
\usepackage{amsmath,amssymb}
\newcommand{\relmiddle}[1]{\mathrel{}\middle#1\mathrel{}}
\usepackage{ascmac}
\usepackage{okumacro}
\usepackage{marginnote}
%\usepackage[top=15truemm,bottom=15truemm,left=20truemm,right=20truemm]{geometry}
\usepackage{cleveref}
\usepackage{plext}
\usepackage{pxrubrica}
\usepackage{amsmath}
\usepackage{fancybox}
\usepackage[dvipdfmx]{graphicx}
\usepackage{cancel}
\setcounter{tocdepth}{3}

%\renewcommand{\labelenumi}{(\Alph{enumi})}
\usepackage {scalefnt}
\makeatletter
\@definecounter{yakuchu}
\@addtoreset{yakuchu}{document}% <--- depende on class file
\def\yakuchu{%
\@ifnextchar[\@xfootnote %]
{\stepcounter{yakuchu}%
\protected@xdef\@thefnmark{\theyakuchu}%
\@footnotemark\@footnotetext}}
\def\yakuchutext{%
\@ifnextchar [\@xfootnotenext %]
{\protected@xdef\@thefnmark{\theyakuchu}%
\@footnotetext}}
\def\yakuchumark{%
\@ifnextchar[\@xfootnotemark %]
{\stepcounter{yakuchu}%
\protected@xdef\@thefnmark{\theyakuchu}%
\@footnotemark}}
\makeatother

\usepackage{atbegshi,etoolbox}

\newcounter{newfoot}
\patchcmd{\footnotetext}{\thempfn}{\thenewfoot}{}{}

\newcommand{\evenfootnote}[1]{%
  \ifodd\value{page}%
    \footnotemark%
    \AtBeginShipoutNext{%
      \stepcounter{newfoot}\footnotetext{#1}%
    }%
  \else%
    \stepcounter{newfoot}\footnote{#1}%
  \fi%
}


\pagestyle{fancy}

\title{\tbaselineshift =4.0pt SNSにおけるキャラクターの人格と距離感------シナモンへ向けられた悪意の記録}
\author{中野由梨花}
\date{\vspace{-5mm}}
\setcounter{page}{101}

\begin{document}
\maketitle

\setlength{\footskip}{-2mm}
\lhead[]{【エッセイ】}
\chead[]{}
\rhead[SNSにおけるキャラクターの人格と距離感------シナモンへ向けられた悪意の記録]{}
\lfoot[]{\thepage{}}
\cfoot[]{}
\rfoot[\thepage{}]{}

\let\yakuchu=\endnote
\renewcommand{\footnoterule}{\noindent\rule{100mm}{0.3mm}\vskip2mm}
%\tableofcontents
\thispagestyle{fancy}
\section*{はじめに}
サンリオのキャラクター群「シナモロール」に、「シナモン\footnote{シナモン【公式】ツイッター \url{https://twitter.com/cinnamon_sanrio}}」というキャラクターがいる。個人的な話から入るが、筆者が愛してやまないキャラクターである。一般知名度も(おそらく)高く、実際にサンリオ内の人気投票でも上位の常連の彼はしかし、短くはない期間、直接的な悪意を向けられていた時期があった。それも、SNS上で。本稿ではその記録と、それを通してSNSにおけるキャラクターの人格と、SNS上の他者との距離感について述べる。

\section*{シナモンのツイッターの軌跡}
こいぬの男の子である彼は、出身地は遠いお空の雲の上ではあるが、ある日シュクルタウンという街に降りてきた。そこで個性豊かなフレンズたちと遊んだり、住んでいるカフェで看板犬として働いていたりする。


シナモンはキャラクター登場当時から高い人気を誇っていたが、第二次シナモンブームともいえる昨今の人気を生み出したきっかけはシナモンのツイッターアカウント開設といえるだろう。シナモンがアカウントを開設したのが二〇一四年であるが、サンリオキャラクターの人気投票である、サンリオキャラクター対象の順位は二〇一四年までの5位前後の順位から上昇、二〇一五年には3位、二〇一六年には2位、二〇一七年と二〇一八年には2年連続で1位を獲得している。


シナモンの人気向上に一役買っていそうなツイッターの内容は様々だ。一日一度はシナモンの日常を可愛い画像とともにお届けしてくれる。季節に応じた遊びをシナモンとフレンズたちが楽しんでいる様子であったり、シナモンやフレンズの趣味や近況をお伝えしてくれたり、サンリオの他のツイッターアカウントを持つキャラクターと交流してみたりする。そしてそれぞれのツイートはバラバラで存在するのではなく、シナモンの日常は毎日続いている。雨の日に外ではしゃいでみたら、翌日に風邪をひいて寝込んでいたりする。そこにフレンズがお見舞いに来て、数日後に風邪が治った報告があって︙︙。明らかにそこには、シナモンの一続きの「日常」が存在するのだ。ここで重要なのは、我々フォロワーはそこでただの傍観者であることもできるが、シナモンの日常に参画することができるという点だ。シナモンが風邪をひいたきっかけになった雨の日、彼は傘を差していたのだが、長い耳は傘の中に収まりきらない。そこでシナモンを愛するフォロワーたちは彼に温かいメッセージを送った(2015/06/19-06/25)。その結果、風邪が治った後のシナモンのツイートによると、「傘から耳がはみ出てるよーってお友達に教えてもらったから、レインコートにしてみたんだ」(2015/07/02)とのこと。そこには風邪から回復したシナモンが黄色い雨合羽に包まれる画像が添付してあった。つまり、シナモンに私たちの「声」が届いたことがこのツイートによって実感されるのだ。


シナモンとの現実のつながりはフォロワーのリプライ以外にもある。シナモロールカフェやシナモロールのドリンクスタンドなど、実在するカフェの新メニュー考案をシナモンが行っている様子を、我々はツイッターを通じて知ることができる。シナモンが悩みに悩んで開発した新メニューは、しばらく後に我々が実際に当該店舗で手に取り味わうことができるのだ。このようにシナモンは我々の現実の延長上に確かに存在している。


\section*{いじめの発生と、その構造}
しかし、このようなシナモンとの双方向のコミュニケーションがもたらすものは、必ずしも平和で幸せな効果だけではない。インターネットの問題について考える際に避けては通れない「インターネットいじめ」、シナモンはこの被害者になっていた時期があったのだ。


シナモンへのいじめが見られ始めたのは二〇一五年五月頃だ。その頃もシナモンは前述のような何気ない日常のツイートをしていたのだが、それらのツイートに対して心無いコメントが寄せられるようになった。いじめのリプライは暴言だけにとどまらず、シナモンの投稿した画像を性的に編集したものも含まれていた。これらについてはインターネットで検索していただければすぐにどのようリプライが送られていたかを見ることができるので、わざわざ引用はしないが、どれも言いがかりにしか見えないような内容ばかりであった。シナモンへの直接的な攻撃以外にも、なりすましアカウントも存在した。アカウントやIDをシナモンとそっくりにして、画像も公式画像を巧妙に加工した、いってしまえば「出来の良い」コラージュ画像が掲載された。面白がって拡散する人や、本物だと思ってしまった人までいて、影響力は大きかった。


さて、なんの悪意も発さず、過激なツイートもなく、無害にふわふわと暮らしていたキャラクターに対してこのようないじめが発生したのか。筆者はその原因を二つの構造にあると考えている。


一つの構造はキャラクターとツイッター利用者との関係だ。SNS上でのいじめやいやがらせにおいてしばしば指摘される問題として、「画面向こうに現実の人間が想定されていないこと」というものがある。シナモンの例においてはシナモンはまさに現実の人間ではなく、人格を持たないキャラクターでしかない。そのため、シナモンに暴言を浴びせることで誰かが傷つくことが想像されないのだ。また、シナモンは攻撃されたとて毎日ツイートを続ける。シナモンがツイートをするのは彼の日常だが、しかしながらこれは彼のプロモーションであり、仕事の一部だ。毎日何かを言われても変わらずツイートを続けるのだから、加害者たちの心に罪悪感は発生しなかったであろう。とはいえ、彼が行うのが機械的なツイートのみであれば毎日コメントをするのも飽きがくるだろうが、前述のようにシナモンのツイッターの中に、そして現実とも境界が曖昧なつながりを持った日常が存在している。変化と継続を共に含んだ確かな質量を持った日常は、ファンにとってはシナモンのキャラクターとしての人格にも質量を持たせ、喜ばしいものであったが、悪意を持った人々にとっても飽きのこないちょうど良い対象となってしまったのだ。


もう一つの構造はツイッターの表示形式による、他者との関係だ。ツイッターのリプライは他者が見ることができる。シナモンのツイートを普段見ている人はもちろん、そうでない人でもリツートやいいねを通してタイムラインで当該ツイートを見かける機会が存在する。そのため、シナモンへのリプライが、なんらかの新規性やブラックユーモアによって人に「ウケる」ものであればあるほど、彼らのツイートへの反応は良くなっていく。日夜開催されているSNS上の大喜利と変わらない感覚がそこにはあったのではないか。この場合、おそらくいじめの対象はシナモンでなくとも良いのだ。求められていたものはいじめる対象のアカウントの規模の大きさだ。シナモンのアカウントのフォロワー数や知名度が、彼へのリプライをより多くの人目に触れさせることを可能にしたのだ。リプライだけでなく、なりすましアカウントについても同様だ。リプライよりも単独のツイートの方が拡散されやすい性質も、当該アカウントが大きくなってしまった一因であろう。


これらの構造によって、いじめが加熱しやすい状況が重なってしまった。「傷つくことのない対象への暴言が他者から承認される」ことに端を発して、ただ流行っているから乗っかった人もいるだろう。ツイッターではよく見られることだが、美しく綺麗な言葉よりも、分かりやすい悪意の方が反響が大きい場合がある。シナモンへのいじめはツイッター上で一つのトレンド、あるいはコンテンツとなってしまったのだ。


\section*{対応と終結}
もちろんサンリオ側もこのようないじめの横行を野放しにしたわけではない。シナモンの友人のキャラクターであるシフォンが、「シナモンはアタシが守る」とシナモンの前に歩み出るイラストや、シナモンが友人たちとともに「みんななかよく♪」と笑う画像(ともに2015/05/12)がツイートされた。シナモンの今までのゆるくやわらかい日常において、「守る」という言葉が出ることは、はっきり言って異常事態だった。それでもそのツイートに対しても暴言は続いた。その結果として、サンリオのとった対応は、暴言リプライを送ってきたアカウントをブロックするというものだった。


宣伝のためツイッターを開設しているアカウントにとって、ブロックは危険な行為だ。フォロワー数やリツイート、いいねの数が重要となってくるツイッターにおいて、それらの担い手の数を自ら減らすことにもなるし、さらに「気に入らないユーザーをブロックするアカウント」という印象を与えてしまうことにもなる。最終手段ともいえるだろう。それを行ったサンリオは、そこまでの危機感を覚えていたということだ。


これらのサンリオの対応の他に、流れを変えたものが、一般ユーザーが行ったあるツイートだ。「シナモン氏のTwitterを担当されている、サンリオの社員さんの写真」として、ある女性の写真が掲載された(この写真はデマであり、実際は台湾のモデルであると言われている)。このツイートが拡散されてから、シナモンへのリプライが急変した。「中の人」、すなわちツイッター担当者の存在\yakuchu{シナモンのツイッターの担当者、とか、中の人とか書いていますが、シナモン曰くシナモンのツイッターはシナモンのまわりにいる鳥さんたち(数匹いる)がシナモンの声を届けてくれるものであり、鳥さんたちはシフト制らしいです。}を前提としたツイートが目立つようになったのだ。中の人の容姿を褒めるものや、それよりさらに過激でセクシャルな言葉を浴びせるものもあったが、内容は明らかに変わった。これまで続いていたようなリプライは下火になっていった。


構造の一つとして現実の人間の不存在を挙げたが、現実の人間の存在が示唆されたことによってその構造が崩されたのだ。それがデマであろうと彼らの中で真実になればシナモンの向こうには「誰か」が見える。シナモンに暴言を浴びせてきた人々は、その誰かの存在に気付けば\tbaselineshift =2.5pt ------\tbaselineshift =4.0ptそしてその誰かの容姿が好みであればいっそう\tbaselineshift =2.5pt ------\tbaselineshift =4.0ptいじめをやめるくらいの気軽さで行っていたのだ。彼らに足りていなかったものは色々あるが、最も足りなかったのは想像力だといえるだろう。


いじめを鎮火させたものには、件の中の人ツイート以外にも様々な要素がある。まとめサイトにまとめられ、新聞やテレビのニュースにも取り上げられた。そしてシナモンのファンたちが戦った。暴言リプライに対して反抗したり、シナモンに優しい言葉をかけ続けたりした。


シナモンが現実とつながれてしまうことで、彼は直接的な悪意をぶつけられてしまったが、ファンの言葉だって彼に届けることができる。暴言リプライへの反抗は、醜い応酬を生みやすいため最善の手とは言えないが、シナモンのことを大切に思っている人の存在を誰かに、そしてシナモンに印象付けることはできる。そしてシナモンへの温かい言葉は、シナモンと、その向こうにいる人たちにとってどれだけ力になったことだろうか。
シナモンへのいじめが落ち着いた二〇一五年六月末に、彼のフォロワーは\rensuji{20}万人を超えた。その時の彼の感謝のツイートには、こんな文言が含まれていた。「悲しい思いをしたこともあったけど、みんなの声にいつも元気をもらっているんだ」(2015/06/30)。いつも元気で楽しそうなシナモンが、「悲しい思いをした」と自ら告げるのは、これも前述の「守る」発言と並ぶくらい異常な出来事である。それでもわざわざおめでたい報告のツイートでこのことを述べたのは、「みんなの声」がシナモンにとってどれだけ力を持っていたかを伝えたかったからだと捉えてもよいだろう。ファンとしては、そう思っていたい。


\section*{その後}
シナモンのツイッターでのいじめは以上のような経緯を辿った。爆発的に増加した悪意のあるリプライが完全になくなるまでにはかなりの時間を要した。いじめが落ち着いて、シナモンは現在でも元気にツイッターを続けている。


しかし、いじめの種や記憶は完全になくなったわけではない。シナモンのツイッターのプロフィールから、サンリオのソーシャルメディアポリシーのURLが消されることはない。騒動から二年経った頃に行われていた「ふわふわシナモロール展」\yakuchu{ふわふわシナモロール展におけるツイートのパネル展示は泣けるものが多くて、ここで言及しているツイートも当然泣けるのですが、シナモンは成人の日ツイートが毎年泣けるともっぱらの評判です。ぜひ検索してみてください。}でまとめられていたシナモンのツイッターの歴史には、いじめのことは書かれていなかったものの、前述の「悲しい思いをしたこともあったけど」を含むツイートが掲載されていた。最近になってまた時々\tbaselineshift =2.5pt ------\tbaselineshift =4.0pt当時の騒動を知ってか知らずか\tbaselineshift =2.5pt ------\tbaselineshift =4.0pt暴言に近いリプライやセクシャルなコラージュ画像が送りつけられることがある。さらには、シナモン以外のサンリオキャラクターが心無い言葉を浴びせられている様子も目にする。


SNS上でキャラクターをいじめの対象にすることは、ストレス解消の手段の一つでもあるのかもしれないが、大賞が人間ではないからといって「物に当たる」と同様の感覚とはいえないのではないかと考える。誰かから愛されている対象に向かって悪意をぶつけることは、それを愛している人々を傷つけるし、もちろん画面の向こうに存在する人たちのことも傷つけている。想像力の欠如によってそれらに気付けないこと、そのまま行動に移してしまうこと。さらにその上に、いじめを行う彼らの行動を見て、反応する人たちの存在も関わってくる。


\section{おわりに}
本稿ではシナモンに起こったことに限定して述べたが、同様の現象は至る所で起こっている。ツイッターだとキャラクターだけでなくアイドルのアカウントや、最近では大学のミスコン用アカウントでも似たようなことが起こる。もちろんツイッター以外でも発生している。彼らは基本的に我々を拒絶しないから、確かに、おもちゃにしたり、甘えたりしやすいのかもしれない。悪意の対象になりやすい彼らはSNS上でしか認識されず、他者の現実に人格を持たないのだ。それゆえにその他者らは適切な距離感を見誤る。


どうか、心無いリプライや、不快に思われる可能性の高いリプライを送る前に、面白がっていいね・リツイートをする前に、一度指を止めて想像してほしい。どんなアカウントでも、その向こうに「誰か」がいること、その「誰か」を愛している人がどこかにいることを。そしてそのなんの気なしに行った行動を、また誰かが見ていることを。\\ \\
\theendnotes

\end{document}
