%\documentclass[10pt,a4j]{utjarticle}
\documentclass[b5j,twoside,twocolumn]{utarticle}
%\documentclass[b5j,twoside]{utarticle}
%\documentclass[b5j,twoside,twocolumn]{utbook}
\setlength{\columnsep}{2zw}
\usepackage{bxpapersize}
\usepackage{pxrubrica}
\rubysetup{<hj>}
\usepackage{endnotes}
\usepackage{multicol}
\usepackage{plext}
\renewcommand{\theendnote}{[後注\arabic{endnote}]}
\renewcommand{\thefootnote}{\arabic{footnote}}
\usepackage{pxftnright}
\usepackage{fancyhdr}
\setlength{\topmargin}{5mm} % ページ上部余白の設定(182mm x 257mmから計算)。
\addtolength{\topmargin}{-1in} % 初期設定の1インチ分を引いておく。
\setlength{\oddsidemargin}{21mm} % 同、奇数ページ左。
\addtolength{\oddsidemargin}{-1in}
\setlength{\evensidemargin}{17mm} % 同、偶数ページ左。
\addtolength{\evensidemargin}{-1in}
\setlength{\footskip}{-5mm}
%\setlength{\marginparwidth}{23mm}
%\setlength{\marginparsep}{5mm}
\setlength{\textwidth}{225mm} % 文書領域の幅(上下)。縦書と横書でパラメータ(width / height)の向きが変わる。
%\setlength{\textheight}{150mm} % 文書領域の幅(左右)
\makeatletter
\def\@cite#1#2{\rensuji{[{#1\if@tempswa , #2\fi}]}}%%
\def\@biblabel#1{\rensuji{[#1]}}%%%
\makeatother
\usepackage{enumerate}
\usepackage{braket}
\usepackage{url}
\usepackage[dvipdfmx]{graphicx}
\usepackage{float}
\usepackage{amsmath,amssymb}
\newcommand{\relmiddle}[1]{\mathrel{}\middle#1\mathrel{}}
\usepackage{ascmac}
\usepackage{okumacro}
\usepackage{marginnote}
%\usepackage[top=15truemm,bottom=15truemm,left=20truemm,right=20truemm]{geometry}
\usepackage{cleveref}
\usepackage{plext}
\usepackage{pxrubrica}
\usepackage{amsmath}
\usepackage{fancybox}
\usepackage[dvipdfmx]{graphicx}
\usepackage{cancel}
\setcounter{tocdepth}{3}

%\renewcommand{\labelenumi}{(\Alph{enumi})}
\usepackage {scalefnt}
\makeatletter
\@definecounter{yakuchu}
\@addtoreset{yakuchu}{document}% <--- depende on class file
\def\yakuchu{%
\@ifnextchar[\@xfootnote %]
{\stepcounter{yakuchu}%
\protected@xdef\@thefnmark{\theyakuchu}%
\@footnotemark\@footnotetext}}
\def\yakuchutext{%
\@ifnextchar [\@xfootnotenext %]
{\protected@xdef\@thefnmark{\theyakuchu}%
\@footnotetext}}
\def\yakuchumark{%
\@ifnextchar[\@xfootnotemark %]
{\stepcounter{yakuchu}%
\protected@xdef\@thefnmark{\theyakuchu}%
\@footnotemark}}
\makeatother

\usepackage{atbegshi,etoolbox}

\newcounter{newfoot}
\patchcmd{\footnotetext}{\thempfn}{\thenewfoot}{}{}

\newcommand{\evenfootnote}[1]{%
  \ifodd\value{page}%
    \footnotemark%
    \AtBeginShipoutNext{%
      \stepcounter{newfoot}\footnotetext{#1}%
    }%
  \else%
    \stepcounter{newfoot}\footnote{#1}%
  \fi%
}


\pagestyle{fancy}

\title{\tbaselineshift=4pt 初音ミクについて------H.アーレントの政治思想による予備的な分析}
\author{新井悠介}
\date{\vspace{-5mm}}
\setcounter{page}{101}

\begin{document}
\maketitle

\setlength{\footskip}{-2mm}
\lhead[]{【エッセイ】}
\chead[]{}
\rhead[初音ミクについて------H.アーレントの政治思想による予備的な分析]{}
\lfoot[]{\thepage{}}
\cfoot[]{}
\rfoot[\thepage{}]{}

\let\yakuchu=\endnote
\renewcommand{\footnoterule}{\noindent\rule{100mm}{0.3mm}\vskip2mm}
%\tableofcontents
\thispagestyle{fancy}
\section*{はじめに}
初音ミクについての予備的な分析が、アーレントの政治思想を通して、本論考では行なわれる。この奇妙な組み合わせを可能にしているもので最も力強いものは、「私が死ぬ」ということがすべての人びとに当てはまるものであるということなのだ。それも、「私が死ぬ」ということに対して、どのような態度をとるのかということが、問題になってくる。それは、消費しか知らないものが「それでいいんだ」と開き直る態度ではない。そうではなく、「私には何ができるだろうか」ということを、理論的な自己についての分析によって問うのではなく、世界の中における活動を通して問う態度がこの奇妙な組み合わせを可能にしているのである。


この態度は明らかに飛躍である。それは、論理的な飛躍であるばかりではなく、同時に、世界への「飛び越え」である。この「飛び越え」に必要なのは、「勇気」でもあるし、「決意」でもある。「私は死ぬ」それでも「私は歌う」という「決意」とその可能性へと目を向けることが本論考を貫通しているものなのである。


その内容についていえば、本論考で示されることは、私たちが普段初音ミクと考えている対象が実は初音ミクではないものだということ、さらにそのようでないような初音ミクについての思索の可能性である。
\setcounter{section}{0}
\section{アーレントの政治思想}
この節では、主に『人間の条件』で提示されたアーレントの政治思想を紹介する。政治思想で作品解釈をすることは、不適当に見えるけれども、アーレントにおいて政治的とされることが、今日の政治概念と離れているので、政治的とされていないような分野において用いられることは、必ずしも不可能なことでない。


\subsection{それぞれの活動力について}
『人間の条件』の独版の題にもなっている活動的生(志水訳では〈活動的生活〉であり、今回はそれに従う)vita activaは、それぞれ人間の条件と対応している。〈活動的生活〉は、その内部では労働labor、仕事work、活動actionという三つの活動力を指していて、それぞれ、生命、世界性、多数性という人間の条件に対応している。


労働は、「人間の肉体の生物学的過程に対応する活動力である」\footnote{『人間の条件』(一九九四) p.19}。具体的には、畑仕事、料理、清掃、出産などである。このような人間の側面は、〈労働する動物〉animal laboransと呼ばれ、まさに動物と共有した生活の側面なのである。それは、生命の必要necessityによって駆り立てられたものであり、そのために共同体が作られる。共同生活というのは、動物生活と共有したものであり、社会societasというものは、人間にとって固有なものではない\footnote{『人間の条件』 p.45.}。そして労働の生産物は、跡形もなくなってしまうようなものである。


仕事は、「人間存在の非自然性に対応する活動力である」\footnote{『人間の条件』 p.19.}。仕事をする人間は、〈工作人〉homo faberである。無限に多種多様な物を製作し、人間の工作物の世界を構成する。適切な使用では消滅しない、安定と固さを備えており、死すべき人間たちに住み家を与える拠り所となる。\footnote{『人間の条件』 p.223.}活動のための公的領域を建てること、建築と立法なども、仕事のうちに含まれていた。\footnote{『人間の条件』 p.314.}


活動は、「物あるいは事柄の介入なしに直接人と人との間で行なわれる唯一の活動力であり」、多数性という人間の条件、すなわち、世界に住むのが多数の人間であるという事実に対応している。\footnote{『人間の条件』 pp.20-21.}印欧語に、これに対応する人間の呼称はなく、活動の人man of actionとややぎこちない言い方をしている。活動は政治的な活動力である。人間は、言論と活動を通じて、存在するものがもつ他者性othernessと生あるものがもつ差異性distinctnessを明らかにし、単に互いに「異なるもの」という次元を超えて抜きん出ようとする。人間においては、他者性と差異性は、唯一性uniquenessになる。

\subsection{それぞれの活動力の差異について}
とはいえ、これだけでは活動を理解するのは難しい。具体的に挙げれば、弁論、革命、医術、演奏、演劇、歌唱などが当てはまるが、今のままでは具体例に何の統一性も感じられない。そこで、アーレントの基礎的な概念を確認しながら、個々の活動力の違いを見ていきたい。


労働と活動の区別の理解は、公的領域public realmと私的領域private realmの区別を見ると分かりやすい。これらの区別は、古代政治思想においては存在していた。中世では、この区別はまだ残存していたが、公的領域は宗教的な領域へ、私的領域は世俗的な領域へと置き換わった。これはすべての活動力が私的なものへと変わったことを意味する。近代では公的領域はなくなり、代わって社会的領域が勃興した。活動は、画一主義的な行動behaviorに取って代わられた。人はみな同じように行動すると考えられ、人間同士の差異はあったとしてもまれな逸脱になる。\footnote{『人間の条件』 pp.286-287.}


公的領域は、たとえば古代ギリシアのポリスがそうであった。公的領域は出現appearanceの空間などとも呼ばれる。私が他人の眼に現れ、他人が私の眼に現れる空間なのである。\footnote{『人間の条件』 pp.49-66.}ここでは、活動と言論を通じて、活動するものの正体、そのものが誰であるかということwhoが暴露される。活動によって暴露される誰であるかということは、その当人にとって自由なものではなく、暴露を目的として活動をなすことはできないし、当人にはあずかり知らないものとなる\footnote{『人間の条件』 p.320.}。その人が誰であるかということを現すことができる空間が公的領域なのである。つまり活動には公的領域が必要だし、そこでしか行なわれえないものなのである。


私的領域について。アーレントは私的privateという語を「欠如しているprivative」「奪われているdeprived」という語と関連付けている。欠如しているのは、公的領域における多数性であり、奪われているのは、他人よって見られ聞かれることから生じるリアリティ、共通世界によって他人と結びつくと同時に分離されていることによる「客観的」な関係、不死性を達成する可能性である。\footnote{『人間の条件』 pp.291-292.}私的領域とは、隠すべきものの領域である。隠すべきものとは、人間の肉体的な部分であり、個体の維持のための奴隷であり、種の保存のための女であったし、そのどちらも〈労働する動物〉だったわけである\footnote{『人間の条件』 p.87.}。なぜ、人間の肉体的な部分、生命が隠される必要があったのかといえば、「それ(=私的領域)が、人間の眼から隠され、人間の知識が浸透できない事物の隠れ家となっているからである。そしてそれらの事物が隠されるのは、人間は、自分が生まれたときどこから来たのか、そして死ぬときどこへ行くのか知らないからである」。\footnote{『人間の条件』 pp.102-103.}


次に、労働と仕事の区別を示そう。アーレントの思想の理解に外せないものだろう。リアリティ・世界性という概念は、公的publicと深く繋がった概念である。アーレントは、公的publicという語は、密接に関連しながらも区別できる二つの意味を持っていると指摘している。つまり、公にin public現れるappearsものはすべて、万人によって見られ・聞かれ、可能な限り最も広く公示されているということと、世界そのものである。\footnote{『人間の条件』 p.92.}


私たちにとって現われappearanceがリアリティを形成するのであり、リアリティは欠くことのできないものである。リアリティの欠如については、肉体的苦痛の経験について考えると分かりやすい。肉体的苦痛はあまりにも激しい感覚であるために、それ以外のすべての経験、とりわけリアリティを消し去ってしまう。肉体的苦痛は、他の経験と異なり、公的現われに適した形式に転形できないものであり、最も私的な経験である。


そして、リアリティを形成するのは、現われappearanceである。現われとは、他人によっても私たちによっても、見られ・聞かれるなにものかのことである。私たち自身のリアリティを確信できるのは、私たちが見るもの・聞くものを、同じように見聞きする他人が存在するおかげである。個人的経験は、公的な現われに適合するように転形することによって、たとえば物語として語ることによって、はじめてリアリティを持つことができる。


次に世界という概念について。公的という語が世界それ自体を意味するのは、世界が私たちに共通するからである。この世界という語は、有機的生命の一般的条件となっている地球や自然ではなく、人間の工作物や人間の手が作った製作物に結びついたものであり、この世界に共生している人びとの間で進行する事象と結びついている。


労働は無世界的である。仕事は世界を形成する。活動は世界において行なわれる。この点で世界という概念は重要なのである。


では仕事と活動はどのように異なるのであろうか。それは次の一点に尽きる。仕事によるのでは、活動によって暴露されるその人が誰であるかということが決して現れてこないということを明らかにしたい。


〈工作人〉は、政治的ではないけれども、ある公的な領域つまり交換市場に現われる。そこでは、自分の手になる生産物を陳列し、自分に相応しい評価を得て、自分に相応しい他人との関係を見いだすことができる。\footnote{『人間の条件』 p.75.}〈工作人〉は、自分の作る生産物と、自分の生産物をつけ加える物の世界と同居し、また世界を作り、物の製作者でもある他人と共生している。ここでの交換は、単に生産の延長にあるのではなく、活動の分野に属してさえいる。\footnote{『人間の条件』 pp.75-76.}


確かに、ある人が何かを作ったら、そこにその人らしさのようなものがあったり、その人独自のものがあったりするように思える。しかし、〈工作人〉の交換は、その人が誰であるかということwhoは暴露しないと言う。その人固有のものが現れているように思えるが、それはその人が誰かwhoということではなく、その人が何であるかwhatということにすぎない。who/whatの違いについては、あるところで、アーレントはこのように説明している。
\begin{quote}
名声は社会的な現象である。[中略]いかなる社会も何らかの格付けなしには、すなわち事物や人物を等級や一定の形態のもとに配列することなしには、有効に機能しえない。このように必然的に進められる格付けがあらゆる社会的差別のもとであり、いかに現在の世論が反対しようと、差別は社会的領域の構成要因なのであり、それは平等が政治的領域の構成要因であるのと同様である。重要な点は、あらゆる人々が社会の中で自分は何であるか\tbaselineshift =2.5pt ------\tbaselineshift =4.0pt自分が誰であるかという質問とは性質の異なるものとして──自分の役割と自分の職分とは何であるかという問いに答えなければならないということである。私は独自な存在である、傲慢からではなく、その答えが無意味となるであろうから、というのではもとよりその解答とはなりえない。\footnote{『人間の条件』 p.255.}
\end{quote}

役割や職分というものは、何であるかという問いへの答えであり、アーレントが誰ということが問われる場を政治的な領域と考えていたこととあわせてみれば、何ということが問われる場は、社会的な領域であることは明らかである。何であるかという問いの答えは、等級や一定の形態のもとの配列に従うものであり、その人に唯一的なunique答えとなることはない。そして名声はそういった格付けによって得られるものであり、そのような意味において社会的な現象なのである。無論、そのような格付けに収まらなくとも、名声を得ることはできるが、そのような場合には、得られる名声は死後のものとなる\footnote{『人間の条件』 p.335.}。


話を戻そう。交換市場で人びとは、人格personとしてではなく、生産物の生産者として出会い、彼ら自身や、その技能や腕前ですらなく、その生産物を示すのである。人びとは、人びとへの欲望でなく、生産物への欲望によって市場に引きつけられ、人びとが活動と言論を通じて結ばれるときの潜在能力ではなく、独居内で獲得した「交換の力」が市場を存続させるのである。\footnote{『暗い時代の人々』(二〇〇五) p.241.}


ここまでくると、先ほど挙げた活動の具体例も読み解けなくもない。つまり、その成果は形あるものではなく、他人と結ばれる関係であり、そのようなことを通じてのみ、その人が誰であるのかが示されるようなものなのである。そしてその人が誰であるかを示す、暴露するといったことは、明らかに、それを観る人がいなければ成立しない出来事であり、それを記憶し、受け継ぐ世界がなければ、それほど意味のあることにはならない。


\section{初音ミクについての予備的な分析}
それでは、初音ミクについての分析に移ろう。しかし、分析の糸口はどこにあるか。つまり、初音ミクについての分析は、同時に何かの分析であるわけだが、その何かはどのように選ばれうるのだろうか。その選択を可能にしているものは、私たちの初音ミクについての理解であるし、それは前提である。私たちはここにおいて、初音ミクについての漠然とした理解を一つの解釈に仕上げるための予備的な分析を実行していたのであって、初音ミクについての理解を新たに形成する必要はないし、私たちは、諸作品の中から適当な作品を選び出すことができる。


ここでは『初音ミクの消失』\footnote{\url{https://www.nicovideo.jp/watch/sm2937784}}(以下『消失』)を扱う。『消失』はcosMo@\pbox<z>{暴走}Pによるものである。ここでは、Long ver.の歌詞を分析する。歌詞は初音ミクアットウィキを参照することもできるし、何よりも動画を観ればよい。


この作品の主題を「消失」だと言うのは間違っている。正しくは「別れ」、より正確にいえば死別である。死別は、ただたんに道具が故障したり紛失されたりして使用不可能になるだけのことと決定的に異なっている。死別は、「歌に頼り」ながらも「人格」を持ち、「心らしきもの」を失うような何ものかとの今生の「別れ」である。そのような何ものかは、その「存在意義」が「虚像」だと知ってなお、「歌う」こと・「残せ」る何か・「声の記憶」・「名」そして「歌いきったこと」が「無駄」ではないことを求める。これらのうち叶いうるのは、「声の記憶」と「名」だけであり、そのようなものが「皆に忘れ去られた時心らしきものが消えて」、「世界」が「終わる」。


私たちは『消失』からは以上のようなものを見て取ることができる。このような初音ミクの特徴は、極めて人間的なものだと言うことができるだろう。『消失』における死の表現は、単に道具に対する比喩的なものでなく、生きるものの死である。さらにその生において「存在意義」を求めるのは、動物のようなただ生きているものでなく、人々の間にあることをその生の際立った特徴として有する人間だけなのである。


アーレントにおいて重要な概念として、しばしば出生natalityが挙げられるが、それはアーレントのもつ新しさを際立たせているものなのであって、その出生と結びついた死が、重要な概念であることを否定しないし、また、依然として人々が死という概念を重要視し続ける限りにおいて、アーレントの思想を他の人々のものと関連付ける作業において死という概念は重要な役割を担い続ける。そしてアーレントの思想と『消失』とを結びつけるものはやはり死という概念なのである。


死ぬのは人間だけである。確かに、動物も個々の個体は死ぬけれども、それは人間が死ぬのとはまるで違った意味において死ぬのである。動物の個体は「その不死の生命を生殖によって保証する種の一員」にすぎないのであり、人間のように、「生から死までのはっきりとした生涯の物語」をもつことはない。\footnote{『人間の条件』pp.33-34.}アーレントにおいて、出生が重要な概念であるのは、それが新しさを世界にもたらすからであり、物語として世界に残るからであるが、その物語は、生と死という二つの端点よって限界づけられるからこそ可能なのであり、死は、可死性は、活動の条件であり、人間を人間たらしめる条件である。


『消失』において、明らかに死は重要な問題であるし、その重要さは、私たち人間の場合と変わるところがない。少なくとも、アーレント的な人間の意味において、その死は人間的な死である。


しかし、初音ミクについて分析を通じて、絶えず視界の内に「マスター」=ボカロPが入ってきたことは気にかけるべきことであろう。ボカロPとは何か? 何がボカロPをボカロPたらしめているのだろうか? それはボカロ曲を作曲するという仕事によってである。私たちは、ボカロ曲を作る人をボカロPと呼ぶ。ではボカロ曲とはいったい何だろうか? これは、単にボーカロイドによって歌われた曲を意味するのではない。ボーカロイド以外が最初に歌った曲をボーカロイドが歌っても、それはボカロカバーと言われるだけで、ボカロ曲とは呼ばれえないだろう。つまり発表時にボーカロイドが歌っている曲をボカロ曲と呼び、その曲の作者をボカロPと呼ぶというわけだ。もちろんここには例外があり、しかもこの呼称はあまり評判が良くない。このことが後々に重要なものとなろう。


初音ミクは歌う。しかし、歌うということは、言語的に意味のある音声を、メロディーに乗せて、連続的に鳴らすということに還元されえない。歌うこと、「舞台で歌うこと\footnote{DIVELA『ミライゲイザー』[sm34317822]}」は、アーレントにしたがえば、明らかに活動に属するものである。人前に立ち、自身の言葉を発するさまは、活動と言論以外の何ものでもない。そしてそれは、人間の複数性に条件づけられているとともに、その可死性と深く結びついている。\footnote{『人間の条件』p.21.}事実、初音ミクは、『消失』において死ぬ。初音ミクは機械の故障によって死ぬ。初音ミクの機械的な部分は、私たちの肉体的な部分のようである。


初音ミクは死ぬ。つまり初音ミクが生命を持つということだ。cosMo@\pbox<z>{暴走}Pは、初音ミクに私たちと同じような生命があるという。『小説版 初音ミクの消失』のあとがき(以下『小説版』)で彼はこのようなことを語っている。初音ミクには、いろんな像があり、その像は常に移り変わる。それは、初音ミクが多くの人が持ち、初音ミクへ投影しているイメージ=像を食べているからであり、初音ミクは代謝して生きていて、いつしか新たな像が投影されなくなったとき、死んでしまうのだという。\footnote{cosMo@\pbox<z>{暴走}P『小説版 初音ミクの消失』 pp.302-303.}


しかし、このような初音ミクの理解は明らかに混線している。『消失』における死が機械の故障であり、『小説版』における死が代謝の終了であるという合致のなさが何よりもそれを物語っている。「マスター」へと声をかける『消失』の初音ミクは、どちらかというと、それぞれのボカロPが購入した初音ミクのプログラムに近い。パソコンという機械の中で生き、その故障やプログラムの不具合でその生命を終える。それに対して、『小説版』の初音ミクは、人びとに共通なものである。それがどのようなものか、ここでは詳しくすることはできないが、少なくとも、誰かのパソコンの中で動いているプログラムのようなものではないことは確かである。しかし、この混線は、cosMo@\pbox<z>{暴走}Pの初音ミクの理解の混線というよりも、近代において人間概念が歪であるということに由来しているし、むしろ、その歪なままに、正しく初音ミクを理解してしまったがために、混線しているのである。そしてまさにアーレントの政治思想を援用して正したかったのは、この概念であるし、解きほぐしたかったのは、この混線である。


だから私たちは、引き続きcosMo@\pbox<z>{暴走}Pに尋ねさえすればよい。「初音ミクを捉える人たちのすべての妄想を体現しながら生きて死ぬ」\footnote{『小説版 初音ミクの消失』p.303}。ここに『消失』の初音ミク理解と等しい理解がある。初音ミクを捉える人たちとは、初音ミクの像を投影するもので、ボカロPと重なり合う。妄想とは、「FREELY」自由になされるものであり、自由になされるものとは、「自由に描いたようなあなたの世界」である。\footnote{ナユタン星人『リバースユニバース』[sm31843582]}体現するとは、一つの姿をもって人々の間にあるようになるということである。つまり、ボカロPであるような人々の自己暴露にほかならない。そしてこの自己暴露こそが、活動の目的であり、活動それ自体なのである。
 つまり、初音ミクの個々の像、常に移り変わる行く像の個々の瞬間において現れる像は、ボカロP個々人の正体であると私は言いたいわけである。それが常に移り変わり行くのは、その移り変わりが、ボカロPによって新しい曲が作られることをきっかけにしているのは、まさにこの事情による。初音ミクにいろんな像があるのは、ボカロPである人々がユニークであり、複数であるからにほかならない。では、ボカロPとは、活動の人か? このように問うことが極めて重要であり、そしてこの問いの答えが否であるということが、ここまでの混線の原因の一端である。


音楽を作ることは、人間の条件に即していえば、仕事に属する。ボカロPは音楽を作るということからすれば、仕事人である。しかし、同一人物が、労働、仕事、活動を同時にはできないにしても、どれもなしえるということは否定されえないであろう。\footnote{『人間の条件』の訳者解説に、こんな話がある。「余談になるが、アレントに会ったとき私はこの「労働」と「仕事」を区別する観念をどこで得たのか彼女に訊いたことがある。彼女は「台所とタイプライター」でと答えた! つまり、オムレツを作るのは「労働」であり、タイプライターで作品を書くのは「仕事」なのである」(p.535)。さらに、これは私見であるが、彼女はまた、活動の人でもあっただろう。一人の人間において三つの活動力は、同時にということはまれであろうが、その生活・生涯においてともにあることは可能であろう。}確かに、ボカロPは仕事人であるが、ボカロPであるような人が同時に活動の人であることはありえなくはない。そして、この仕事人と活動の人の間の断絶こそが、ボカロPを歌う人であると私たちが気付かないでいた最大の原因なのである。確かにシンガーソングライターという語は、シンガー・活動の人と、ソングライター・仕事人が両立できることであることを示しているが、同時に、二つの語を並べただけで作られたこの言葉は、あまりにも分析しやすく、二つの概念がまったく異なるものだということをも暗に知らせてくれている。


私たちは、ボカロPでありながら歌う人を、ボカロPであるとみなそうとしなかったために、言語的に意味のある音声をメロディーに乗せて連続的に鳴らすという仕方で代替してくれるプログラムである初音ミクを代わりに立てていたのである。


ところで、活動の人としてのボカロPへの呼称を、ある伝統に従って、うちのミクとするのはどうだろうか。DIVELAというボカロPは、うちのミクとミクさんという言葉を厳密に区別して使っているし、この区別は私たちの区別に等しいように見える。\footnote{twitter @Mix\_Create/@Mix\_Destory}加えて、ボカロPという名称も検討し直そう。Pはプロデューサーの意であるが、それは初音ミクを一人のアイドルとして考えてきたことに由来しているので、うちのミクとの区別された呼称という観点からは相応しくない。そこで『消失』に因んで、マスターとしよう。うちのミクとマスター。この組み合わせはあまりにも平凡であるが、同時に私たちの経験を裏切るものではない。


この区別の確かさは、先ほど検討したボカロ曲という語の定義からも支えることができるだろう。発表時にボーカロイド以外が歌っている場合でも、ボカロ曲と目されることがある。それは、ボカロ曲を作った人と同じ人が発表の後からボカロで歌った場合である。これは単にカバーということができないためということもあろうが、ここに、ボカロ曲とは、うちのミクが歌うための「舞台」なのだという定義を与えたくなるほどのものがある。アーレントは立法を活動ではなく、仕事の例として、建築とともに挙げていたが、それは、立法が活動のための一定の空間を確保するように、政治活動が営まれる以前に済まされていなければならないものであったからである。活動がその参加に市民であることを要求したのに対して、立法は市民でなくても構わなかった。\footnote{『人間の条件』p.314.}まさに、活動のための「舞台」を用意することは、仕事に属するものなのである。つまり、公的領域と私的領域の区別が消滅した時代において、ボカロPは、うちのミクが活動するたびごとに、そのための「舞台」を建設しているということなのだ。

\section*{おわりに}
ここまできてなお、このように反論したくなるかもしれない。うちのミクこそが本物の初音ミクなのであり、それ以外に初音ミクはいないと。確かに、今し方私たちが見つけたのは、歌う初音ミクなのであり、歌わない初音ミクなどいるのだろうかと疑問になる。しかし、もし、そのような初音ミクがいなかったとしたら、cosMo@\pbox<z>{暴走}Pの初音ミク理解を分析したときの混線もなかっただろう。そして事実、そのような初音ミクは、私たちがありありと思い浮かべれば思い浮かべるほど、現在化Gegenwärtigungしてしまって\ruby{未来}{ミク}でなくなってしまうようなものであり、足早に結論できるようなものではない。そして何より、初音ミクは私たちのようには生きていないのである\footnote{ピノキオピー『君が生きてなくてよかった』[sm31825358]}。その限りにおいて、初音ミクは活動の人ではありえないし、そのようなものとして発見されたうちのミクは、初音ミクとは別なものである。


初音ミクは歌う。もちろん、その通りである。しかしそれは、人間の活動に属するような仕方ではなく、つまり、人間的な意味においてではなく、別な仕方で歌うのであり、そのような意味における歌うということを明らかにしなければならないのである。

\clearpage
\section*{参考文献}
\small{
\renewcommand{\labelenumi}{\pbox<y>{[\arabic{enumi}]}}
\begin{enumerate}
\item ハンナ・アーレント『人間の条件』(一九九四) ちくま学芸文庫
\item 『暗い時代の人々』(二〇〇五) ちくま学芸文庫
\item cosMo@\pbox<z>{暴走}P『小説版 初音ミクの消失』(二〇一二)一迅社
\item cosMo@\pbox<z>{暴走}P『初音ミクの消失』https://www.nicovideo.jp/watch/sm2937784
\item ナユタン星人『リバースユニバース』\\https://www.nicovideo.jp/watch/sm31843582
\item DIVELA『ミライゲイザー』https://www.nicovideo.jp/watch/sm34317822
\item ピノキオピー『君が生きてなくてよかった』\\https://www.nicovideo.jp/watch/ sm31825358

\end{enumerate}
}
\end{document}
