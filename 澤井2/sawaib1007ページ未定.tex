%\documentclass[10pt,a4j]{utjarticle}
\documentclass[b5j,twoside,twocolumn]{utarticle}
%\documentclass[b5j,twoside]{utarticle}
%\documentclass[b5j,twoside,twocolumn]{utbook}
\setlength{\columnsep}{2zw}
\usepackage{bxpapersize}
\usepackage{pxrubrica}
\rubysetup{<hj>}
\usepackage{endnotes}
\usepackage{multicol}
\usepackage{plext}
\renewcommand{\theendnote}{[後注\arabic{endnote}]}
\renewcommand{\thefootnote}{\arabic{footnote}}
\usepackage{pxftnright}
\usepackage{fancyhdr}
\setlength{\topmargin}{5mm} % ページ上部余白の設定(182mm x 257mmから計算)。
\addtolength{\topmargin}{-1in} % 初期設定の1インチ分を引いておく。
\setlength{\oddsidemargin}{21mm} % 同、奇数ページ左。
\addtolength{\oddsidemargin}{-1in}
\setlength{\evensidemargin}{17mm} % 同、偶数ページ左。
\addtolength{\evensidemargin}{-1in}
\setlength{\footskip}{-5mm}
%\setlength{\marginparwidth}{23mm}
%\setlength{\marginparsep}{5mm}
\setlength{\textwidth}{225mm} % 文書領域の幅(上下)。縦書と横書でパラメータ(width / height)の向きが変わる。
%\setlength{\textheight}{150mm} % 文書領域の幅(左右)
\makeatletter
\def\@cite#1#2{\rensuji{[{#1\if@tempswa , #2\fi}]}}%%
\def\@biblabel#1{\rensuji{[#1]}}%%%
\makeatother
\usepackage{enumerate}
\usepackage{braket}
\usepackage{kyakuchu}
\usepackage{url}
\usepackage[dvipdfmx]{graphicx}
\usepackage{float}
\usepackage{amsmath,amssymb}
\newcommand{\relmiddle}[1]{\mathrel{}\middle#1\mathrel{}}
\usepackage{ascmac}
\usepackage{okumacro}
\usepackage{marginnote}
%\usepackage[top=15truemm,bottom=15truemm,left=20truemm,right=20truemm]{geometry}
\usepackage{cleveref}
\usepackage{plext}
\usepackage{pxrubrica}
\usepackage{amsmath}
\usepackage{fancybox}
\usepackage[dvipdfmx]{graphicx}
\usepackage{cancel}
\setcounter{tocdepth}{3}
%\renewcommand{\thesection}{\Roman{section}}
%\renewcommand{\thesubsection}{\thesection--\Roman{subsection}} 
%\renewcommand{\labelenumi}{(\Alph{enumi})}
\usepackage {scalefnt}
\makeatletter
\@definecounter{yakuchu}
\@addtoreset{yakuchu}{document}% <--- depende on class file
\def\yakuchu{%
\@ifnextchar[\@xfootnote %]
{\stepcounter{yakuchu}%
\protected@xdef\@thefnmark{\theyakuchu}%
\@footnotemark\@footnotetext}}
\def\yakuchutext{%
\@ifnextchar [\@xfootnotenext %]
{\protected@xdef\@thefnmark{\theyakuchu}%
\@footnotetext}}
\def\yakuchumark{%
\@ifnextchar[\@xfootnotemark %]
{\stepcounter{yakuchu}%
\protected@xdef\@thefnmark{\theyakuchu}%
\@footnotemark}}
\makeatother


\AtBeginDocument{\tbaselineshift =2.0pt}
\usepackage{atbegshi,etoolbox}

\newcounter{newfoot}
\patchcmd{\footnotetext}{\thempfn}{\thenewfoot}{}{}

\newcommand{\evenfootnote}[1]{%
  \ifodd\value{page}%
    \footnotemark%
    \AtBeginShipoutNext{%
      \stepcounter{newfoot}\footnotetext{#1}%
    }%
  \else%
    \stepcounter{newfoot}\footnote{#1}%
  \fi%
}


\pagestyle{fancy}
\title{\tbaselineshift =4.0pt 野原しんのすけ------『クレヨンしんちゃん』における五歳の超人}
\author{澤井優花}
\date{\vspace{-5mm}}
\setcounter{page}{101}

\begin{document}
\maketitle

\setlength{\footskip}{-2mm}
\lhead[]{【エッセイ】}
\chead[]{}
\rhead[野原しんのすけ------『クレヨンしんちゃん』における五歳の超人]{}
\lfoot[]{\thepage{}}
\cfoot[]{}
\rfoot[\thepage{}]{}

\let\yakuchu=\endnote
\renewcommand{\footnoterule}{\noindent\rule{100mm}{0.3mm}\vskip2mm}
%\tableofcontents
\thispagestyle{fancy}
\section{はじめに}
野原しんのすけという五歳児をご存じだろうか。彼は漫画「クレヨンしんちゃん」の主人公である。いつも独自の言葉遣いをし、少し変わった振る舞いをする。例えば「ケツだけ星人」というお尻を使った高速移動や、「半ケツフラダンス」という独自のフラダンスをする。それらは周囲の大人たちや幼稚園の友人を困らせ、戸惑わせるが、その振る舞いは明らかに超人のものである。


超人とは哲学者フリードリヒ・ニーチェの思想であり、これは「永劫回帰」と並ぶニヒリズム克服のための柱となる思想である。ニーチェは「神は死んだ」という有名な言葉の通り、キリスト教における神は死んだのだと宣告する。神が死んだ世界では従来の道徳は意味をなさず、一切は無意味となる。これがニヒリズムである。このニヒリズムを克服するには、我々はこれまでの道徳の一切を否定し、自らの価値観に従い行動し、人生を意味づけしてゆく超人とならねばならない。これが「超人」思想である。


周囲の視線を気にすることなく、自分独自の振る舞いをするしんのすけは、トラブルメーカーであると同時に超人なのである。しかしながらここで一つの疑問が生じる。しんのすけは自らの価値観に従っているという点で超人であるが、しんのすけは自身の生活に意味づけをしているというよりむしろ、自らのマイペースな行動で全てを破壊しているように思われる。


先に述べたようにニーチェの言う超人において、超人は自らの価値観に従い人生に意味づけしてゆくことが求められる。しかし実際に超人的に振る舞うこととは周囲にどのような影響を与え、超人はどのような人物像として見られるのだろうか。しんのすけのように全てを破壊し、周囲に迷惑をかけることも、一つの超人像なのだろうか。本稿ではニーチェの超人思想に至るまでの議論を明らかにしながら、野原しんのすけという超人とはいかなる存在であるのかをより鮮明に描写することを目標とする。2章においてはニーチェの思想の用語であるルサンチマン・ニヒリズム・永劫回帰を簡単に説明し、3章においては超人思想を理解する上で重要な『ツァラトゥストラはかく語りき』における議論を明らかにする。


\section{神は死んだ・ルサンチマン・ニヒリズム}
以下ニーチェの超人思想に至るまでの議論を見ていくこととする。まず本章においてはニーチェの思想における用語「ルサンチマン」「ニヒリズム」について説明する。ニーチェによれば、神は死んだのである。神とは特にキリスト教における神を指し、ニーチェが生きた時代、キリスト教を信じているということは人々のアイデンティティの中核をなしていたと言っても過言ではない。しかしながらニーチェはこの神の存在を否定した。ニーチェは神の存在を否定するにあたり、まずルサンチマンという概念について説明する。ルサンチマンとは弱者が強者に対して抱く感情のことを言う。例えば経済的に困窮したものが、贅沢する強者を、贅沢しているのが妬ましいという感情によって悪であるとみなす。そして弱者のほうが善なのである。ニーチェはルサンチマンの道徳を奴隷道徳と、貴族のもつ道徳を貴族道徳と呼び以下のように述べる。
\begin{quote}
すべての貴族道徳は自己自身にたいする勝ち誇れる肯定から生まれでるのに反し、奴隷道徳は初めからして〈外のもの〉・〈他のもの〉・〈自己ならぬもの〉にたいし否と言う。つまりこの否定こそが、それの創造的行為なのだ。価値を定める眼差しのこの逆転―自己自身に立ち戻るのでなしに外へと向かうこの必然的な方向―こそが、まさにルサンチマン特有のものである\footnote{『善悪の彼岸 道徳の系譜』}。
\end{quote}


つまりニーチェによれば、弱者が自らのみすぼらしさを善であると肯定するために、まず強者を悪であると否定することこそがルサンチマンの特徴であり、このような奴隷道徳はキリストの信仰によって起こる価値の反転である。みすぼらしい生活をする弱者が善であると言うには、キリストが必要となる。キリストを信じ、倹しい生活をするものは来世に救われるから、まさにこの理由で弱者は善であるとみなされることが出来るのである。よってキリストは人為的に生み出されたものであり、これは誤りである。ゆえに神は死んだのであり、これまでの善悪の指標はもはや成り立たない。一切は無意味なのである。このように一切が無意味であることを、ニーチェはニヒリズムと呼んだ。
これまでニーチェの「ルサンチマン」「ニヒリズム」の概念を説明し、神の存在を否定したニーチェの議論を明らかにした。ニーチェによればキリスト教に由来する善悪は、ルサンチマンの感情によって起こった価値の反転であり、そもそもキリストはルサンチマンによって作り出されたものなのであった。ゆえに神は存在しないのである。神が死んだ世界において従来の善悪の指標は成り立たず一切は無意味となる。このようにこの世の一切が無意味であるという考え方をニーチェはニヒリズムという。続いて次節では、ニヒリズムを克服するための柱となる思想の一つである「永劫回帰」について説明する。その後、次章において「超人」についての説明を行う。%、「永劫回帰」と「超人」思想について説明する。
\subsection{永劫回帰}
ニヒリズムに陥った我々はこんな状況に陥る。例えばこれまでのキリスト教的な善悪の価値観に従えば、現世がどれだけ辛いものであったとしても来世は救われると信じることが出来る。また苦しい試練があったとしても、それを達成する事を目標とし、耐えることができる。けれど神の存在が否定されてしまえば、来世に救われることもなければ、何か目標設定する事さえ何の意味もなさない。人生は常に前起こったことの繰り返しで、終わりがないのである。このようにすべてが回帰することを、ニーチェは「永劫回帰」と呼んだ\footnote{『ニーチェ入門』pp.155-159}。
この永劫回帰と、「超人」思想は初めに述べたようにニヒリズムを克服する二本柱である。すべてが無意味の人生において、我々はいかにして生きることができるのだろうか。

\section{超人思想}
ニーチェによればニヒリズムを克服するために永劫回帰と超人になることが必要である。以下では永劫回帰と超人思想が、そしてこれまで説明したルサンチマンやニヒリズムといった概念がどのように関連しているのかを明らかにする。またニーチェは永劫回帰と超人について『ツァラトゥストラはかく語りき』において重要な説明を行っており、本章ではこの小説を元に議論を行う。
\subsection{自己の没落と超克}
主人公のツァラトゥストラは民衆に演説する中で、神は死んだのだと宣告する。以下ツァラトゥストラの演説を見ていくこととする。


ツァラトゥストラは、民衆に対する演説の中でまず初めに「超人」について説明する。人間は克服されねばならず、しかしながらこれまでの存在とは異なり、今民衆は自らを超克するための創造を行ってはいない。これでは超人からみた人間とは猿以上に猿であり、哄笑の種か恥辱の痛みを覚えさせるものであると。
そして彼によれば、神は死んだのだから、地上を超えた希望を抱くのではなく、大地に忠実であるべきであり、今最も恐るべきことは大地への冒瀆である。
さらに、かつては霊魂が肉体を醜いものとして軽蔑していたが、いまや霊魂こそが貧弱であり不潔であり、みじめな安逸である。そして汚れた人間は不潔にならぬために、自己を超克するために、大海とならねばならない。「大いなる軽蔑」は大海へと没するのであり、超人とは大海であるのだ。そして「大いなる軽蔑」とは自らの徳や理性を全否定するとき体験できる最大の自己軽蔑である。


そしてツァラトゥストラは以下のように言う。
\begin{quote}
「人間における偉大なところ、それは彼が橋であって、自己目的ではないということだ。人間において愛さるべきところ、それは、かれが移りゆきであり、没落であるということである。わたしが愛するのは、没落するものとして以外には生きるすべを知らない者たちである。……わたしが愛するのは、自由な精神と自由な心情の持ち主だ。かれの頭脳はたんにかれの心情の臓腑にすぎない。そして心情はかれを没落に駆り立てる。……かれらは稲妻がくることを告知し、告知者として破滅するのである」\footnote{『ツァラトゥストラはこう言った・上』pp.14-21}
\end{quote}


以上のツァラトゥストラによる説明を考慮すると、自己の超克は自己の没落と表裏一体の関係となっていることが明らかである。自己を超克するには、自己の霊魂が不潔であり、自己の徳や理性に対しても最大の軽蔑を体験せねばならない。自己軽蔑を経て自由な精神と自由な心情を持つものは、その心情ゆえに没落へと駆り立てられる。そして軽蔑を大海に没しようとするときに、自己の没落・破滅、自己の超克が同時に起こるのである。
このような自己否定とは、キリスト教的な価値観によって形成された自分の徳や理性・価値基準といったものの全面否定であり、ここで従来の善悪の価値基準は破壊され、超人は自ら自分の価値基準を打ち立てる事になる。


以上を踏まえると、野原しんのすけが超人であるならば、彼は「究極の自己否定の上に自らの価値基準を打ち立てている」ということになる。まだ五歳児であるしんのすけは自ら自己を否定しているよりは、周囲の大人たち、例えば家族である母のみさえや父のひろし、または春日部幼稚園の先生や友人によって否定を受けているように思われる。しかしながら五歳児にして、周囲の人間による否定を受けながらも、その上で独自の振る舞いを行うしんのすけはやはり超人なのである。

\subsection{自己否定の先にある自己超克の意味・永劫回帰}
前節において、自己の没落と自己の超克とは表裏一体の関係にあるのだということが明らかになった。そこでも、周囲の人間に自信を否定されながら、その上でマイペースを築き上げるしんのすけは超人であることが説明された。しかしながらそこで、自己の没落、つまり破滅と自己の超克が本当に同時に起こるのであれば、それにはいかなる意味があるのか。筆者は、究極の自己否定における、生や魂そのものの肯定というものがあるのだと考える。そしてその肯定には「神の死」によってもたらされる「永劫回帰」が深く関連しているのだということを明らかにする。


ニーチェにおいて超人と対比されるものとはルサンチマンである。先に述べたようにルサンチマンとは、弱者が強者に抱くネガティブな感情であり、ニーチェにおけるルサンチマンとは特にキリスト教を信仰する民衆による、彼らの「価値の逆転」を指す。この逆転とは、強いものは悪であり、弱いキリスト教徒は善であるという考えである。そしてこのような考えは、弱者が「こうしか生きられない」という事実に対する反動形成である。そして弱者は自らのみすぼらしさの原因を「過去」にたずね、それが動かしえないことに悔恨をもつのである。つまり時間への復讐である。これは、結局その弱者のみすぼらしさの原因が原罪とされたことからも、キリスト教説が人々を支配したことの証である。\footnote{『ニーチェ入門』p.177}


それでは、超人はどうだろうか。過去や未来といった「時間」について語るツァラトゥストラの言葉を引用する。
\begin{quote}
「わたしが愛するのは、未来の人々を正当化し、過去の人々を救済するものだ。なぜならかれは現在の人々によって破滅しようと欲しているからである」\footnote{『ツァラトゥストラはこう言った・上』p.20}
\end{quote}


この語りが意味することは何か。先に述べたように、ルサンチマンは、自分がみすぼらしいことの原因を「過去」にたずね、その動かしえないことに対し悔恨をもつ。またキリスト教においては、現世を倹しく生き、天国で地上の罪の許しを得られると考えられている。つまりルサンチマンは、人生を一回きりのものとして想定する。\footnote{『ニーチェ入門』p.177}


しかし超人は異なる。超人は自己の没落、自己の超克を経て、従来のキリスト教的価値基準の全てを失うのであった。つまりそのような価値基準のない世界とは、完全に無意味な世界なのである。このニヒリズムによって超人は永劫回帰という時間的概念を得る。永劫回帰とはニーチェによる概念であり、キリスト教の否定された世界においては天国での神による許しなどは存在しないため、世界が全く無意味であるとともに、我々の人生には終わりがないのだというような見方を含蓄するものである。このような永劫回帰に直面した人間はいかにして生きるのか。彼らは過去や未来を「然り」と言うことによって生きるのである。\footnote{『ニーチェ入門』p.178}


つまり現在という時を破滅に導いてまで超人へと成り変わり、自らの価値基準を打ち立てることによって、過去や未来を「そうあることを私は欲したのだ」「このような未来を欲している」と言うことによって生きるのである。筆者はこのように、自らの過去や未来を是認することが自己否定の極致にある生の肯定であると考える。

\subsection{しんのすけにおける生の肯定}
しんのすけのふるまいにおいても、やはりこのような側面は多く見受けられる。例えばアニメ第478話「流れるランチだぞ」\footnote{まとめるにあたって、Wikipediaページ「クレヨンしんちゃん アニメエピソード一覧(2002~2011)」を参考にした。}において、しんのすけは家族に流しそうめんをしたいと訴える。ちょうど日曜で休みだった父ひろしは渋々倉庫にとってあった竹を用いて簡易の流しそうめんの装置を作る。はじめ家族は順調にそうめんを食べていたが、しんのすけが外にでた際にねねちゃん、かざまくん、ぼーちゃん、まさおくんの友人四人に流しそうめんを作ると言ったために、友人が家に押しかけ、家は益々騒がしくなる。みさえの機嫌が悪くなる中、子供たち(特にしんのすけ)は流しそうめんに飽き、そうめんでないものを冷蔵庫の中からとってきて全て流すと、流しそうめん台に野菜や牛乳パックなどがいっぱいになり、崩壊する。流しそうめん台につけた蛇口のホースは外れ、騒がしさから起きた妹のひまわりはホースを振り回し、家を水浸しにする。これらの惨事はほとんどしんのすけが自ら起こしたものであり、全てが破壊されていて何かが解決されたわけでは無い、しかしこのまま話はエンディングとなる。(永劫回帰)しんのすけはこのような惨事に対し落ち込む様子もなければ、むしろ非常に満足気である。このような描写にも見られるように、永劫回帰する野原家において、しんのすけは惨事に対しても、全てはこうあるべきだったのだと、「然り」と肯定しているのである。そしてこのような振る舞いがたった五歳児であるしんのすけに見られることは驚くべきことなのである。


\section{おわりに---しんのすけという超人}
少し長くなったが、超人思想に至るまでのニーチェの議論を明らかにした。
まず2章においてはニーチェの基本的な概念である「ルサンチマン」「ニヒリズム」「永劫回帰」について説明を行った。3章においては2章までの概念説明を踏まえ、超人思想の全体を明らかにした。超人になるには神の死を受け入れ、従来の価値基準や自分自身をすべて否定しなければならない。またルサンチマンと超人の対比においては、ルサンチマンは自らの人生を一回きりと捉え、自らのみすぼらしさの原因を過去に付すが、一方超人は全てが回帰する永劫回帰の中で、すべてがこうあるべきであったのだと、「然り」とすることで自らの生を肯定するのである。超人とは究極の自己否定の先にある自己超克、生の肯定なのである。


そしてこの意味で、野原しんのすけは超人である。
彼もまた、究極の自己否定の上に自らの価値基準を打ち立てている。家族や友人、幼稚園の先生によってなされた否定の上に、しんのすけは自分自身の価値基準を打ち立て、独自の振る舞い・言動を続けるのである。また、野原家という永劫回帰そのものの中においても、起こった悲劇に対し落ち込むことなく、それらは全てこうあるべきだったのだと、「然り」とすることによって、それらを受け入れ、満足を享受しているのである。本稿の目的はニーチェの超人思想に至るまでの議論を明らかにしながら、超人野原しんのすけとはいかなる存在であるのかをより鮮明に描写することであった。野原しんのすけとは、冒頭に述べたように、超人でありトラブルメーカーでもある。しんのすけの振る舞いは超人的でありながら、やはり周囲の人間に対し迷惑をかけていることは否定できない。しかしながら我々が見るしんのすけという超人像とは、ただのトラブルメーカーでないことは明らかであろう。五歳児の無邪気な姿のうちに、全てを然りとする強靭さを秘めているのである。


本稿においては『クレヨンしんちゃん』の主人公である野原しんのすけに焦点を絞り超人像の描写に努めた。しかしながら漫画やアニメを見る限り、超人的な側面を含みもつキャラクターはしんのすけだけには留まらない。筆者はおよそ野原家の全員が超人であると考えている。ゆえに今後は、本稿でのしんのすけという超人像を元に、しんのすけに留まらず『クレヨンしんちゃん』の世界における様々な超人を考察してゆきたい。

\section*{参考文献}
{\small
\renewcommand{\labelenumi}{\pbox<y>{[\arabic{enumi}]}}
\begin{enumerate}
\item ニーチェ著、信太正三訳、ニーチェ全集11『善悪の彼岸 道徳の系譜』、ちくま学芸文庫、二〇一八
\item ニーチェ著、氷上英廣訳、『ツァラトゥストラはこう言った・上』、岩波書店、二〇〇九
\item 竹田青嗣著、『ニーチェ入門』、ちくま新書、二〇一〇
\end{enumerate}
}




\end{document}
