%\documentclass[10pt,a4j]{utjarticle}
\documentclass[b5j,twoside,twocolumn]{utarticle}
%\documentclass[b5j,twoside]{utarticle}
%\documentclass[b5j,twoside,twocolumn]{utbook}
\setlength{\columnsep}{2zw}
\usepackage{bxpapersize}
\usepackage{pxrubrica}
\rubysetup{<hj>}
\usepackage{endnotes}
\usepackage{multicol}
\usepackage{plext}
\renewcommand{\theendnote}{[後注\arabic{endnote}]}
\renewcommand{\thefootnote}{\arabic{footnote}}
\usepackage{pxftnright}
\usepackage{fancyhdr}
\setlength{\topmargin}{5mm} % ページ上部余白の設定(182mm x 257mmから計算)。
\addtolength{\topmargin}{-1in} % 初期設定の1インチ分を引いておく。
\setlength{\oddsidemargin}{14mm} % 同、奇数ページ左。
\addtolength{\oddsidemargin}{-1in}
\setlength{\evensidemargin}{14mm} % 同、偶数ページ左。
\addtolength{\evensidemargin}{-1in}
\setlength{\footskip}{-5mm}
%\setlength{\marginparwidth}{23mm}
%\setlength{\marginparsep}{5mm}
\setlength{\textwidth}{225mm} % 文書領域の幅(上下)。縦書と横書でパラメータ(width / height)の向きが変わる。
\setlength{\textheight}{155mm} % 文書領域の幅(左右)
\makeatletter
\def\@cite#1#2{\rensuji{[{#1\if@tempswa , #2\fi}]}}%%
\def\@biblabel#1{\rensuji{[#1]}}%%%
\makeatother
\usepackage{enumerate}
\usepackage{braket}
\usepackage{url}
\usepackage[dvipdfmx]{graphicx}
\usepackage{float}
\usepackage{amsmath,amssymb}
\newcommand{\relmiddle}[1]{\mathrel{}\middle#1\mathrel{}}
\usepackage{ascmac}
\usepackage{okumacro}
\usepackage{marginnote}
%\usepackage[top=15truemm,bottom=15truemm,left=20truemm,right=20truemm]{geometry}
\usepackage{cleveref}
\usepackage{plext}
\usepackage{pxrubrica}
\usepackage{amsmath}
\usepackage{fancybox}
\usepackage[dvipdfmx]{graphicx}
\usepackage{cancel}
\setcounter{tocdepth}{3}

%\renewcommand{\labelenumi}{(\Alph{enumi})}
\usepackage {scalefnt}
\makeatletter
\@definecounter{yakuchu}
\@addtoreset{yakuchu}{document}% <--- depende on class file
\def\yakuchu{%
\@ifnextchar[\@xfootnote %]
{\stepcounter{yakuchu}%
\protected@xdef\@thefnmark{\theyakuchu}%
\@footnotemark\@footnotetext}}
\def\yakuchutext{%
\@ifnextchar [\@xfootnotenext %]
{\protected@xdef\@thefnmark{\theyakuchu}%
\@footnotetext}}
\def\yakuchumark{%
\@ifnextchar[\@xfootnotemark %]
{\stepcounter{yakuchu}%
\protected@xdef\@thefnmark{\theyakuchu}%
\@footnotemark}}
\makeatother

\usepackage{atbegshi,etoolbox}

\newcounter{newfoot}
\patchcmd{\footnotetext}{\thempfn}{\thenewfoot}{}{}

\newcommand{\evenfootnote}[1]{%
  \ifodd\value{page}%
    \footnotemark%
    \AtBeginShipoutNext{%
      \stepcounter{newfoot}\footnotetext{#1}%
    }%
  \else%
    \stepcounter{newfoot}\footnote{#1}%
  \fi%
}


\pagestyle{fancy}

\title{精神科の先生「うーん。そやねぇ」}
\author{田所}
\date{\vspace{-5mm}}
\setcounter{page}{101}

\begin{document}
\maketitle

\setlength{\footskip}{-2mm}
\lhead[]{【戯曲】}
\chead[]{}
\rhead[精神科の先生「うーん。そやねぇ」]{}
\lfoot[]{\thepage{}}
\cfoot[]{}
\rfoot[\thepage{}]{}

\let\yakuchu=\endnote
\renewcommand{\footnoterule}{\noindent\rule{100mm}{0.3mm}\vskip2mm}
%\tableofcontents
\thispagestyle{fancy}
\begin{description}
\setlength{\leftskip}{-3.0zw}
\setlength{\itemsep}{-0.3zw} % 項目間
\item 私「回すの忘れてました。」
\item 先生「ほいほい。」
\item 私「でも、そうっすねぇ、あーーー︙︙」
\item 先生「いやでもまあ、いつも言うてる、大半の人はそんなに頭良くないん、バカなんだよって言ってる、ところで、まあ典型例としては、やっぱ取り上げざるを得ないところやね。」
\item 私「そうですねぇ。」
\item 先生「うーん。」
\item 私「まぁまぁまぁまぁまぁ、もう、あのー、アクションを起こしてくれなかったら、それでいいよって感じですねぇ。」
\item 先生「うんうん」
\item 私「もうほんとに、ほんとにもう、なんか、次ほんとにおっきな事やってきたらまぁ︙︙・また、考えないといけないんですけど。」
\item 先生「うん、まあ宗教関連の事はもう本人は絶対変わらへんやろうから、まあそこに関してというかそこに触れる形で、関わってきたらまぁあのー、また、断絶してるぞって見せつけるような、」
\item 私「はい」
\item 先生「所作をあえて示さなしゃあないなって思うけど、」
\item 私「うん、いやぁ︙︙」
\item 先生「信じてって言葉もそうやけど、なんか自分にその悪意はないとか、誠実に、良い、良かれと思ってっていうことはあの、アピールしておかなあかんってのはあるわね。」
\item 私「もーう、すごいっすよねぇ。未だになんか、こちら側があたかもある程度の信用・信頼を持ってるかのように振舞ってる︙︙」
\item 先生「ふふふっ」
\item 先生「んー、や、持ってるかっていうか、自分が良い、良ければ、相手が良いと思ってくれて当然という、ことがあるね。︙︙いやぁそれはこう、信仰心みたいなもんやわな。」
\item 私「はい。もう、うん︙︙こう、自我の中にというか自己同一性の中に取り込まれちゃってる、」
\item 先生「うんうん。」
\item 私「しかたがない︙︙」
\item 先生「そうやって生きるのが正しいという風に、あのー、思ってしまったらもう、こうやって、そこで止まりっぱなしやからねぇ。」
\item 私「︙︙いやぁーーー、いやぁ、しょうもないっすねぇほんとに。しょうもない。」
\item 先生「うん、まああの、能力が足りないって切り捨て方もあるけど、まあやっぱり弱いんよね。」
\item 私「うん。」
\item 先生「うーん、そこはもうあの、気の毒やなと思うところもあるわ。」
\item 私「︙︙やーなんで、なんでこう、んーー、分かんねえの、まあまあ、分かんないから分かんないんでしょうけど。ふふっ、やっぱこう、分かるようになった人は分からない人間は分からないし、分からなくなるし、分からない人間って分かることを分からないですよねぇ。︙︙うん、むずかしい︙︙まあ、んーまあ、血縁者に関してはもうなんかほんとに、まあどうでもいいから。」
\item 先生「ふふふっ」
\item 私「まあどうでもいいですけど。」
\item 先生「まあね。まあそこは、別に、血縁だとは言うけれども、お母さんが自分で生きていく方向に、そこと絡み続けないかん道を選ばなければ、全然いいことや。」
\item 私「はい。︙︙そうですねぇ︙︙」
\item 先生「でもやっぱりこうやってあのー、自分のその、弱さとか至らなさとかいうのを、あの、変えないかんとか、あのー賢くならないかん強くならないかんって言う事を、あの自覚、するっていうこと、耐えれる人間と耐えれない人間がいんのよね。」
\item 私「あーー」
\item 先生「耐えれない人間はその、見なくて済む、逃げて済むように、あの、してもらえる理屈があったら飛びつくし、それを取り込んで自分はもうこれでいいって言えてしまう生き方にやっぱなっちゃうよね。なんかあのー、全く、なんて言うん、違うものを持って生まれてきたとは思わへんのよ。田所さん、と例えばお母さんとかね。」
\item 私「あぁ」
\item 先生「でもあのそれがこうー、分かってしまうとか、克服せなあかんって言う風に、背負えてしまうか、あの背負う事をそもそも、拒否し続けたかとか、まあそうとこでどんどん違っていったんやろなという感じがあるよね。」
\item 私「うーん︙︙」
\item 先生「いやもうお母さんが反面教師になったから田所さんはあのーそうはあかんと思って、こういう道に来たのは、あるやろうけどね。」
\item 私「そうですねぇ。母親というかまあ、あの家、全てですねぇ。」
\item 先生「ふふっ。」
\item 私「なんかあの、うーん︙︙」
\item 先生「まあ、まああの、逆説的にあんたらのお陰で今の俺があるという、要素はどっかにはあるー感じやんね。別に感謝する、筋ではないけど。」
\item 私「はい。」
\item 先生「︙︙あのー本当にどうでもいいわ知らんわって言えるように、なるっていうのは、そういうとこでこう、なんか絡まったしんどさを、どうほぐすか、を考えて生きてきたっていう段階を超えないと終わらんのよね。うん。それが過ぎると、あぁそんな人もいるねそんなこともあったねぇで、」
\item 私「ふーん。」
\item 先生「今の自分の、なにも動揺させない、あの、要素やわって言う風に、あのー、距離を変えずに、とか感情的にも反応せずに付き合えるっていう、のはいずれ段階的にはくるかなぁって期待はするけど、まだあの、そこをどうやって終わらせるかとか、自分を、作り上げるときに、これをどう、どういう風に、あの、始末つけるか自分の中でも、あの、正解をつくるかって言う事を、まあやっぱもがいてる最中ってのがあるので、やっぱりあの、揺さぶられるというか、ちょっと意識したときに、自分が、穏やかではなくなる反応にはなるかなぁって思うね。」
\item 私「うーーん。」
\item 先生「いやもう、間違いなくこう、なんやこう、ちゃんと切り捨ててというか自分にその嫌な、要素というか、あの、まあ方向性、修正っていうの、排除して、矯正して、あの、次に進もうという発想にはなってるので、まあ時間の問題でいずれどうでもよくなる日がくるかなぁって思うけどね。」
\item 私「そう、ですねなんかそう、なんですよねなんか最近なんか、その辺の感覚、まあわりかし強くって、強くってというか、こう、環境が一回、そう新しい環境に変わって、」
\item 先生「うん。」
\item 私「半年ぐらい過ぎたら、なんか全部、こうー、ほんとにどうでもよくなりそうな、気がしてて、」
\item 先生「うんうんうん。」
\item 私「あー、家のこともそうですけど、あーのー、大学生活で関わってきたクソども、」
\item 先生「ふふふふふっ。」
\item 私「とか、まあ、前の付き合ってた人とか、」
\item 先生「うん。」
\item 私「あの辺が、なんか、うん、こう、なんかもう一個、きっかけがあれば、なんか、こう、そうですねえ、あーの、まあ思い出なんてものには絶対ならないですけど、」
\item 先生「うん。」
\item 私「こうー︙︙うーん、こーう、うん、それこそ、、肉芽みたいな感じに、肉芽が吸収された後の、」
\item 先生「うんうんうん。」
\item 私「瘢痕みたいな感じに、なっていくかなぁっていう感覚が最近、まあ、最近というか、うーん、なんとなーく、こう、多分、一か月間くらい持ってたのをいま話しててぱっと、上に昇ってきた感じがして、」
\item 先生「うーん。」
\item 私「うーんなんか、うーんそれこそ、結構、そうですねあのー、就活で、忙しくなる前というか忙しく動き始める前、」
\item 先生「うん」
\item 私「くらいまではなんかもう何かしらにつけてなんか、こうー、あぁ、思い出し憎みみたいなことを、」
\item 先生「あぁ。」
\item 私「結構してたんですよね。二週間に一回とか一週間に一回みたいな、夢、なんかその、出てくる、人は、まあ毎回変わるけど。」
\item 先生「うん。」
\item 私「夢にも、まあ毎日一人は誰かしら出てきて、その、あの不快な、不快な目覚めをするんですけど。」
\item 先生「うん。」
\item 私「あのー、ばっと目が覚めるみたいな、なんか毎日あったんですけど、なんかこう、先の事を、固めつつある、状況に、まあほんとの意味で、踏み込んだわけじゃないですか試験受け始めて。」
\item 先生「うんうんうん。」
\item 私「で、なんか、あのー、役所の訪問行ったりとか、なんかこう新しい人間関係ができそうな、というか新しいコミュニティができそうなところ、に触れていってる、経過の中で、まあーなんか、その辺—が、だんだんと薄れてきたなあという、こうなんか、ふと、話題に出されたり、してもこう、まあ、それまでは、結構、表面では笑ってますけど、なんか、でたらめのおべんちゃらーみたいな感じの、やってますけど、中身はほんとにはらわた煮えくりかえってるみたいな。」
\item 先生「うん。」
\item 私「感じの、反応だったんですけど、なんかわりかし、うーーんなんか、こうー、あぁでも若干防衛機制みたいなんは残っててまだちょっと、なんかどうでもいい、は、まあまあ、まあ今はまだなってないんで。」
\item 先生「うん。」
\item 私「あのー、かわいそうな人間たちやなぁっていう風にこうある種、」
\item 先生「あぁ。」
\item 私「思える、なんかたぶん、こう、一個次の、憎む一個次の段階じゃないですか。」
\item 先生「うんうんうんうん。」
\item 私「まあそこもまだ、多分防衛の、」
\item 先生「うん。」
\item 私「中ですけど、かわいそうな人間、まあ実際かわいそうな、というかまあ愚かなやつらだなぁって、感じ、まあこう︙︙あの、客観的って言葉あんま好きじゃないですけど、まあ愚かじゃないですか。」
\item 先生「うん。」
\item うち「まあまあ、うちの、客観からみれば愚かじゃないですか。でまあその一個、あのー、一つ機制を突破した先の機制に、入ってきたなあっていう感じが、して、でそこをだんだんだんだん、こう、自分の、体験の中で、追体験していく回数が減ってきた、ので、夢に出で来るとか、まあ未だにまあ、出てくることは出てきますけど。」
\item 先生「うん。」
\item うち「あの、実家の人間だったり、あのークソ、クソどもだったり、」
\item 先生「うん。」
\item うち「付き合ってた人だったり、出て、きますけど、回数は減ってきて、で追体験する回数も減ってきたので、まあそのまんま、こう︙︙ゆるーく、消滅していく、感じになんのかなぁって。目には見える、傷として、残ってはいるけど、こうそれを、異質なものとして取り扱わなくなる、感覚、というかまあ、なんか、まあ、それこそ、傷跡みたいに、なんかこう触って凸凹してたら意識向きますけど、なんか凸凹してないところで、」
\item 先生「うんうん。」
\item うち「なんかちっちゃい傷というか、吸収されたところの、瘢痕。まあ見たら傷跡だけど、いやまあ、在る。こう、」
\item 先生「うんうん。」
\item うち「在る。って、存在するだけだよねみたいな。感じの、ビジョンは結構見えてる、んで。」
\item 先生「ふーん。」
\item うち「そうまあ、まあまあ、この、ほんとに、新社会人になったら、」
\item 先生「うん。」
\item うち「まあそれどころじゃなくなるっていうのも多分、あると思うんですけど忙しさとかで。あるんじゃないかなぁって。」
\item 先生「うん。」
\item うち「思ってるんで、わりかしこうほぐれてきては、いますよねその呪縛みたいなものが。」
\item 先生「うんうん︙︙まあその、無理に、ほぐらせるとか無理に見ないようにする、感じないようにするとかじゃなくって、ちゃんとこう、あ、自分の中ではもうそれ、それだけの価値が、なくなったとか、その、反応するだけのエネルギーがそこにあの、つぎ込めなくなったなあっていう風に、変わっていってる感じがあるわね。」
\item うち「うん。」
\item 先生「すごくこう、良いというか前向きな変化よね。やっぱり。なんか自分がこう、今までいてた段階と違う段階に移る、違う状況に移っていってるっていう風に、自分の中でこう、そういう事に絡まってた状況が終わっていってるっていう風に、なんかどっかでやっと理解しだしたというか、新しいことが始まったことで、そういう変化がどうしても起きるんやっていう事を、自分がなんかわかりだした感じやね。」
\item うち「そう、ですねー。」
\item 先生「︙︙うん。まあすごい、良いなんか変化やね。無理なことをして、あのーなんか頑張って乗り越えようとか、あの、なかったことにしようじゃなくって、自然にあぁなんか、過去の記憶の一点になるだけで、あの、意味、意味としてとかその自分の中での重たさとしては、なんか0にどんどん近づくんだなあっていう風に、勝手にこう変化が起きてるのを自分が眺めてる感じになってるよね。」
\item うち「そうですね、なんかこう多分、こう、まあ、消えはしないんですけどあの、永続性はあるんですけど、あのー、重みが0になって、」
\item 先生「うん。」
\item うち「質量0になってただそこに、在る。っていうだけみたいな、こう、思い出だとか過去のことだとかそういう話には多分ならないんですけど、まあでもこう付帯物として、かな、ただ付いてる、というか、こう、うん︙︙んーなんかこう、んーー目に見えないほくろというか。」
\item 先生「ふふっ。例え、その例えが難しいけどまぁうん。」
\item うち「なんかうーーん一番近いのはまあそれこそこんな感じの(自傷痕)。」
\item 先生「そうねえ。」
\item うち「傷跡の、瘢痕。みたいな感じなんですよねぇなんか、そう、うん。」
\item 先生「まあ、あの、新しいこと、は、ちゃんと就活うまくいけば、それこそ忙しくってそれどころじゃないとか、今現在とこれから先の事で、あの頭の中いっぱいになるから、過去のことなんかほんとに取り合う暇もない、価値もないっていう風になっていくと思うんで。」
\item うち「はい。」
\item 先生「まあそうなっていったら、色んなことが勝手に終わってるよねきっと。」
\item うち「そうですねぇ、うん。」
\item 先生「うん。そういう、なんかあのー︙︙自分が色々こう、まあ、積極的に頭を整理する感情を整理するというような、作業を頑張ろう︙︙えっとそれそのものと取っ組み合わなくっても、自分のためになることを、やりたいことやろうってしてるだけで、そこは勝手にあのー、解決されたことになっていくんだっていうのがまあ、見えだしたっていう感じやなぁ。」
\item うち「はい。」
\item 先生「んーそれなんか正しい変化が起きてるよねぇ。」
\item 僕「なんかこう最近、なんか、あぁ、こう世界中心にまわってんなぁ、僕って感じが。」
\item 僕「んふふふふっ。」先生「はっはっはっはっはっはっはっ。」
\item 僕「してるんで、うーん。まあいい感じではあるかなぁ。」
\item 先生「うーん。そやねぇ。」
\item 僕「︙︙んーーまあそれ以外の︙︙あーなんか、でも、こう、まあ、前の、週末に、」
\item 先生「うん。」
\item 僕「あるって言ったじゃないですか、就活が。」
\item 先生「うん。」
\item 僕「二つ。まああれ終わって、で、まあなんか、こうー、相変わらず、なんていうんだろうなぁ、んーー、やっぱまだちょっとこう身体にしみついてる感覚みたいなのは、こーーぅ︙︙まあ、面接の前とか、薬飲んで行ったんですけど、頓服、出してもらってたんで。」
\item 先生「うんうん。」
\item 僕「んーーーなんというか、あー、あっ、違う話を、先にこっちを話します。」
\item 先生「うん。」
\item 僕「〇〇の、試験が通ってたら、あのー八月の、一七・一八のどっちかで、最終試験なんですよ。」
\item 先生「ほうほうほう。」
\item 僕「ほんで、××の筆記が通ってたら、一七日、に、二次試験なんですよ。」
\item 先生「あら。」
\item 僕「ほんで、まあ終わった後にそれを確認して、あーなんかどっちか選ばんといけんようになるかもしれんなぁ。」
\item 先生「うんうん。」
\item 僕「ってなるじゃないですか。で、まあでも実際発表は八と九なんで、まだまだ先、というか。」
\item 先生「うんうんうん。」
\item 僕「まあ発表されてから、考えればいい、というか、」
\item 先生「うん。」
\item 僕「それまでは別に何にもこう、せっつくものとして持ってなくてもいい、じゃないですか。」
\item 先生「まあ両方とも引っかかるというすごい嬉しい、」
\item 僕「あ、はい。」
\item 先生「事があったときに、困るってことな。」
\item 僕「はい。」
\item 僕「で、なんかそこを考え︙︙なんか一時期、二日間三日間くらい試験終わった後、」
\item 先生「うん。」
\item 僕「うん、試験終わった後というか試験中とかもですけど。」
\item 先生「うん。」
\item 僕「試験中と試験終わった月曜日くらいに、なんか、どうしようみたいななんかこう、」
\item 先生「うん。」
\item 僕「焦りみたいななんか、高まっていく感じが、あーこれ被っとったらどうすればええんやろみたいな、なんか色々まあ、」
\item 先生「ふーん。」
\item 僕「やり方はあると思うんですけど、」
\item 先生「うん。」
\item 僕「どっちか選ぶか、なんか、時間帯別で、こうずれてたら行けるし、でもずれてなかったら、」
\item 先生「うん。」
\item 僕「とか、あと一七・一八どちらかって言ってるからまあ電話したらもしかしたら行けるかもしれんなぁとか、まあ色々なんか、可能性、まあでも、あの発表あるまで、そんなことに頭を悩ませなくていいじゃないですか。」
\item 先生「うんうん。」
\item 僕「で、悩ませなくていい、のにそっちになんか、まあ、いい、いいでしょ別にって、まあいいでしょって言いながら別の作業をしてたんですけど、」
\item 先生「うん。」
\item 僕「こう、隅に、」
\item 先生「まあね。」
\item 僕「こう、出てきて、なんかこう、未だにその辺の、こう、タスクに関する︙︙こうー、予期される不祥事に、対する、こう︙︙事前準備的な、焦り不安みたいなんが拭いきれてないなぁっていう。」
\item 先生「うんうんうん。」
\item 僕「そん時すごい実感して、」
\item 先生「うーーん。」
\item 僕「でも、まあ、まあ、うーん︙︙そこに関してはもう、何回も数こなしていくしかないよなって思いながら、」
\item 先生「うん。」
\item 僕「やってたんですけど、一回なんか、いつだったかなぁ試験の前の、週、から、」
\item 先生「うん。」
\item 僕「△△(カウンセリング)に行った週のどっかで、なんか、なんかすごい、頭ん中が煩雑になった時があって、こう、なんにも考えがまとまらなくってなんか、こう思考の種みたいな、こう考え、考える、」
\item 先生「うん。」
\item 僕「ものの種みたいなものが、頭の中でぼこぼこぼこぼこって生まれては消え生まれては消えみたいな。」
\item 先生「はーはー。」
\item 僕「感じになってなんか、こう目の前で試験勉強をするために本を開いてるんですけど、どんどん目が滑っていって、」
\item 先生「うんうん。」
\item 僕「目が滑るだけだったらまだ疲れてる時とかあるんですけど、」
\item 先生「うん。」
\item 僕「それの理由がなんかそういう、なんか、こう、何かを、考えようと、してる、兆し、の時点で、あのことを考えようこのことを考えようって思った時点でこうなんかそれが消えるみたいな。」
\item 先生「うーん。」
\item あたし「すごいいっぱい起きて、こうそん時になんか、あたしはなんか、ボールのなんか、なん、なんていってたかなぁ、んーー、こう、頭の中が、反射機構みたいな感じにになって。」
\item 先生「うんうん。」
\item あたし「こう、破壊性がある、運動エネルギーを持った、いっぱい、粒子が、」
\item 先生「うんうんうん。」
\item あたし「こう、ぶわぁーってランダムに、攪拌というか、」
\item 先生「うん。」
\item あたし「ぶつかられまくってて、でその、その粒子自体まあ、そもそも最初なんか、おっきい球みたいなものが、数個あって、こう反射していった中で、その反射であったり、球同士がぶつかったりして、」
\item 先生「うんうん。」
\item あたし「どんどん細かくなっていって、なんか、で、どんどん小さくなって数が増えて、で最終的になんか、ノイズまみれっていうか、ノイズしかない頭の中、状態になって、こうほんとに、身体を動かすこと以外できないみたいな、こう、」
\item 先生「反射っていうかなんかブラウン運動を見てるみたいなね。」
\item あたし「ああーー、なんか、ほんとに︙︙あーの、でも最初の方、こう、あれやるかみたいな、あの、皿洗うか、皿洗うか球と、本読むか球と、勉強するか球と、あーやっぱトイレ行きたいな球と、」
\item 先生「うんうんうん。」
\item あたし「それがなんか動き始めてバーンってぶつかっていって、だんだんそういう、皿洗うかが、皿球と洗う球としよう球みたいななんかどんどん、こう、区切りがなされていって細かくなっていって、最終的になんか意味を持たない、粒子の集合、集合が頭の中に、こう粒子たちが、吹き荒れてるみたいな、状態になって、こう、統率が失われてるというか、そんな感じになって、そん時はなんかすごい落ち着かないし、やろうとしてることができないから落ち着かないし、」
\item 先生「うんうん。」
\item あたし「そのノイズで自分が何をしたいのかもわからないし、なんか、すごい挙動不審で気持ち悪かったんですけど。」
\item 先生「うん。」
\item あたし「まあなんか、次の日も若干、んー、マシにはなってたんですけどそんな感じで、まあでも二日後には、すっきり治ってたんですけど。うーんなんか、そう、そういう、こう、異質な、ことが、あって。まあまだ健全と言い切るには程遠いなぁって。」
\item 先生「そういう時はでもほんとは薬、はよ飲んだ方がいいのよねぇ。」
\item あたし「あー︙︙」
\item 先生「頓服をねえ。」
\item あたし「なんかそういうときって、効くんですかね。いま、あの、前そのー、デパスの効き目をめちゃめちゃ実感したのは、」
\item 先生「うん。」
\item あたし「もう、あの︙︙殺される!みたいな、こう動的な、」
\item 先生「うんうん。」
\item あたし「なんか、ああああぁぁぁぁぁぁーーー、やべえ。こう、すごい、感情の攪乱を伴った、」
\item 先生「うんうんうんうん。」
\item あたし「感情を、そう、感情の、こういう、過剰、もう過剰も過剰な、攪乱、でなにも︙︙なんか、手が付かなくなって、こう、こう、胸、んーーー錆びた釘でほじくりまわされるようななんか痛みみたいな体験とかそういう、なんか、いつ殺されるかわかんないみたいな」
\item 先生「うん。」
\item あたし「切迫感とかそういうものが綯い交ぜになって、手が付かなくて涙ぼろぼろ出始めるみたい時に、飲ん、で、スッ、それらが、こう潮が引いていく波が引いていくみたいにこう、はあぁぁーー、なくなったわ。みたいな。あーねむたぁ。みたいな。」
\item 先生「うん。」
\item あたし「感じ、の、あのー、感覚はあるんですけど、なんかその時の感覚とは、全然違って一番、違う点がその感情の動きが全く伴ってなくって、」
\item 先生「うんうんうん。」
\item あたし「ただこう事象がまとまらないみたいな。」
\item 先生「うん。」
\item あたし「統率、事象の統率が全く取れてなくって、まあなんかそんな、めちゃめちゃ、こう、extraordinaryな、悲しみだとか、焦り不安みたいなのは、なくて、それ効く、んですか?そういうのに効く、もんなんですかね。」
\item 先生「うん。あのー、」
\item あたし「はい。」
\item 先生「感情的な反応はあんま伴ってなかったって実感はあるやろうけど、そのー、一つ一つのそのー行動の選択とか、あの思考の整理とか、そういうことに対してなんか、まずこれを終わらせようとかこれに、関心を持ってたら今はいいんだっていうような、まあそれで、上手くいくいくはずやっていう、安心感がそもそもない状況やったはずなんよな。」
\item あたし「ああぁー。」
\item 先生「あのー、切迫してるとか、つらいとかはないけど、先これ、いやこっちのほうがいいかな、とかなんかこう、これではないんじゃないかみたいな、こと含めて、」
\item あたし「あーはいはい。」
\item 先生「あの、頭の中が一つのことにとどまるとか関心を持つことを、許せなくなってるような混乱がまあ、わーっておきてるんですけど、そうやって、一応やっぱりあの感情じゃないけど神経の活動自体がやっぱりあの、やらんでいい高ぶりがあるんですよね。」
\item あたし「あー。」
\item 先生「だからそこをちょっとこうなだめるって意味ではデパスとか安定剤でいいんですよ。」
\item あたし「あ、そうなんですね。」
\item 先生「うん。だからあのちょっと飲んで、あのむしろ、そんなたくさんの事に、あの意識持てないとか注意が向かない、気づかないくらい、ちょっと頭を鈍らせて、あの最低限の事しか、興味が持てないというか関心が、あの持てないみたいな緩んだ状態を作って、でやれることだけやろうかって収まるっていう風に、」
\item あたし「あーー。」
\item 先生「していったほうがいいので、今回それ一日二日で済んだからいいけど、その混乱ってあのーブレーキ掛けへんかったら、あのーちょっとね、あの、質が悪いというか、長続きして脳がどんどん疲れる、あのー、原因が明確なものがなかったとしても、その頭の構造から勝手に鬱っぽいような神経の疲れ、しばらく動けませんってなるような、あの、燃料の枯渇ぶりっていうのは出ることはあるので、」
\item あたし「あー。」
\item 先生「ちょっと早めに薬飲んでそこから脱した方がええんですわ。」
\item あたし「わかりました。なんか当時は、読んで頭の中に入れるっていうことに集中ができなかったんで、逆にコンサータとかあっちの方なのかなって思ったんですけど。」
\item 先生「あーー。」
\item あたし「それって逆なんです、か?」
\item 先生「うんまずは、あんね、こう、ちょっと暴走気味やなっていう、発想で、とりあえずまず、ブレーキをかけるっていうかそんなにこう勢い出せないように、まずしてみる。いう事が先かな。あの、コンサータとか飲むっていうのは、あのーもうちょっと、実感表現難しいけど、あの、色んなものに注意を向けるだけのエネルギーは十分ありすぎる状態なんよね。」
\item あたし「あー。」
\item 先生「だから一個一個のことをその、気が付いて考えようとはできるけど、考えられないまま、次の事が意識に昇るじゃなくって、これもやれてしまうあれもやれてしまうっていうのを、一個一個踏み越えながらあの、踏み込み過ぎながら移りすぎちゃうって時にはコンサータ使うけど。」
\item あたし「あぁ。」
\item 先生「うんそこはちょっとちがうかなぁ。その、種というかその、考えた方がいい考えられる対象はいっぱい出てくるけど、それをどうこうできる間もなく次のことがいっぱい見えてくる、」
\item あたし「そう!」
\item 先生「そういう時っていうのはあの、単にちょっと混乱して、頭が暴走してるっていうだけなので、まず鎮めること。」
\item 先生「︙︙まあ、また次おんなじ事あったらまず一旦、安定剤飲んだほうがいい。」
\item あたし「わかりました。そうですね、うん、それが、一番気になった点としてあって。」
\item 先生「ふーん。」
\item あたし「でもなんか、こう、それ以外の点に関しては、こう、身体の反応的にはあんまり、まあ、特異なことはなかったかなぁっていう。」
\item 先生「うんうん。」
\item あたし「感じーです、ね。あ、一回眠剤飲むのを忘れて、全然寝れなくって、あこれ飲まないといけないなって実感したことはあったんですけど。うーん︙︙」
\item 先生「今の眠剤ってなんやったけ。」
\item あたし「えっとー、あの、あれだ、オレンジ色の、名前忘れたなぁ。オレンジ色の。」
\item 先生「丸い?楕円?」
\item あたし「楕円。楕円の、なんか、徐放剤、の錠剤。」
\item 先生「え、楕円で徐放剤、徐放剤︙︙」
\item 先生「徐放剤、チックな感じでした確か。あーの、なんか今まで、試してなかったタイプの、その、鬱軽減系の、薬を、」
\item 先生「リフレックスとかかな」
\item あたし「えっとー」
\item 先生「レメロン、リフレックスではなく、楕円形、、トラゾドン、」
\item あたし「えっとーーー、いや、トラゾドンではなくって︙︙あのもう一個はなんか、すごいちぎりにくい、アルミで全部覆われてるような、」
\item 先生「ほう。ああベルソムラか。」
\item あたし「あぁそうです。そうそうベルソムラ。ベルソムラの、なんぼやったかなぁ、\rensuji{20}、」
\item 先生「うんうん。」
\item あたし「と、あとー、その、」
\item 先生「オレンジ色の徐放剤︙︙なんやろ。」
\item あたし「オレンジ色、で、」
\item 先生「オレンジ色っていったら普通マイスリーかなって一瞬思うけどちゃうねんなぁ。」
\item あたし「えーっと︙︙あ!今あるかな、」
\item 先生「お?」
\item あたし「あ、ありますわ。これですね。」
\item 先生「あぁ!あーこれは思いつかんかったごめん。確かに徐放剤やんなこれ。」
\item あたし「はい。これなんか、いまいち、あのー、眠るためとしてあの、服用ですけど、」
\item 先生「うん。」
\item あたし「調べてもいまいちなんかこう、どう反応してるのか、分かんなくって。」
\item 先生「それはね、あのもともとはあの、抗精神病薬、なんですよ。」
\item あたし「はい。」
\item 先生「で、あのその時の作用は、あのー頭の働きを止める、」
\item あたし「あぁー。」
\item 先生「やつです。であのー、おんなじ成分やのに、あの徐放剤にして、そうもともと鬱の、その気分のしんどい時も、多少効く、まあしんどさを和らげるって意味でね、元気にはしないんやけど。」
\item あたし「はい。」
\item 先生「っていうのが分かってたので、その、あの薬の名前と、徐放剤っていう作り方を変えて、鬱の薬っていう風に出しなおしたっていう部分がある薬なんですけど、」
\item あたし「あー」
\item 先生「あの、基本は、頭の働きをとめる、」
\item あたし「あぁ。」
\item 先生「過剰に興奮するとか、そのー、どうしてもあのくすぶるような、感情があるときに、しんどさを、一旦は、あの神経の働きを止めることで、打ち消しておいて、良い部分が出てくるように、あの脳の働きかたのバランスを整えてあげようみたいな、薬になるかな。」
\item あたし「ふーん、、あー系統としては、あれですかあの、前飲んでたオランザピン。」
\item 先生「あっ、まさにそんな感じ。でもともとこっちの成分の方が、あのー頭の止められ方が、あのすごい軽いっていうか、」
\item あたし「ああぁー。」
\item 先生「評判はいいのよ。」
\item あたし「そうですね、あのオランザピン飲んでた時はなんかほんとにどんどん、こう人形になっていく感じがしてたんで。」
\item 先生「うん。」
\item あたし「でもこっちは、そんな。」
\item 先生「だから止められるというか頭の、機能がなんか、欠けたみたいな感覚になるっていうのはこっちの方がはるかに少ない、」
\item あたし「うん。」
\item 先生「止まり方するんです。」
\item あたし「なるほどなぁ。」
\item 先生「うん。」
\item 先生「でもその二つやったら一回抜いて寝れなかったっていうのは確実にその、あの反動で寝れないじゃなくって、」
\item あたし「はい。」
\item 先生「あのー薬、で、神経を鎮めないと、」
\item あたし「うーーーん。」
\item 先生「薬が抜けたことの問題じゃなくって、薬が必要な神経の状態で、あのー睡眠薬必要な段階、状況なんやろなって感じがあるから。」
\item あたし「あぁ。」
\item 先生「そこはあの、いや、薬が反動で、抜けた反動で寝れないっていうだけやったらもしかしたら、うまくしたらあの、減らしたりやめたり、」
\item あたし「うん。」
\item 先生「できる道が見つからんかなってなるけどその二つで一回抜いて寝れないんやったら」
\item あたし「はい。」
\item 先生「まだ、まだまだ。」
\item あたし「うんうん。」
\item 先生「まあ状況とかまあ生活環境とか色んなものが、変わりきるまでは、」
\item あたし「はい。」
\item 先生「あの、やめん方がええっていう風な、ことになるやろな。」
\item あたし「うんうん。︙︙それで、あの一応処方は、寝る前なんですけど、」
\item 先生「うんうん。」
\item あたし「あのーイフェクサー以外は。」
\item 先生「うん。」
\item あたし「寝る前に飲んだら、朝ほんとに活動できないんで、あのー、九時くらいに、」
\item 先生「はいはいはいはい。」
\item あたし「飯後にこれを、イフェクサーを飲んで、九時くらいにこの、」
\item 先生「ビプレッソ」
\item あたし「ビプレッソと、あーの、あれですわ。ロフラゼプでしたっけ。」
\item 先生「あぁ、メイラックスのやつね。」
\item あたし「の、これを九時くらいに飲んで、」
\item 先生「うん。」
\item あたし「で一時間後くらいに、これを飲むっていう。一〇時くらいに。してたら、まあ朝活動できるかなっていう、感じの、そう最近服薬サイクルを見つけて。」
\item 先生「うん。うんいい調節やね。」
\item あたし「はい。であんまり、あの、八時とか七時に飲んでると、あの、まあそれでもなんか、五時、、六時前とか、六時半とか多分、一回目が覚めるんですけど、」
\item 先生「うん。」
\item あたし「まだ、あの寝れる。」
\item 先生「うんうんうん」
\item あたし「もう一回。でももうちょっと、前、七時とかに飲んでると、」
\item 先生「うん。」
\item あたし「四時とかに目が覚めて、」
\item 先生「あぁーー。」
\item あたし「それからもうずっと寝れないみたいな、感じになってたんで。」
\item 先生「それは早く飲めば飲むほど切れ目がちゃんとわかんねや。早まんねやね。」
\item あたし「あぁはい。」
\item 先生「ふーん。」
\item 先生「あのまあ、きわどいところなんかうまくおうてる感じやな。」
\item あたし「うん、ちょっと、あの使い、慣れしてきた、感じはあるので。」
\item 先生「うんうんうん。」
\item あたし「まぁまぁまぁまぁって感じで。」
\item 先生「でもその合わせ方は、うまくしてもらった感じやね。」
\item あたし「ふーん。」
\item 先生「うんうんうん。なるほどね。」
\item あたし「そう、そんな感じですね。」
\item 先生「︙︙うん。それはまあ今の続けていって。まあ忘れたらいかんね、薬。」
\item あたし「はい。そうですね、ほんとに。あとー、あ、そう今、あの、頓服で、貰ってるのが、メイラックスの\rensuji{0.5}なんですけど、」
\item 先生「メイラックス?」
\item あたし「あ違うメイラックスじゃないわ、エチゾラム、エチゾラムの\rensuji{0.5}なんですけど、」
\item 先生「うんうんうん。」
\item あたし「あのー、一番最初ここにもらってたのって、」
\item 先生「うん。」
\item あたし「\rensuji{0.5}でしたっけ1でしたっけ」
\item 先生「\rensuji{0.5}、\rensuji{0.5}。ここは\rensuji{0.5}しかないからね。」
\item あたし「なんか、あーの、まああの時、の、飲んでた時の、感覚のブレが、ブレというか、攪乱され方が、だいぶ動的だったっていうのもあるんかもしれないですけど、」
\item 先生「うん、うん。」
\item あたし「あのー頓服として飲んだ時に、その、院試前の時ほど、こう、」
\item 先生「うん。」
\item あたし「身体が、こう、スッと、ストーンと落ち着くような、感覚、あそこまでこうー、落ち着ききる感覚がないなぁって。」
\item 先生「んーー。」
\item あたし「感じだったんで。」
\item 先生「えっと、あのー今、その半錠飲んでるやつ、」
\item あたし「あ、はい。」
\item 先生「あの、あれが本来は、あのーデパスの、ちょっと改良版。」
\item あたし「うんうんうん。」
\item 先生「で、ただ長時間型で効き目がほとんど一緒なんですよね。あのー、ベースにその成分が入ってる状況で、」
\item あたし「はい。」
\item 先生「それでもまだ神経が反応してる、っていうところになるので、あのー、要は全く0の、なにも薬がない状態で薬が初めて入るっていうのと違いもあるし、」
\item あたし「あーあー」
\item 先生「そのベースにあるにも関わらず薬飲まなあかんところまで、」
\item あたし「あぁーーーなるほど。」
\item 先生「神経が高ぶってるっていうところでの、あの、薬飲むっていうところもあるので、そのー、対象となる状態も、ちょっとあの違うしこの落差も全然違うっていうところがあるから、効いた実感としては前よりも、あのちょっとぼやけてる感じには、どうしてもなっていくかな。︙︙んーでもあの、それはちょっと薬慣れってしてしまってるとこもあるかな。」
\item あたし「あー。」
\item 先生「んーそこはあの、もうちょっとシャープなというか切れ味を求めてって言ったら、もうちょっと強い安定剤ってあるので、」
\item あたし「はい。」
\item 先生「その頓服、不十分なんで何とかなりませんかって言ったら相談はできると思うんだけど、対応はしてもらえると思うけど、あのーそこまでの事が、要は、あのちゃんと収まるとかもっときつい効き目でないと、その場を自分があの収めきれない乗り越えきれないっていうのになるんだったら要求したらいいだけで、まあ何とかなる程っていったら、強くし過ぎない方が、」
\item あたし「あぁですよね。」
\item 先生「いいとは思うね。」
\item あたし「いやまあ、それこそあの面接のときに、」
\item 先生「うん、うん。」
\item あたし「あーの、院試勉強してる時とかは、なんかもうほんとに、緊張の種すらなくなって、むしろ、あーもういいや眠てぇー寝ようー」
\item 先生「あははっ。」
\item あたし「ねむくなってきたなぁって、ちょっと一〇分か一五分寝よって感じになってたんですけど、面接前飲んだ時はまあ、あのまあ多分、おそらく、こう︙︙僕も、まあ、分かんないですけど、その、感覚が、長い間、なかったので、」
\item 先生「うん。」
\item あたし「あのーおそらく正常な範囲の緊張感?」
\item 先生「うんうんうん。」
\item あたし「こう健全な、人が、緊張するってこんな感じなんかなぁみたいな、感じの緊張感、で収まって、でまあ、あの、手が震えることもなく、なんか、そんなになんか身体が反応することもなく、殺されるような、緊張感というか、」
\item 先生「うん。」
\item あたし「そういう命の危険を感じるような、ものでは全くなく、多分、多分まあ、こう健全な緊張感なんかなぁっていう感じだったんで、まぁまぁ、いっかって思ってそのまま、面接に臨んだんですけど、うん。」
\item 先生「まあ、そういう風なあの、感覚やったら、」
\item あたし「はい。」
\item 先生「あの、その形で臨むっていう、選択で間違ってない気がするけどね。」
\item あたし「はい。」
\item 先生「ただまあどっかでその緊張してる故に失敗したかもっていうような、こと経験しちゃうと、まあ飲んどこかになるやろうけど、それはなってから考えて対応していくってかんじやねぇ。」
\item あたし「そうですねえ、はい。」
\item 先生「︙︙まあでも、薬は、まあまあ、おんなじ形で続けるのが今一番大事やねきっと。」
\item あたし「うん。」
\item 先生「じわじわよくなっていっていらなくなるっていう変化が起きるってことはあってもいいんやけど、」
\item あたし「はい。」
\item 先生「まあ、この先すごく大きく生活環境が変化するはずやと、思うと、」
\item あたし「はい。」
\item 先生「じわじわよくなっていくのを待つよりも、そういう変化が来る方が時期的にきっと早い。」
\item あたし「うんうん。」
\item 先生「やから、まあその変化が起きた後自分がどんな風にその状況にフィットするかっていうのを見た上で、薬の相談するっていうまで、」
\item あたし「はい。」
\item 先生「まあ、何も考えずに続けとけばいい気がするね。」
\item あたし「んーいやそうなんですよねぇ、薬大事やなぁって︙︙」
\item 先生「うん、ほんまにあの、反動が来るだけで寝れない薬とかがあるから。」
\item あたし「あー。」
\item 先生「それだったらあの、逆にやめるチャンスかなって思ったけど。ちょっと違ったね。」
\item あたし「そうですねぇ。」
\item 先生「んー、まああの、今度、忙しいねんな?だから。あのー、来週?試験を受けて、ちゃうわ結果が来て、」
\item あたし「そうですね。」
\item 先生「で、一七日をどうするかになんねんね。」
\item あたし「はい。まあでもそこは、なんか、生協の人に聞いて、ダメだったらまぁ、両方もうブッキングしてしまって、どっちか選ぶしかねぇってなったらまぁ、そうするしかないん、で。まあそうするんですけど。だからなんか、それこそ、今までは、こう、可能性が、」
\item 先生「んー。」
\item あたし「ちっちゃい可能性がいっぱいあるからそれに踊ら、されてた、だけって感じなんで、別に、実際事象、現象だけ見て、考えてしまやぁ、ブッキングしたら、選ぶ。しなかったら両方受ける。」
\item 先生「うん。」
\item あたし「だけなので。」
\item 先生「そやねぇ。」
\item あたし「はい。んー、で、別に、あのー対策も、もう筆記、は全部終わったんで、対策することもないんで。まああとは、いつも通り、まあおそらく、あのー日常から頭を回してない人間は多分対策しないといけないですけど、」
\item 先生「ふっはっはっはっはっ」
\item あたし「てかあのー、なんか、就活の、軸を、」
\item 先生「うんうん。」
\item 私「ちゃんと自分の中でかみ砕いて持ってないと。あの、私は、そんなことはまあない、ないというか、ないじゃないですか。」
\item 先生「まああのー、ある程度色んなことを自覚した上で、」
\item 私「はい。」
\item 先生「希望してるとか言えるもの持ってるっていう感じはあるもんね。」
\item 私「はい。」
\item 先生「うん。」
\item 私「でまあ、なに言われてもこう、、んーまあ答えはできるので。」
\item 先生「うんうんうん。」
\item 私「あとはまあ緊張しないように、メンタル持っていくだけかなぁって感じなんで。んーあとなんか、なんかあったかなぁ。︙︙あぁなんか、まあこれはなんか多分そんなに、おっきい問題には発展しないんですけど、」
\item 先生「うん。」
\item 私「あのー、それこそ、その、あのー頭が、暴走してる日に、ネットの訪問の人が来て、あーの、」
\item 先生「ネットの訪問?」
\item 私「ネット回線の、アパートの、」
\item 先生「あぁ、全体で、やってる回線の。」
\item 私「はい。乗り換え、多分あの、終わった後に、違うところへの乗り換えの奴だったって気が付いたんですけど。」
\item 先生「はーはー。」
\item あたし「まあ別に、それで、いまあたしが使ってるプラン、学割切れてたって思ってて、あの、ネットの、要は料金下がりますよって。」
\item 先生「うんうんうん。」
\item あたし「でまあ、その人が全部把握してるような言い方だったんで、あ、そうなんやって思って書類書いて、SMSで母親に概要を伝えたら、学割がまだ効いてて安かったのでキャンセルをしたってことがあったんですね、で、母親とSMSである程度やり取りをしたんですけど、まあ、そのあと、速達が来てて、昨日くらいに。まあでもこれは写真撮って見せてもしゃあないなって思ってすぐ捨てたんですけど。内容が、なんか、完全にキャンセルしましたって送ったんですけど、なんか、あの、乗り換えのこととかなんか、こう、自分のなかでこうこう思ってるみたいなのがうだうだ書いてあるだけで、なんかある種、あれなんですかね、うちがまだ、多分こう庇護下にいるみたいな、感じのを、意識してるかわかんないですけど持ってて、こっちが能動的に動いたから。」
\item 先生「うん。」
\item あたし「なんか、私の、力の及ぶというか、及ばせれる、領域はまだあるんやみたいなのを。」
\item 先生「あーー。」
\item うち「うちに提示したかったんかみたいな、深読みすればまあ、そういうこともみれるかなぁって。」
\item 先生「うーん、そう、そういう方向なんかな、なんか、あの、自分をこう受け入れてもらえるとか、」
\item うち「あぁ。」
\item 先生「自分の正しさを、またあの、証明できるというかあの、」
\item うち「うんうん。」
\item 先生「保証する、材料に子供が使えると思って、」
\item うち「うん。」
\item 先生「嬉々としてこう、絡んできたみたいな押し付けに来たみたいなような、イメージをちょっと持っちゃうけど。」
\item うち「あーあー、こう、そう、そうですねなんか、結局、うん、そう、じゃないすかなんか、自分が正しいから。」
\item 先生「うん。」
\item うち「こう、うちを自分の傘下に入れとけば、うちは全部正しいみたいな。」
\item 先生「うん。まあ傘下っていうか、んーそやねまあ、上下がどうなるか分らんけどでもなんかあの、私が正しい、あの、、あのなんていうか、自分で言うんじゃない証明を、子供でできるみたいな。」
\item うち「あぁ。」
\item 先生「道具になってる感じはあるよね。」
\item うち「なんかこう、最近の彼女の事は多分、先生の方が知ってるんですけど。」
\item 先生「はっはっはっは、ふふふ。でもまあ、ちょっと何回か喋っただけやけどね。」
\item 僕「あーの、それこそ、んーー、僕が、ここに通い始める前、くらい、まで、の、あの人の、親というものの認識、」
\item 先生「あぁー。」
\item 僕「は、あのー、保護をする人間っていう、過剰、過剰というか、病的なまでに保護をする人間。」
\item 先生「うんうんうん。」
\item 僕「なんかこう、対等じゃないんですよね。」
\item 先生「うんうん。」
\item 僕「対等じゃなくって、あの、特に中学辺りくらいまで、」
\item 先生「うん。」
\item 僕「は、こう、子供は無条件に親の言う事を聞くものだみたいなことを無意識に行っていて。」
\item 先生「うんうんうんうん。」
\item 僕「︙︙でそれがおそらく続いてるんですよね。続いてるというか、まあそのイメージがすごい強くって。」
\item 先生「はーはー。」
\item 私「私のなかで。」
\item 先生「うん。」
\item 私「そういうので、まあ今回の件も、そう、そういうところ、」
\item 先生「うん、ふーん。」
\item 私「が、見て取れるのかなぁって感じで。」
\item 先生「なるほどなあ。︙︙過去の流れから汲むとそういう風に解釈できるとこは確かにありそうやね。」
\item 私「はい。ただまあほんとに、ここ数年の彼女のことは僕は全然知らないんで。」
\item 先生「なんかあのー、子離れを、するようにとか、んーーお母さん自身の存在が、あの毒になるかもみたいなこととか、まあ僕が話すスタンスってそんな感じじゃないですか。」
\item 私「はい。」
\item 先生「そういうことをなんかあの、お母さんがあの、ほんとに自分が悪かったとか間違ってるって言わずに、飲み込もうと、」
\item 私「あーーー。」
\item 先生「するときの、その折り合いのつけ方っていうのは、あの、支配的、っていうことではもう持たなくなってる。」
\item 私「あぁ。」
\item 先生「のはやっぱりあって、うん、ちょっとこう、いや本当は、あの優位に立つというよりも、なんかあのー田所さんよりも自分の方が、何が正しい何がうまくいくって知ってるから、あたしの言うとおりにやったらうまくいくっていう、発想で押し付けたがるけれども、ちょっと、まあ僕が介入して以降は、あのー、あんたは、あの子供に、あの、もう及ばんところまで、あんたの子供は変わってしもうてんねんでいうような、切り口で言うてるのでまあ僕に対する、あのー、その僕とのコミュニケーションを上手くやるための、」
\item 私「はい。」
\item 先生「仮面かもしれないけど、なんかあのそういう、自分に何かしら優位性があると思って、ふるまっていいっていう態度はもう、僕にはあんまりみせなくなってるな。」
\item 私「あぁー、なんか、ほんとに、あの、こう、んーーー、まあほとんどの人間がそうだと思いますけど、こう権力っていうものを日常生活に、落とし込んでるというか、」
\item 先生「うん。」
\item 私「それありきで、生活していってるというか、こう歳、歳の差であったり、金であったり、それをあたかも力だと勘違いして、」
\item 先生「そうね、まあ、今は日本ではだいぶ、少しづつ特に都会では薄れてきたけど、まあそもそも男女、の違いだけで、」
\item 私「うんうん。」
\item 先生「なるからねぇ。︙︙うん。」
\item 私「でまあ、おそらく、地で行ってるのでそれを。」
\item 先生「うん︙︙やっぱりこう、なんやこう、立場というか家庭の中の力関係というか、何の根拠がなくっても自分がこう人に、支配的というか優位性を誇示できるっていう、既得権益ってみんななんか手放さないよねぇ。」
\item 私「そう、いやそうですねぇほんとに。んーー。」
\item 先生「それをこう、なんかフェアに、ていうかあの、ほんとにそうなんかって、疑問をもって謙虚になるっていうのは、完全にこう知的に、鍛錬されないとできないみたいやねぇ。」
\item 私「あー︙︙うーん、ほんとに、あぁ、前時代の遺物だなあって思いながら、付き合ってますけど、付き合ってるっていってもほぼ、断絶してますけど。」
\item 先生「前時代の遺物っていうか、人間の本質はきっとそうで、」
\item 私「あぁ。」
\item 先生「個体ごとにあの毎回啓蒙するってことをしないと、きっとそうにしかならないんよね。」
\item 私「あーあー、そうですね。」
\item 先生「うーん、でまあ社会的にはなるべくそうなるのが正しいっていう、ことが通用し始めてはいるけど、ま大多数やっぱり、あの、そういう価値観はほんとには理解してないよね。」
\item 私「もう、なんか、はやりチックにこう、まあ雑な言い方をしたら、」
\item 先生「うん。」
\item 私「はやりチックにぽっと出てきてるじゃないですか今のその流れって。」
\item 先生「まあねぇ。」
\item 私「風潮が。で、全然、分かってないじゃないですか実際なんか、それを、取り扱った、ワイドショーとか、」
\item 先生「うん。」
\item 私「政治家の演説とか、役所の対応とかみてたらもう、まあ基本的にLGBTとかの情報が入ってくるんですけど。」
\item 先生「うんうん。」
\item 私「あとはフェミニズムとか。ツイッターにはびこってる、内容を理解してないような、アカウントとか見てると、まあまったく違うこと言ってたりとか、いやちゃうやんお前らってなって︙︙」
\item 先生「うん、あのでも、既得権益とかそういう権力闘争の、反動もあるけど、これまでその、普通に扱ってくれとか、ただその、権利とか尊厳を認めてくれっていうだけで済む、話を、」
\item 私「はい。」
\item 先生「するべき人たちも、結局、権力闘争の枠組みの中で、あの運動してしまって、あの自分たちでも分かってないやんっていう風な、結果になることはたびたびあったからね。」
\item 私「あーー。」
\item 先生「なんか、それに関してはあのー、なんかどちらも、不勉強ってのがあるかなぁって思う。」
\item うち「︙︙なんかほんとに見てて、なんか、まあうちもそんなに、周りの人間ほど、きちっと勉強できてないですけどまだ。それでもわかるヤバさが、」
\item 先生「ふふふふふ」
\item うち「すごい、というかまあそれが今の流れの大多数にあって。」
\item 先生「うんうん。」
\item うち「なんかもっと、そう、根本の話をしたらなんか、権力、とか、印象操作とか、」
\item 先生「うん。」
\item うち「なんかこう吉本の事件と、選挙と、なんかおっきいのが二つあったじゃないですか。おんなじ期間で。」
\item 先生「うんうん、うん。」
\item うち「で、なんかもう全部、ちゃんと追って行って、」
\item 先生「うん。」
\item うち「選挙の事とか全部調べて追っていってたら、」
\item 先生「うん。」
\item うち「え、いや、や、ヤバい、としか言いようがない。」
\item 先生「ははっはっはっ。」
\item うち「いやなんかほんとに汚いことしかやってない、し。」
\item 先生「うん。」
\item 私「それこそ、吉本の件に関してはなんか、もう、もーう、社長ヤバいし。」
\item 先生「うーーん。」
\item うち「あの、やり方が。」
\item 先生「結局その程度の人間でも、あのそんな上まで行って社会回せることになんねんなっていうもののまあ悪い例やんな。」
\item うち「あ、はい。あと、あの、選挙期間中の、あの、演説、で、」
\item 先生「うん。」
\item うち「野次飛ばしただけで、野次飛ばしただけでそう警察が、動いて警察に連れていかれたあれ見て、」
\item 先生「はっはっはっは。」
\item うち「ヤバ、え、ヤバっ!って、独裁政治ひいてるやんって。」
\item 先生「あぁ、まあ、んー色々あるけどな。いや野次に関したら、うーん、まあ、反対側もあったしなってのもあるしねぇ。うん、ていうかまああの、結局、あのー、野次に関してはあの、警察やってくんのは極端やんって思う反面、」
\item うち「はい。」
\item 先生「なんかあの反体制活動、」
\item うち「あぁ。」
\item 先生「みたいなこと、に、主眼が置かれてしまって、」
\item うち「うーん。」
\item 先生「あのー昨今政策論争でどうこうではなくって。」
\item うち「あーーー。」
\item なんかあの
\item 先生「邪魔したろうだけの。」
\item 先生「気に入らんから、あの邪魔して排除して、なんかちょっと、まあ共産党的な、言えばその革命を成し遂げようみたいな。」
\item うち「うんうん。」
\item 先生「なんかそういう問答無用のこう、力の、なんか、なんかあの、行使みたいなのが出てくる場面はあるので。」
\item うち「はい。」
\item 先生「まあある程度ちょっと、にらみを利かせる必要もあるよねとは思ったけどね。」
\item うち「あぁ。」
\item 先生「んー、全員、全部が全部あのー、野次飛ばしたらいかんって言うんやったらそもそも国会の中で、議論してる人間、議論してる人間が一番野次飛ばしてるんやけどね。」
\item うち「いやぁ︙︙」
\item 先生「うん。」
\item うち「え、逆、の、なんかほかの政党に、野次飛ばしてる、の、あと、あーの、なんか、プラカードとかあげてるのは、全く警察取り締まらない︙︙」
\item 先生「はっはっはっは。」
\item うち「あーやばって思って。あとは、あれですよね、なんか、こう、あーの、候補者が回ってるところに、」
\item 先生「うん。」
\item うち「旅行バスが、二、三台、」
\item 先生「あぁ。」
\item うち「ついて回って、降りて、プラカードみたいなん渡されて安倍政治賛成みたいなプラカード、で、あぁそう、プロ市民の方々なのかな?演説の前で。」
\item 先生「はーはー。」
\item うち「こうやってみんな、掲げてる。で、それが終わった後に、中身が何かは知らないですけど、モノが配布されてるんですよなんかちっちゃい。」
\item 先生「えーーそれはもうほんま選挙法違反やんね、なんかにしても。」
\item うち「まあ中身がなにか、押さえられてないから何も言えないですけど。」
\item 先生「うん。」
\item うち「もし、あの現金じゃなくてもティッシュでもだめじゃないですかあれ。」
\item 先生「そうね、うん。」
\item うち「換金性のあるものだったら全部だめじゃないですか。」
\item 先生「うん、うん。」
\item うち「んまあ、もし、その、ポチまあポチ袋みたいなやつだったんですけど、」
\item 先生「うん。」
\item うち「ポチ袋オンリーでも、中に何にも入ってなくても極端に言えば換金性があるってなってしまうから、」
\item 先生「うんうん。」
\item うち「えダメなんじゃないって。」
\item 先生「うん、そうねえ。なんか協力してくれてありがとう感謝状みたいなんだけやったら別やけどねぇ。」
\item うち「まあ、そのなんか、バスに乗ってる人間と、」
\item 先生「うん。」
\item うち「その、候補者が、あの完全に繋がってるかどうかの証明もなされないとっていうのもありますけど。」
\item 先生「まぁ、まあね、誰がやっても、でもなぁ。」
\item うち「はい、まあそこはもうなんかもうトカゲのしっぽ切りみたいにしたらどうとでもなるじゃないですか。」
\item 先生「うんうんうん。」
\item うち「したらなんかそもそも今持ってる権力を、いっぱい、金と権力をもってるやつらがやっぱりどんどん、設けていく仕組みになるんやなあって。」
\item 先生「あぁ、まあでもそんなことをやってほんとにみんなが先導されればね。」
\item うち「うん、いやぁでもまぁ、結果、見てもある程度多分先導、先導されてるじゃないですかぁ、あれ。先導、されてない人間が、一〇〇%だったら、文句は、」
\item 先生「いやぁ、でもあの、民主党政権でごりごりしたっていうのは本当にまだ引きずってるところはあるわねぇ。」
\item うち「あぁ、いやぁですよね。はい。なんか、その辺も色々含めて世の中の人間ほんとに何にも、情報を仕入れないしものを考えねえし。」
\item 先生「うん。」
\item うち「こう、何も、そう見えるものしかみねえんだなぁってひしひしと感じて、」
\item 先生「ふふふふふ」
\item うち「その二つの、事件を見てて。あぁ、しょうもなって。こんな世界におってやる義理ねえわっていう、考え方がまた、」
\item 先生「うんうん。」
\item うち「より一層、こうメーターが、ドゥルルっと」
\item 先生「ふふっ。」
\item うち「たまった感はあります。」
\item 先生「うーんでもそいつらに直接の被害をこう、こう受けたらこうそういう発想はあっていいけど、」
\item うち「あぁ。」
\item 先生「いやねそいつらと付き合うために生きてるわけ違うけどねぇ。」
\item うち「あぁーそうですねぇ。」
\item 先生「うん。」
\item うち「あぁ、そっかまあそうだよなぁ。」
\item 先生「なんかあのー俺の目に入るものが汚いってのと単に俺が不愉快、不快やわっていうことに関しては、」
\item うち「あぁ。」
\item 先生「なんか自分の人生邪魔されてるわっていう感覚はちょっと出てもいいとは思うけど。」
\item うち「あぁ、うーん、、やなんかこう、うーん、不快、うーん、不快なんですよねぇなんかすげえ、こう大雑把で、暴力的な言い方したら馬鹿嫌いなんで、好きじゃない、ねーまあこう、あの、バカにこの、うーん、今まですげえ迷惑をかけられてきてたんで。」
\item 先生「んー。」
\item うち「馬鹿嫌いなんですよ。そう。」
\item 先生「たぶんねぇ、それはあの、また、こんなこと、言い方がちょっと申し訳ないけど、あの一時期あの、大体あの、そんなことになる。」
\item うち「あー、うーん。」
\item 先生「うん。で、それあの通り過ぎるねんいつか。」
\item うち「あぁ。」
\item 先生「馬鹿大っ嫌いっていうのが、あの︙︙なんか、反応しなくって、あのー、スルーして、っていう風に、やり過ごして、も、そのー、それ以上に、あのー自分の世界っていうかまあ馬鹿じゃない人たちとの関わりとか、」
\item うち「うん。」
\item 先生「まあ自分の居場所の整備とかが進むとかっていうのが起きると、そういう、そこをなんか世界のあちこちにあるけど、関わらずに、あのー自分の生活はそこそこの、あの、居心地の良さで作れるなぁみたいな。」
\item うち「あぁー。」
\item 先生ことができてきたら、あの、関わらないように生きようっていう、あのー知恵は、ちょっと使うけど。」
\item うち「はい。」
\item 先生「あのーいちいち反応してやる程の、」
\item うち「うんうんうんうん。」
\item 先生「あの生き方っていうか、あの労力って、」
\item うち「はい。」
\item 先生「実はいらんねんなっていうような、あの、とこまで行けるわ。」
\item うち「はー。」
\item 先生「んーいやあのー実際色んなとこで、どうしても関わらなあかん時に馬鹿がいるっていうのはいつでもあることあり得る事やねんけど、」
\item うち「はい。」
\item 先生「あのー、そういうやつとのやり取りを、自分がしないと生きていけないっていう事に、ならずに済むっていうような生き方って選べるのよね。」
\item うち「うーん。」
\item 先生「んーー、それはね、あのー、いずれ終わる。その、馬鹿を許すでもなく、」
\item うち「はい。」
\item 先生「自分があのー、馬鹿を愛せるようになるでもなく、」
\item うち「あー。」
\item 先生「うん。なんかあのー別の、あの、ところで生きてたらもういいやって、いってそれはそのー、受け入れるとか許すとかじゃなくって、もう、あのそんなもんなんやろって言って、なんかあの、フーンって言って、それ以上の意味付けとか、あの、関連性の整理とかせずに、終わらせれるようになっちゃう。」
\item うち「あぁ。」
\item 先生「うん。あのー、でも、その前にだいぶ、その馬鹿嫌いって言って本気で憎んだり腹立てたりするひと手間って、残念ながらいるんよ。」
\item うち「うんうん。」
\item 先生「その先に進むためにも︙︙でもあの、感覚は、あの僕は、全然間違ってないと思うよ。うん︙︙だって言ってもわかんないし、変わってもくれないんだもん。」
\item うち「うーー、いやぁそうなんですよねぇ。ほとほとあきれ果てる。」
\item 先生「うん。なんかそれが自分に影響を及ぼすっていうことがあり得ること自体がもうそもそも、許しがたいっていう風になっちゃうので。」
\item うち「はい。」
\item 先生「うーん、で、あとそこに反応するとか自分が、なにかこう、影響を受けるみたいなことがなくっても、まあいずれ、自分の生きてる場所とか生き方、に関しては、あのーそこなしで、成立させれてるなっていう風に、言える日が来ると思う。」
\item うち「うん。」
\item 先生「こう、上手くやっていこうじゃなくって、なんか馬鹿のコミュニティとか、馬鹿の世界・文化っていうのはあるんだな、ふーんで終わる、みたいなね。」
\item うち「うーん。」
\item 先生「まあそれとこうなんか、自分なりの、その、もの、受け入れるとか自分が満足できるもの、で、違うものがちゃんと自分の周りに用意されてしまうのよ。いずれ。同じように生きてるはずやのに、なんかあのー、絶対あんなん嫌っていうのと、あの似たようなこと、自分はこれがいいわっていうのが、おんなじ、全くおんなじことをしてるような、あの見え方で、勝手に、あの成立するっていう風な、ことが起きてくる。︙︙それをなんかある意味生きる場所作りっていうか、自分で生き方決めるっていう事かなぁって思ってるんね。」
\item うち「あぁー。」
\item 先生「あのー、うん、すごい、大事な、健全な感情なので。」
\item うち「はい。」
\item 先生「これは、こう、こういう場所で言うたらいかんこと、特に田所さんに言うべきことじゃあんまないけど、基本その、なんか馬鹿を憎むとかっていうことから始まって、さっきの特権意識じゃないけど、人を蔑む差別するとか、あの嫌うとか、あのー、不寛容になる、っていうのって、本能的に必要な感情なんよね。」
\item うち「あぁ。」
\item 先生「ある程度。まあそこをあのー、折り合いつけるというか、ある、そういう気持ちが考え方反応の仕方が良くないっていうと、それこそ白痴になんのよな。」
\item うち「んーー。」
\item 先生「なんかあの、博愛とか平等とか、そういうこと言うてはみるけど、その宗教みたいな、もんで、無批判にすべてを、容認しよう、っていう風に、そんなことをすると、あのー、自分で考えることをやめただけやないか。っていう、ことになっちゃうのよね。」
\item うち「うん。」
\item 先生「やっぱりそういう事をそのー、悪い感情とか反応も、込みで、ただ、自分が、自分の中でそれが、自分にとって正常とか、自然とか、正当性がある。っていう風に、思えることでちゃんと構成されることが大事。なぁ。まあそれに今の、感情を、あの、言ったらいかんことやろなという風に、なんか気が引ける方向を勝たせるんじゃなくて、」
\item うち「はい。」
\item 先生「あの自分の、そう思う事はでも自分は正しいから、そこを進めるしかない、言う風になる方が、まあ自分にとって、絶対的に正義なんよね。」
\item うち「んーー。」
\item 先生「まあそれがあの、田所さんの場合はあの、人を踏みつけようとか、自分の権力を、あの大きくしようではない、っていう事をわかって、やれてるから、まああのうまく辻褄が合う形で、あの納得いくように、作れると思うので、であのー、なに、はた迷惑でない形でね。」
\item うち「うん。」
\item 先生「だから、良いことやと思うけどね。︙︙うん。まあ、ちょっと世の中で事件とかそういう神経の動きがあると、いっぱい矛盾が目に付く、目に付くというかあの、考えの、思慮の浅さが目に付くよね。」
\item うち「はい。」
\item うち「やぁそうなんですよねーー。ほんとに。」
\item 先生「それがあの、分かってしまう、時点でしゃあないわ。」
\item うち「ふふふふっ」
\item 先生「はっはっはっは」
\item 先生「分からない幸せがあんねんなっていうのを」
\item うち「んーーーー」
\item 先生「横目で見ながらね。」
\item うち「そうですねぇ。まあ、数がいる以上、仕方がない。」
\item 先生「ほんとに、色んな、あの、性質の人間が、あのうまいことして散らばるようにしか設計されてないので。」
\item うち「うん。」
\item 先生「良いものが、あの増えるようには設計されてないっていう風に、まあ受け入れなあかんからねぇ。」
\item うち「はい。」
\item 先生「まあ、まあぼちぼちと。」
\item うち「はい。」
\end{description}
\end{document}
