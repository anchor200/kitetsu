%\documentclass[10pt,a4j]{utjarticle}
\documentclass[b5j,twoside,twocolumn]{utarticle}
%\documentclass[b5j,twoside]{utarticle}
%\documentclass[b5j,twoside,twocolumn]{utbook}
\setlength{\columnsep}{2zw}
\usepackage{bxpapersize}
\usepackage{pxrubrica}
\rubysetup{<hj>}
\usepackage{endnotes}
\usepackage{multicol}
\usepackage{plext}
\renewcommand{\theendnote}{[後注\arabic{endnote}]}
\renewcommand{\thefootnote}{\arabic{footnote}}
\usepackage{pxftnright}
\usepackage{fancyhdr}
\setlength{\topmargin}{5mm} % ページ上部余白の設定(182mm x 257mmから計算)。
\addtolength{\topmargin}{-1in} % 初期設定の1インチ分を引いておく。
\setlength{\oddsidemargin}{21mm} % 同、奇数ページ左。
\addtolength{\oddsidemargin}{-1in}
\setlength{\evensidemargin}{17mm} % 同、偶数ページ左。
\addtolength{\evensidemargin}{-1in}
\setlength{\footskip}{-5mm}
%\setlength{\marginparwidth}{23mm}
%\setlength{\marginparsep}{5mm}
\setlength{\textwidth}{225mm} % 文書領域の幅(上下)。縦書と横書でパラメータ(width / height)の向きが変わる。
%\setlength{\textheight}{150mm} % 文書領域の幅(左右)
\makeatletter
\def\@cite#1#2{\rensuji{[{#1\if@tempswa , #2\fi}]}}%%
\def\@biblabel#1{\rensuji{[#1]}}%%%
\makeatother
\usepackage{enumerate}
\usepackage{braket}
\usepackage{url}
\usepackage[dvipdfmx]{graphicx}
\usepackage{float}
\usepackage{amsmath,amssymb}
\newcommand{\relmiddle}[1]{\mathrel{}\middle#1\mathrel{}}
\usepackage{ascmac}
\usepackage{okumacro}
\usepackage{marginnote}
%\usepackage[top=15truemm,bottom=15truemm,left=20truemm,right=20truemm]{geometry}
\usepackage{cleveref}
\usepackage{plext}
\usepackage{pxrubrica}
\usepackage{amsmath}
\usepackage{fancybox}
\usepackage[dvipdfmx]{graphicx}
\usepackage{cancel}
\setcounter{tocdepth}{3}

%\renewcommand{\labelenumi}{(\Alph{enumi})}
\usepackage {scalefnt}
\makeatletter
\@definecounter{yakuchu}
\@addtoreset{yakuchu}{document}% <--- depende on class file
\def\yakuchu{%
\@ifnextchar[\@xfootnote %]
{\stepcounter{yakuchu}%
\protected@xdef\@thefnmark{\theyakuchu}%
\@footnotemark\@footnotetext}}
\def\yakuchutext{%
\@ifnextchar [\@xfootnotenext %]
{\protected@xdef\@thefnmark{\theyakuchu}%
\@footnotetext}}
\def\yakuchumark{%
\@ifnextchar[\@xfootnotemark %]
{\stepcounter{yakuchu}%
\protected@xdef\@thefnmark{\theyakuchu}%
\@footnotemark}}
\makeatother

\usepackage{atbegshi,etoolbox}

\newcounter{newfoot}
\patchcmd{\footnotetext}{\thempfn}{\thenewfoot}{}{}

\newcommand{\evenfootnote}[1]{%
  \ifodd\value{page}%
    \footnotemark%
    \AtBeginShipoutNext{%
      \stepcounter{newfoot}\footnotetext{#1}%
    }%
  \else%
    \stepcounter{newfoot}\footnote{#1}%
  \fi%
}


\pagestyle{fancy}

\title{日常生活における「箱」のイマージュの役割について}
\author{野上貴裕}
\date{\vspace{-5mm}}
\setcounter{page}{101}

\begin{document}
\maketitle

\setlength{\footskip}{-2mm}
\lhead[]{【論考】}
\chead[]{}
\rhead[日常生活における「箱」のイマージュの役割について]{}
\lfoot[]{\thepage{}}
\cfoot[]{}
\rfoot[\thepage{}]{}

\renewcommand{\thesection}{\pbox<y>{\arabic{section}}}
\renewcommand{\thesubsection}{\thesection―\pbox<y>{\arabic{subsection}}} 
\renewcommand{\thesubsubsection}{\thesubsection―\pbox<y>{\arabic{subsubsection}}} 

\let\yakuchu=\endnote
\renewcommand{\footnoterule}{\noindent\rule{100mm}{0.3mm}\vskip2mm}
%\tableofcontents
\thispagestyle{fancy}
\section*{〈前置き以前のつぶやき〉}
 本稿の構成は以下のようなものです。「0 前置き」から「1―2―1 隠喩とイマージュ」までは主にイマージュ一般について論じています。「1―2―2 「小箱」のイマージュ」ではガストン・バシュラールにおける「箱」のイマージュを検討し、「2 日常生活における箱」では筆者独自の「箱」イマージュエッセイを書いています。ですので「箱」にのみ興味がある方は1―2―2以降をご覧ください。それではよろしくお願いします。
\setcounter{section}{-1}
\section{前置き}
\subsection{日常の批判}
この原稿は日常生活の批判のために書かれる。ここでの「批判」は日常から脱出するための一つの方途を意味する。ところで、日常の日常性を特徴づけるのは習慣である。習慣化された認識や行動をそれ自体として問いに付すことはあまりないだろう。生活しながら一々懐疑などやっていては生きてゆけないからだ。完全に習慣化された日常もまたあり得ない。私たちは日々新しいものに出会っているはずだ。さらに、習慣に頼り切っていてはそこに不測の事態が発生した際に対処が難しくなるということも考えられる。ここに日常を反省してみる気運が生じる。改めて考えてみることで日常を対象化し、\textbf{そこへの介入によって日常の「世界」を変える可能性}が生まれるのではないか。


また日常を反省してみることは、そこに働く権力関係へと目を向けることにも繋がる。私たちの普段の行動や認識の様式は自然に出来上がったわけではない。それらは文化や環境、一方では個々人の意志や思惑などといった様々な要素の相互作用のなかで形成されたものだ。私たちがもはや前提としてさえいる様々な概念や制度は歴史的に、諸権力の作用のなかで作り上げられたものである。ミシェル・フーコーが主体概念や監獄制度などへの分析を通じて行った批判は、まさにこの事実を明るみに出すものであった。日常ごく当たり前だと考え、あるいはそれ以上に\bou{それらをもって}思考や行動を起動させているところのモノやコト、これらが多かれ少なかれ形づくられたものであるという可能性を考えること。これは翻って私たちの生そのものが決して自明のものではなく、何らかの権力作用に浸されている可能性を考えることである。


もちろんこの意味での「権力」から逃れることはできない。できないがその力場の様子を変化させることは可能なのではないか。これはジュディス・バトラーが性に関する議論の領域で試みたことであろう(『ジェンダー・トラブル』)。あるいはシチュアシオニスト・インターナショナル(以下SI)\footnote{シチュアシオニスト・インターナショナルは一九五〇年代から七〇年代にかけてフランスを中心に活動した集団である。彼らは文化・芸術・社会・政治を統合的に批判し、一九六八年の五月革命において大きな影響力をもった。}が都市生活における「習慣」に対してとった戦略である。
\subsection{シチュアシオニスト・インターナショナル}
私たちは自分の生活環境を様々なイマージュによって塗り分けている。それらのイマージュの差異は道路の僅かな傾斜、人口の密度、交通量など様々な要因によって生み出されている。自らの生活の拠点である家からある地点までの「距離」の感覚は、決して等質的な地図の示す距離とは一致しないだろう。台地や山の上に建てられた大学には行きづらいし、大きな国道に横切られたスーパーに向かうのは億劫である。そうして何のストレスもなく辿りつけるコンビニを選んでしまう日もある。何の店もない住宅街よりは、小さいながらも活気のある商店街の方に注意の多くの部分を払っている。こうした様々な心理的イマージュは文字通り街に「起伏」を与えているだろう。この街の「起伏」が個人の日常生活にどのような影響を与えているのかを研究しようとしたのが、SIによる「心理地理学」の試みである。SIの指導者であるギー・ドゥボールは論文「都市地理学批判序説」において「心理地理学」について以下のように述べている。\footnote{シチュアシオニスト・インターナショナルの機関紙『アンテルナショナル・シチュアシオニスト』はその全訳が「シチュアシオニスト・オンライン文庫」としてインターネット上に掲載されている(https://situationniste.hatenablog.com/)。「この機関誌の編集規則は集団的編集である。個人によって書かれ、個人の署名のあるいくつかの記事も、われわれの同志全員に関係があり、その共同の探求の個別的側面と見なされなければならない。われわれは文学雑誌や美術雑誌のようなかたちで生き残ることには反対している。 『アンテルナシオナル・シチュアシオニスト』に発表されたすべてのテクストは、出典を明記しなくても、自由に転載、翻訳、翻案できる」。ここにおいて行われる「転用」の戦略は、著作権の原理を解体するための「犯罪行為」である。単に権威を貶めるだけではなく、物が元々位置を占めていた場所をずらし、それにより新たな価値を生むという側面がある。
}
\begin{quote}
地理学は、たとえば、土壌の構成や気象状況のような、一社会の経済的編成に対して、そして、そこから、その社会が世界を把握する仕方に対して、一般的な自然力が及ほす決定的な作用を考察する。〔一方で〕心理地理学は、意識的に整備された環境かそうでないかにかかわらず、地理的環境が諸個人の情動的な行動様式に対して直接働きかけてくる、その正確な効果を研究することをめざしている。(『状況の構築』p.305、括弧内筆者による補足)
\end{quote}
木下誠(一九九三)のまとめを援用するとするならば、ドゥボールはここで「具体的には、都市における個人の行動パターン、住民がそれぞれの地域に対して持つ心理的イメージ(悲しい街、幸せな街など)、異なる地区の心理的関係、ある地域への接近方法、二点間の最短距離、都市における心理的切断線、都市のパサージュ・出口・防衛店など」\footnote{木下誠「ギー・ドゥボール『スペクタクルの社会』訳者解題 付「シチュアシオニスト・インターナショナル」の歴史」(ギー・ドゥボール『スペクタクルの社会』木下誠訳、筑摩書房、2003所収)p.216}の情報を盛り込んだ地図の作成を企図している。心理地理学の成果を盛り込んだ地図を描くのである。しかしそれはどのようにして? また何のために?


どのようにしてか。彼らはここで「漂流dérive」と呼ばれる方法を用いる。漂流とは簡単に言えば、様々な環境的な要因によって規定され習慣化された普段の移動から、敢えて逸れるような行動を指す。\footnote{漂流の具体的な規定に関してはドゥボールの論文「漂流の理論」に詳しいのでそちらを参照してもらいたい。}例えばドゥボールはこんな例を挙げている。
\begin{quote}
いかがわしいと見なされながらも、われわれの周りで常に人気を博してきたある種の悪ふざけ──例えば、夜中に取り壊し中の建物に入り込んだり、交通ストの時に、でたらめな方向に車を走らせて混乱を悪化させる目的で、パリの街をひっきりなしにヒッチハイクして回ったり、侵入を禁じられているパリの地下納骨場(カタコンブ)の地下道をさまよい歩いたりすること──までもが、まさに漂流の感覚以外の何ものでもない、より一般的な感覚に属する行為となりうる。(『状況の構築』p.145)
\end{quote}
ただし、こうした行為は意識的になされなくてはならない。意識的に漂流を行うことで、ある意味で学的に、都市の心理的分節articulationを探求することができるのである。


では何のためにか。SIの目的とは「状況の構築」であるとされる。彼らは「構築された状況situation construite」を「統一的な環境と出来事との成り行きを集団的に組織することによって具体的かつ意図的に構築された生の瞬間」と定義した(『状況の構築』pp.42-43)。政治や文化、歴史、資本主義、あるいはそれらに由来する合理性などによって押し付けられ、引き受けさせられた受動的な状況が一体いかなるものであるかを探究し、それらを批判することによって具体的にそして意図的にまさに自分たちの生の瞬間を構築すること、これがSIの目的である。そしてその状況の構築の現場は「社会」などといった抽象的な場ではなく、まさに今私たちが生活している「ここ」である。そして彼らは彼らの生活する「都市」において心理地理学に取り組み、またそれを基にした批判を遂行したのだ。
 もちろんSIは一つの例である。しかし本稿の目標は彼らの試みと方向を同じくする。それは日常生活において日常性と化してしまっているものに目を向け、それを露わにした上で自らの生の状況を組み直すことである。その第一歩として日常を構成するイマージュへと目を向けてみる、というのが本稿の大まかな方向性である。

\subsection{イマージュを考えるとはどのようなことか}
具体的な記述を開始する前に、日常生活におけるイマージュを考えるとは一体どのようなことであるのかについて少し詳しく記しておきたい。私はここまで「イマージュimage」という語を特に説明なく使用してきた。しかし、タイトルに用いた用語について触れないままに進むのは不誠実であろう。たんにイマージュといっても、絵画や映像などを含めた表象作用をもつ「モノ」をイマージュと呼んだり、あるいは人やモノに与えられた感覚的印象などを意味することもあるからだ。これらの定義とは少し異なり、本稿での「イマージュ」は、基本的には「しかじかのものとして予め判断を下された表象像」を指す。以下ではジャン=ポール・サルトルがその想像力論で見出したイマージュの定義を援用することで、ここでのイマージュという言葉の含意について触れておきたい。


サルトルは『想像力の問題\sl{L'imaginaire}\rm』のなかでイマージュ的な対象を知識savoir、感情性affectivité、運動感覚sensation kinesthétiqueが表象的要素あるいは身体corpsをもったもの(=受肉incarnation)であると規定した(IMR 161-162/156-157および216/212)。例えば登り坂を想像してみよう。そのイマージュは、それがどのようなものであるかという知識と、登るのがつらいなどといった感情と、坂を上登る際の身体的な運動の感覚などが具体的な表象像としての登り坂の表象に合わさったものとして現れるのではないだろうか。サルトルによれば、このような綜合的性格をもったイマージュが眼前には\bou{ない}ものとして現れるとき、私たちは想像している、ということになる(IMR 32/27)。


また、このようなイマージュ的対象は知覚の対象と明確に区別される。知覚の対象は「観察」することができる。つまり、今現在見えているのとは違う他の面を見てみることによって、その対象についての知識を増やすことができる。初めてサイコロを見た人は、見る面を増やすことによって、向かい合う面の数の合計が七になることを知るかもしれない。またさらに観察を続けることによってそのサイコロが石でできているということに気が付くかもしれない。しかし、想像されたサイコロはどうだろうか。上記のサイコロに関する記述を読みながら想像したサイコロについて考えてみてほしい。そのイマージュは想像されたその瞬間にすでに六つの面を持つ立方体であり、向かい合う面に書かれた数字の合計は七であり、なおかつ石でできた物体だったのではないだろうか。このようにイマージュはあらかじめその内に決定や判断を含んでいる。ただし、例えば「六の面に汚れがついている」という判断を、そのイマージュとしてのサイコロを\bou{観察することによって}引き出すことはできない。あらかじめ「六の面に汚れのついたサイコロ」を想像することによってしか得られない。想像されたものから何か新しい内容をもつ判断や知識を引き出すことはできないのである。サルトルはこの点においてイマージュと知覚を決定的に区別する。


では私たちの日常生活のことを考えてみよう。私たちが生きている「世界」はほとんどがこのようなイマージュによって出来上がっているのではないだろうか。\footnote{ただし、恐らくサルトルはこの意見には反対するだろうと思われる。サルトルにとってイマージュは「無」であるから、つまり現実には属さないという措定を与えられた対象であるからである。(Sartre 1940: 32/27)もちろんサルトルの言う「現実」および「非現実」を広義のリアリティの領域に含みこむという解釈ができないわけではないが、この見解に関しては詳細な検討が必要であり、ここで触れることはできない。(というか筆者はこのことについて来るべき卒論で触れるつもりである。)}私たちが一度に相手取ることのできる知覚の対象は、私たちがふつう「現実世界」として考える場のほんの一部である。しかし私たちは目の前にある観察可能な対象を相手にしつつも、その存在を支える背景としての世界を、まさに背景\bou{として}描き出している。この世界はほとんどがイマージュによって構成されていると言ってもいいのではないだろうか。あるいは、「~へ行こう」などと考えるとき、私たちはその行き先を想像する。もちろんその場所が実在することを疑いはしない。しかしやはりそれはイマージュであろう。このように私たちはイマージュなしで、あるいはイマージュを生み出す想像力なしで世界を生きていくことはできない。


加えて、私たちは目の前にあるものに対してさえ、イマージュをもって関係することがある。というかほとんどの場合がそうであるといっても過言ではないだろう。サルトルはイマージュの規定として、それが眼前にはないということあるいは少なくとも眼前にあるものとして考え\bou{ない}、という限定を設けた。しかし、私たちは観察可能な対象を、すでに獲得したイマージュをもって理解するということが当然ある。例えば目の前の一人の人間を「日本人」のイマージュをもとに「こういう人間だ」と判断する場合などがそうであろう。サルトルはすでにその危険性を警告していた。\footnote{「もしひとたびその思念を形成するさいに想像的態度をとったとしたら、それに直接近づくことは私たちには決して許されなくなる。私たちはいつまでもイマージュからイマージュへと移り行くことになるであろう。理解とはいつまでも果しのつかぬ運動となり、それはあるイマージュに対するに別のイマージュを以てし、さらにそのイマージュに対するにまた別のイマージュを以てする精神の連鎖反応となり、かくて無限に続くべき可能性を蔵する。このような無限の退行に、露わな思念の端的な直観をとって替わらせるためには、意識態度の根本的変更、真正の革命、を実践することが必要であり、すなわち、非反省的次元から反省的次元へと移行することが必要である。」(IMR 223/219)
}しかしそうしたイマージュによる理解を排除してしまえば、私たちの世界はなんと貧しいものになるだろうか。この意味でもまた、私たちの生にイマージュは必要であるし、本稿が中心的に扱うのはむしろこの目の前のものに関係する際に働くイマージュである。


ところで、それらのイマージュは各々が単に個別のものであろうか。想像された諸々のイマージュのなかには、何か共通の原器のようなイマージュに結びつけられるものがあるのではないか。それらのイマージュは何か原-イマージュのようなイマージュの変奏として捉えられるのではないか。地理学者のイーフー・トゥアンは『空間の経験』のなかでこのようなことを言っていた。建築物の空間について話している場面である。
\begin{quote}
われわれは、自然界にはそのような建築のイメージよりも遥かにもっと強力なイメージがあるのではないだろうか、という疑問を抱くことがある。〔中略〕たしかに、このような疑問には一理ある。しかし人間は、人の手によってつくられた、知覚できる形態と尺度をあらかじめ経験することなしに、自然のなかのこれらの特徴を素直に理解することができるかどうかは疑わしいのである。自然はあまりに拡散しているために、そして自然のなかの刺激物はあまりに強力で相互に争っているために、人間の精神と感受性によって直接理解することはできない。(Tuan 1977:110-112/198-200)
\end{quote}


私たちは観察可能な、つまり知覚可能な形態や尺度を基準に世界を測る。ここでトゥアンは建築物によって感性の能力を客観化する、という事態について語っている。私たちはギリシャの神殿を見ることによって「静謐」を学び、バロック建築を見ることによって「たくましい、生命力に溢れたエネルギー」を学ぶ。単純化すれば、大きな建築物を見ることによってはじめて、私たちは広大さの意味を知るのである。この過程は「ぼんやりとした感情と観念」を、「客観的なイメージをもつことによってin the presence of objective image明確なものにする」過程だと考えられる(Tuan 1977:110/198)。つまり、建築物によって得られたイマージュとそこに現れる感性や範疇をもって、私たちは世界のさまざまなものに対して関係していく。知覚や認識のみならず、感動、愛着、欲望などもこうした原初のイマージュを足場にしてはじめて膨らんでいくことができる。ここで言われる「イメージ」は個別のものではないだろう。ギリシャ神殿によって受肉させられた「静謐calm」の観念は、ギリシャ神殿の表象像を伴いながらも他の事物にも適用できるものである。従ってそのイマージュは一定の普遍性を帯びていると考えることができる。私たちはギリシャ神殿だけではなく、他の静謐なものに触れることによって徐々に静謐さの観念を豊かにしていくだろう。もちろんギリシャ神殿の観察的印象は時と共に失われていくだろうが、静謐さの観念に結びついている限りでのギリシャ神殿の表象像は依然として残り続ける。そうして豊饒さを獲得していくイマージュには、個別の「静謐なもの」と、一方では「静謐さ」の観念の両方が重ね合わされているのではないだろうか。イマージュはこうして個別性(ギリシャ神殿をはじめとした諸々の「静謐なもの」)と普遍性(「静謐」さ、「静謐さ」のイマージュ)を同時に含みこむようなものである。もしそうであるとするならば、この普遍性\tbaselineshift =2.5pt ------\tbaselineshift =4.0pt本質、と呼んでもいいだろう\tbaselineshift =2.5pt ------\tbaselineshift =4.0ptの部分に焦点を当て、記述していくという試みも可能なのではないか。


さしあたり私たちはイマージュを、原初に獲得された「表象像を伴う観念」と考えることができる。そしてこの原初的イマージュは、その観念的な部分を、その適用対象を変更することで豊かにしていくことができるものである。トゥアンは特に建築物について語ったが、この見解をもう少しミクロにして、私たちが普段手に取る、例えば「箱」のような対象を考えることもできるだろう。ふと改めて考えてみると、私たちは日常生活のそれなりに多くの場面で「箱」のイマージュを用いてさまざまな対象を捉えているのではないだろうか。私たちが日常相手にしているさまざまな箱について考え、またそこから原器的な「箱」のイマージュを取り出してみたい。これが次の節以降の目標である。


ここまでを一旦まとめておこう。まず私たちは日常生活世界と呼ばれるものがどのような分節を持っているのかを明らかにしようという、比較的大雑把な目標を立てた。そのために、日常生活におけるイマージュに目を向けるという方針を立てた。そしてイマージュについて考えるとはどういうことであるかについて筆者の考えを述べた。イマージュは個別的な表象的要素とともに、そのイマージュの原器として働く普遍性を帯びた成分とから構成されている。その普遍的な部分に目を向け、私たちが日常で諸々のイマージュをどのように用いているのかを考えること。これがイマージュについて考えることの意味であった。こうした前提の上で、本稿では「箱」のイマージュについて扱いたい。


次節以降の構成を記しておく。まずはガストン・バシュラールが「箱」のイマージュについて語った箇所について検討する。その後、その記述を導きの糸として筆者自身の「箱」のイマージュについて記述を行いたい。あらかじめ断っておくが、本稿におけるイマージュの記述は客観性を目指すものではない。それでは見ていこう。

\section{バシュラールにおける「箱」のイマージュ}
\subsection{バシュラールのイマージュ論}
バシュラールの晩期の著作に『空間の詩学la poétique de l'espace』という本がある。本章では「引き出し、小箱、戸棚le tiroire, les coffres et les armoires」と題された章を参照しつつ「箱」のイマージュについて考えていきたい。ただし、ここではいわゆる読解という作業は行わない。バシュラールは詩的イマージュをウジェヌ・ミンコフスキーの言う「反響」によって特徴づけた。筆者もそれに倣い、バシュラールの提示するさまざまなイマージュによって反響させられ、展開していった思考の記述を試みたい。


とはいえ、『空間の詩学』においてバシュラールが何を試みたのかについて簡単な説明はしておこう。バシュラールは詩的イマージュを「現象学的」に規定することを目指す(PE 9/11)。ただし、これは金森(一九九六)も指摘するように、フッサール的な意味での現象学ではなく、単に現象を研究の中心に置くという意味での「現象学」である(金森 1996:247-248)。そしてそのイマージュの現象学を遂行するためには、原理や基礎などといったあらゆる学問的・客観的基盤を放棄し、イマージュの現れを直接に研究しなくてはならないとされる。この直接性は「反響retentissement」によってもたらされる。


この「反響」概念は精神病理学者ミンコフスキーによって導入された。私たちは音を聴くとき音に浸透されるような感覚をもつ。音の響きは世界の全てを満たすように思われるが、しかし私たちはそのことによって世界と合一することはない。それは私たち自身がまるで\bou{井戸}のように、内側で音を反響させるような仕方でその音を聴くからである。音により他の存在者と「共鳴résonance」することはあっても、あくまで「私」は音をその内に反響させる個的な存在者なのである。\footnote{Minkowski, Eugène. 1999. Vers une cosmologie: fragments philosophiques, Payot, (1ère édition, Paris, Aubier-Montaigne, 1936)(『精神のコスモロジーへ』中村雄二郎・松本小四郎訳、人文書院、1983)および、佐藤愛. 2016「ウジェヌ・ミンコフスキー研究――分裂性と同調性」博士論文(筑波大学)を参照。}ミンコフスキーはこのような反響の性質を、私たちと世界との関係において記述した。一方でバシュラールはこれを詩的イマージュへと適用する。バシュラールは以下のように述べる。
\begin{quote}
われわれは感情の共鳴によって\tbaselineshift =2.5pt ------\tbaselineshift =4.0pt豊かさがわれわれのうちにあるのであれ、詩そのものにあるのであれ\tbaselineshift =2.5pt ------\tbaselineshift =4.0ptとにかく豊かに芸術作品を受容することができる。ところが詩の現象学的研究は、極めて遠くかつ深く沈潜することをねがうので、方法上必然的にこの感情の共鳴をとびこえなければならない。共鳴と反響という現象学的姉妹語を鋭く感じとれる可能性がここにあることに注意しなければならない。共鳴は世界のなかのわれわれの生のさまざまな平面に拡散するが、反響はわれわれに自己の存在を深化することを呼びかける。共鳴においてわれわれは詩をききとり、反響においてわれわれは詩をかたり、詩はわれわれのものとなる。反響は存在を反転させる。詩人の存在がまるでわれわれの存在のようにおもえる。そして多種多様な共鳴が反響の単一の存在からうまれてくる。もっと簡単にいえば、これは熱烈な詩の読者なら熟知の印象であるが、詩がわれわれを完全にとらえるということなのだ。(PE 13/17)
\end{quote}

比喩的ではあるが、まず初めに詩が響く。私たちは詩の響きに共鳴することでそれを受容することができる。そしてその響きは受容者である私たちにおいて反響し、私たちの奥深くまで響き渡る。しかしここで反転が起こる。あまりに深く反響した詩の響きは、まるで私たち自身がそれを響かせている\bou{詩人}であるかのように思わせさえする。そうしてむしろ私たちにおける反響こそが、詩の伝達である共鳴を引き起こしたかのように思えてくるのである。これこそが\bou{詩にとらえられる}という事態である、そうバシュラールは考える。


また以下の記述。
\begin{quote}
われわれはこの反響によって、ただちに一切の心理学や精神分析学をとびこえて、自分のなかに素朴に生まれでる詩の力を感じる。われわれが共鳴や感情の反射や自分の過去の呼び声を経験できるのは、この反響ののちのことである。しかしイマージュは、表層をゆさぶるまえに、深部にふれている。またこれは読者の単純な経験にもあてはまる。詩をよんでわれわれにあたえられるイマージュは、こうして真にわれわれのイマージュとなる。イマージュはわれわれのなかに根をはる。たしかに外部からうけいれたものだが、自分にもきっとこれを創造することができた、自分がこれを創造するはずだった、という印象をもちはじめる。イマージュはわれわれのことばの新しい存在となる。イマージュは、そのイマージュが表現するものにわれわれをかえ、これによってわれわれを表現するのだ。いいかえれば、それは表現の生成であり、またわれわれの存在の生成である。ここでは、表現が存在を生成する。(PE 14/18-19)
\end{quote}

反響によって至る地点とは、詩によって与えられたイマージュが、むしろ私たち自身の生み出すイマージュとなるような場所である。私たちはさまざまなイマージュを用いて、対象を、そして自己をとらえる。そのときのイマージュは、そのようなイマージュを用いる存在\bou{として}の私たちを「表現」するものとなる。いやむしろイマージュこそが、そのイマージュによって表現されるものとしての存在を生み出すのである。ここにイマージュの力がある。


バシュラールはこれらのイマージュ論を特に「詩的イマージュ」を論じる上で展開している。しかし、より広くイマージュ一般についても同様の事態が考えられるのではないだろうか。もしそうであれば、私たちが日常においてどんなイマージュをどのようなものとして捉えているか、またそれをどのような仕方で用いているか、を考えることは私たち自身がどのような存在であるかを考えることに等しいと言える。このようにバシュラールの文章を読んだ上で、それを筆者が前置きにおいて提示した姿勢と重ね合わせ、引き続きイマージュについて考えていこうと思う。


バシュラールはこのあと(幸せな)空間のイマージュについて書き記していくことになる。彼の記述は体系的なものではない。彼は与えられた詩的イマージュにおける、主観的な空間の質的印象とでも呼べるものを描き出そうとしているように見える。筆者もまずはそれにつき従う形で、「箱」のイマージュについて検討してみたい。

\subsection{バシュラールの「箱」と「内密」}
でに予告した通り、『空間の詩学』の第三章「引き出し、小箱、戸棚」\footnote{原語ではle tiroire, les coffres et les armoiresと題されており、岩村行雄による邦訳では「抽出 箱 および戸棚」とされている。しかし、「抽出」は「引き出し」の方が読みやすく、また「箱」は本文中他の箇所でboiteの訳語として用いられてる一方coffreの訳語には「小箱」が用いられており、ここでも「小箱」とした方がよいのではないか、という意図の下訳語を少し変更した。またimageは「イメージ」と訳されているが、本稿における用語の統一性の観点から「イマージュ」と書き換えることにした。}について見ていきたい。ここでバシュラールは「内密intimité」\footnote{「内密性」に関してバシュラールはすでに『大地と休息の夢想La terre et les rêveries de la repos』(1948)において詳しく書き記しているが、敢えてその内容についてここでは触れないことにする。}のイマージュについて考える過程において引き出しや箱について考えているため、筆者の歩みとは逆方向に進む。しかし、箱と内密性との関係についての記述は大きな示唆をもたらしてくれる。


まずは引用。「内密のイマージュは、引き出しや小箱とかたくむすばれ、錠の偉大な夢想家である人間が自分の秘密をしまいこみ、隠している一切の隠し場所とかたくむすばれている。」(PE 100-101/148)バシュラールは箱的なものと内密性とを結びつけて考える。内密性とは、私たちの「家」に対する触覚的な関係によく見出される性質である。自分の家での安心、外部世界からの隔離の感覚あるいは外部世界そのものの消失、包まれていると感じるときの温かさ。内密性は無限に関係する。限界や距離の感覚は外部世界の秩序の下にのみ存在する。適切な温度のもと、温かい布団にくるまった経験のある人なら誰しも経験したことのある、世界に対しての\textbf{距離なし}の経験。これがここで内密性と呼ばれるもののことである。箱は内密性を包蔵する。ただしここで注意しなくてはならないことがある。それは箱が内密性という概念の隠喩ではない、ということである。またしても少しばかり道を逸れよう。
\subsubsection{隠喩とイマージュ}
バシュラールは本章の初めの部分で暗喩métaphoreとイマージュとの違いを強調する。暗喩は\bou{自己自身とは異なる}心的存在être psychiqueに関係する。例えばある人々が「まるで犬だ」と言われるとき、その人々は「犬」という存在の持つ「忠誠心に篤いが卑しい」という性質をもって捉えられている。ここで「忠誠心に篤いが卑しい」という性質を表すために「犬」という存在を持ち出すことに必然性はない。偶然「犬」が慣習的に用いられているだけであり、同様の性質をもつ他の存在でもいいわけである。ここから、暗喩的な表現(「犬」)は、それが指し示す何か(忠誠心に篤いが卑しいという性質)と外在的な関係をもつに過ぎないと言える。暗喩とそれが表しているもののあいだにあるのは恣意的な関係なのである。一方でイマージュは存在の現象であるとバシュラールは言う。イマージュは存在と直接的な関係をもつ。というよりもイマージュは一つの存在である。この点についてもう少し見てみよう。


バシュラールはベルクソンの例を挙げる。ベルクソンは概念によってのみ遂行される哲学を批判するために引き出しの隠喩を用いた。カテゴリーという名の引き出しが考えられている。新しい対象に出会った私たちは、それをどの引き出しに入れようかと考える。これが「認識する」という事態である。しかし、こうしたモデルによって認識や知能を考えることをベルクソンは批判する。ただし、こうした比喩はひととき論争的なものであったとしても、その批判が一旦落ち着いた先に再考されるべき代物ではない。ベルクソンの批判の矛先はカントや素朴な科学的思考であったが、それらへの批判の内実は、比喩であるところの引き出しをいくら詳細に分析したところで理解しようがないのである。「このあわただしい説明は、暗喩は偶然の表現にすぎないし、これについて思想を展開するのは危険だということを、指摘するものにほかならない。」(PE 103/152)


そう、暗喩は「偶然の表現」に過ぎないのである。つまり、それは分かりやすく説明を行うための一時的な道具に過ぎない。従って隠喩をもとに思考を展開することは危険であるとすら言える。これは筆者の実際の経験であるが、大学の授業でヴィトゲンシュタインの『哲学探究』を読んでいたときのことであった。その日読み進めた箇所のなかで、ヴィトゲンシュタインは言語を例えて「朝鮮アザミ」のようなものだと記していた。丘沢静也による訳が偶然「玉ねぎ」になっていたこともあいまって、この例は何ぞやという方に話は進んだ。朝鮮アザミも玉ねぎも、皮のような部分を順に剥がしていったところで芯には辿りつかず、ただ剥がした皮だけが残るような構造をもっている。恐らくヴィトゲンシュタインの意図はこうであった。皮の部分がことばの各々の「使用」であり、それらを一枚ずつはがしていってもそのことばの「意味」や本質という名の芯には辿りつかない。語の意味とはむしろ朝鮮アザミや玉ねぎの全体であり、つまるところその使用こそが意味である、と。しかし「玉ねぎには芯がある」というある学生の発言から議論はおかしな方向に進んでいってしまった。朝鮮アザミや玉ねぎの画像を検索し、その姿から言語の姿を探ろうとするという場面もあった。これはまさに本末転倒であろう。暗喩において重要なのはそこで言及されている対象ではない。暗喩はあくまで観念や概念の理解のための補助的な手段に過ぎないのである。暗喩をもとに思考を展開することの危険性とは上記のような事態を指しているのではないだろうか。


「引き出し」に戻ろう。ベルクソンの例において暗喩を用いる場面では、実際の知能を「引き出しのようなもの」として考える、ということになる。一方イマージュにおいては、引き出しこそが知能そのものなのである。前者においては知能なる何ものかがすでに存在することを確認した上で、それを引き出しのようなものとして考えるか、あるいは別の何かのようなものとして考えるか、が問題となる。一方後者では、まさに引き出しのイマージュこそが知能\bou{であり}、知識とは引き出しにしまわれることで所有されたもののことである。アンリ・ボスコの小説に登場するカル=ブノアという人物は樫の整理箱に記憶や知識を収め、管理していた。それはすぐにでも取り出すことができるのである。彼にとってその整理箱の引き出しこそが記憶であり、知能であったのだ。(PE 103-104/152-153)


バシュラールの言うイマージュは「~のように」という仕方で表される代物ではない。それは一つの存在である。つまり「~である」という仕方で言い表されるものである。私たちは日常生活において、物的存在であれ心的存在であれ、それを「~のようにみえるもの」などとして捉えているわけではない。そうではなくまさに「これは~である」という仕方で存在していると考えている。バシュラールの「イマージュ」はこの\bou{存在}のレベルで考えられている。それは何か実在するものの再現前=表象などではない。


それではようやく彼の言う「小箱」のイマージュについて見ていこう。
\subsubsection{「小箱」のイマージュ}
小箱の幾何学géométrie du coffretと秘密の心理psychologie du secretとのあいだには相同性がある。小箱の幾何学的特徴、それは側面、底面、そして蓋によって形作られているという点である。また小箱には内部と外部がある。こうした特徴と「秘密」に関わる心理が相同性をもつとされる。例えば、ある小説の登場人物は、自らの娘への贈り物として絹のスカーフを選ぶか日本漆の小箱を選ぶかに悩む。彼は「娘の内気な性格にふさわしいと考えて」小箱を選ぶことにする。(PE 109/160)小箱は「内部」を作り出す。内気な娘の目指す「内」を生み出す効果がそこにはある。しかし、閉鎖された心理というものを描く際に、その拒絶や冷淡な態度、沈黙などを数え上げるだけでは十分ではない。つまり箱の外側の面だけを見ていてもそうした心理の本性は理解できない。むしろその人の「新しい箱をひらくときの積極的な悦びの瞬間」を見なくてはならないのである。娘は、父から小箱を贈られることによって、そこに秘密をため込む許可を受けた。新しいその箱を開き、そこに内密性を見出すその瞬間にこそ彼女の閉鎖的なたましいの心理状態が輝くのだとバシュラールは言う。


ここには小箱が「開かれる事物objets qui s'ouvrent」であるという重要な事実の反響がある。閉じられた箱は一つのものであるかのように振舞う。そこに何かが入っているという顔はしない。しかし、それが開かれたとき、その内密の吸引力は外部を消し去ってしまうほどの威力をもつ。引用しよう。

\begin{quote}
小箱、とくにわれわれがもっと確実に所有している小箱は、ひらかれる事物である。小箱は、しめられると、ふたたび事物の共同体へかえされる。すなわちそれは外部空間のなかに位置する。だがそれはひらかれるものなのだ。〔中略〕(小箱が開かれる瞬間には、)外部は一気にけしさられ、すべてが新奇であり、驚愕であり、未知である。外部にはもはや意味はない。最高のパラドックスだ。すなわち新たな次元、内密の次元がひらかれたために、主体の次元は無意味になってしまった。(PE 112/165、括弧内補足は筆者による)
\end{quote}
実際に箱のことを考えてみよう。お菓子箱でも、段ボール箱でも、何でもいい。開かれるまえの箱は外部の秩序に整然と従っているように見える。収納ボックスのことなどを考えると分かりやすいだろうか。箱の持つ幾何学的な外観は秩序のイマージュにふさわしい。しかし、それはどこかよそよそしさを感じさせる。閉じられた箱は私たちを拒み、沈黙している。ところが箱が開かれたとき、その内部からはそれまで露にも見せなかった親密さ、新しさ、驚き、未知が溢れ出してくる。これはまさに箱が開くその瞬間に最高潮となるだろう。私たちの意識からは外部空間の秩序など消え去り、その驚きに取り込まれてしまう。対象としての箱に外在的に、距離をとって向き合う主体のような次元は無意味となる。そこで私たちは世界に対して純粋な内面性となるのである。バシュラールはこのようなことを言っているのではないか。\footnote{このバシュラールの記述を受けた金森(1996)による記述も参考になるので引用しておく。「小箱の心理学はより明らかだ。何かを隠す秘密の場所。内気な娘にはスカーフではなく小箱がふさわしい。どんな秘密でも凝縮して収納できる不思議な空間。小箱は閉じられるとき普通の事物の秩序に与し、公の事物連鎖のなかの一員になる。それは例えば机の上にあり、部屋のなかにある。ところが小箱は開かれるために、その瞬間に自分の価値を膨張させるために存在するのだ。小箱が開かれたとき公共的外部はどこかに消え去り、あとはただ新奇と驚愕だけが姿を現わす。」(金森 1996: 251)}


この内密性の次元は無限を含みこむ。バシュラールはジャン=ピエール・リシャールの次の言葉を引き、その無限性を言い表している。「われわれは絶対に小箱の底には到達しないのだ」。(PE 113/166)そう、われわれは絶対に箱の底には到達できないのである。なんということか。確かに私たちは小箱を開けてその底に手を触れることすらできる。しかし、再びその蓋が閉じられたならどうか。小箱の内部にはまた、内密性が充満することになる。こうして私たちは小箱が閉じられる限り、その底には辿りつけないのである。この無限性は箱のイマージュにとって枢要なものだ。バシュラールはこうも言う。「物は、開いた小箱よりも、閉じた小箱のなかの方に、いつもたくさんはいっていることであろう。評価はイマージュを殺す。\bou{想像すること}はつねに\bou{体験すること}よりも偉大であろう」(PE 115/168)この箱にはこれこれのものしか入っていないなどという評価は小箱の無限性を殺してしまう。箱のイマージュには常にその内部の無限性が伴っている。


箱の内部には無限の「秘密」が隠されている。そして秘密というものはそもそも各々が自分自身を収める箱をもつ、とバシュラールは言う。
\begin{quote}
秘密はみなそれぞれに小箱をもち、しっかりとしまい込まれたこの絶対の秘密はなんら力の作用をうけない。ここでは内部の生は記憶と意志の綜合を経験する。ここには鉄の意志があるが、外部に対するものでも他者に対するものでもなく、対立の心理を超越したかなたにある。われわれの存在の思い出のまわりには絶対の小箱の安心がある。(PE 111-112/164)
\end{quote}
ここには二つのことが書かれている。まず、小箱のうちにあると考えられる秘密は、その外部からの作用を受けることはない、ということ。それに、箱の内部には記憶と意志との綜合があるということである。


まずは前者について。秘密は外部からの作用を受けない。秘密は自らが秘密であることを明かさないからである。開かれた秘密は秘密ではない。秘密には位相があることをバシュラールは二重底の箱のイマージュを用いて示す。その中には第一の箱とその奥に第二の箱がある。箱に付けられた錠は泥棒をだますための仕掛けである。鍵を開けた泥棒は第一の箱の秘密に満足してしまう。しかしこれは「秘密」ではない。秘密としての秘密は、常に、見ることのできる秘密の外部にある。深い秘密の底に私たちは達しえない。


そして後者。箱のイマージュに親密に接する生、あるいは箱のイマージュの存在する世界に住む生は、箱の中にしまい込まれた「秘密」としての記憶と、それを秘密たらしめている意志、しかも「鉄の意志」との綜合を知ることになる。箱は記憶と意志の綜合である。この意志は箱の外部、つまり公共性の領域から内部を守ろうとする意志ではない。それはむしろ絶対的な内面性、あるいは内密性\bou{への}意志である。この意志においては外部も、他者も存在しない。私たちの主観の個別性は記憶によってもたらされる。私の記憶としての記憶を、私秘的なものとして抱え込むことで、従って箱にしまうことで私は私の内面性たりうる。この内面性\bou{への}意志こそがここで「鉄の意志」と呼ばれているものである。箱は内面性への意志によって、私の記憶を「秘密」として小箱に収める。秘密をもつということへの安心感、すなわち「絶対の小箱の安心」こそが私たちの存在を自己たらしめるのである。


少し先へ行き過ぎてしまったかもしれない。要約に意味があるとは思えないが、一度まとめてみよう。バシュラールにとって重大なことは、まず箱には外部と内部があるということである。そして箱には蓋がついている。蓋が閉じているあいだ、箱は外部の世界に存在する。しかし一旦蓋が開けば、箱の内密性は外部の世界を無意味にするほどに己を開き、私たちはそこに取り込まれてしまう。筆者がバシュラールを通して提示しようと試みた「箱」のイマージュは以上のようなものである。


次節以降では上記のような「箱」のイマージュを保持しつつもバシュラールのテクストからは離れ、筆者が日常生活において見出した「箱」について記述していきたいと思う。記述はより主観的なもの、非体系的なものとなるだろう。(ここまでの記述が客観的であったり体系的であったと言うつもりはもちろんない。)
\section{日常生活における「箱」}
私たちの日常には箱が溢れている。小物入れの箱、靴箱、段ボール箱、お菓子箱、おもちゃ箱、ごみ箱、煙草の箱、重箱、弁当箱、宝石箱、救急箱、マッチ箱。箱と名付けられてはいないものの、Blu-rayデッキも箱に見えるし冷蔵庫やあるいは本なんかも箱に見えるかもしれない。カラオケボックスもそうだし、ライブハウスを「ハコ」と呼称する人々もいる。アイドルの箱推し? バシュラールの箱のイマージュは確かに重要であるが、ここではもう少し、有り体に言えばわかりやすいところから始めてみよう。箱には様々なかたちのものがあるが、ここでは差し当たってスタンダードなものと思われる方形の箱を想像しておいてもらうのがいいだろう。


明確には箱でないものの、箱のイマージュによって捉えられている対象にはどのようなものがあるだろうか。


いきなり抽象的な例で申し訳ないが、「ブラックボックス」と呼ばれるものがある。私たちはこの言葉を、仕組みはよく分からないが、分からないままに機能はしているようなもの、に対して用いる。ブラックボックスは普段目を向けられることがない。ふとした瞬間に目を向けると、それがブラックボックスであったことを知るのである。これは私たちの「理解」がどれほどブラックボックスに支えられているかを示している。私たちはブラックボックスへの入力と、そこからの出力さえ分かっていればその内実をわざわざ知らずともそのまま進んでいける。なんなら理解したとさえ思うのである。しかしこれはより原理的な問題である。そもそも理解にブラックボックスでない部分はあるのか。無限に細かく見ていくことが可能である。従って私たちは、ある理解がブラックボックスを伴ったものであるかそうでないかという点に究極的に拘りはしない。むしろあるものがブラックボックスであると認識されたときに\bou{それが開かれるか否か}、という点に賭けがなされるのである。ブラックボックスは「理解できないもの」を秘密の内部として閉じ込め、しかしそこに生まれた箱を理解可能な秩序のうちに置き入れるという役割を持つ。確かにブラックボックスはブラックで中身が見えない、という点ではネガティブなイマージュを背負わされている。しかし、思考の上では極めて重要な役割を果たしているのではないだろうか。


あるいは意識という例。これまた抽象的かもしれない。少なくとも現代日本に生きる私たちは意識を箱のイマージュに近いものとして捉えているのではないだろうか。私の痛みは私にしか分からない。私の思考は私の内面に属しているので人に見られることはない。私には私しか知らない内面があり、それはひとに見せる外面とは異なったものである。これらの言説は果たして自明なものであろうか。むしろさまざまな原子的な事実(もちろんこんなものは権利上の存在に過ぎないが!)を箱のイマージュの形に成形した結果生まれてきた考え方なのではないか。 先日大阪大学内で開催した映画祭において、ブラジルの\tbaselineshift=2.5pt[\tbaselineshift=3.41666ptSSEX BBOX\tbaselineshift=2.5pt]\tbaselineshift=3.41666ptという団体の映像を上映した。彼/女らのキャッチコピーは〝Sexuality out of the box〟というものであった。しばしば「心の問題」として語られがちなセクシュアリティの問題がどのようなイメージの下扱われているかが見える(もちろんセクシュアリティの問題は「心の問題」などではないと思う)。私たちはよく「心を開く」とか「心を閉ざす」とかいう言い方をする。どうやら心は開けたり閉じたりすることのできるものらしい。やっぱりこれは箱じゃないか?


あまり日常的ではないかもしれないがノアの箱舟なんてものもある。一体、ノアの箱舟以上に有名な箱はあるまい。ノアの箱舟は救済の箱である。神はノアに向かって言う。「あなたはゴフェルの木の箱舟を造りなさい。箱舟には小部屋を幾つも造り、内側にも外側にもタールを塗りなさい。」(創世記6.14、新共同訳)そして神は洪水を起こした。「洪水は四十日間地上を覆った。水は箱舟を押し上げ、箱舟は大地を離れて浮かんだ。水は勢力を増し、地の上に大いにみなぎり、箱舟は水の面を漂った。」(創世記7.17-18)箱の中には地上の全ての種がつがいとなって収まった。そこは一つの世界であった。箱は大地から離れ、水面を漂う。箱は本質的に大地から切り離されるものである。全てがそこに基づけられる大地から、箱の内部は切り離されている。安全地帯としての箱の内部。


神話に近いものと言えば玉手箱やパンドラの箱のイマージュも捨てがたいだろう。玉手箱を開けば「ほんとうの姿」になるという話は示唆的である。パンドラの箱を開けば希望だけが残る!


また、\textbf{シュレーディンガーの猫は、おそらくシュレーディンガーの箱である。}


少し別の話題。私たちはしばしば、ものを整理する際に箱を用いる。誰も着ない服や二度と見ない書類の入った箱が家のなかにいくつあるのか数えたことのあるひとはいるだろうか。箱は便利である。箱の中にどれだけのカオスが収められていようと、それが収められている以上整っているという外観を与えてくれる。箱には中になにが入っていても依然としてひとつの箱であるという性質がある。このことはバシュラールが閉じられた箱について言っていたことに符合するだろう。しかしこの性質はある重大な問題を引き起こすことがある。それは箱の中に人間が入れられるような場面である。箱は「管理」という思想に接近しうるのである。


それほど田舎でない街を見てみよう。箱でできている。バシュラールも言った。「パリには家がない。大都会の住民たちはつみかさねた箱boitesのなかにすんでいる。」(PE 53/78)多少の例外はあるにせよ、日本の都市の住民の多くは箱の中で生活しているだろう。なぜここまで箱なのだろうか。もちろん、合理的だからである。こうした合理主義的建築は、ミース・ファン・デル・ローエに代表される一九二〇年代のモダニズム建築にまで遡れる。エドワード・レルフのまとめた、モダニズム建築の原理を見てみよう。
\begin{quote}
\begin{enumerate}
\item  建築物を塊りではなく空間を内包する容積として扱うべきである。これは、鉄やコンクリートなどの新しい建材によって、壁はもはや負荷を支えるものではなくなり、内部空間を構造力学的な必要性からではなく、用途や目的に応じて仕切ることができるようになったからだ。つまりそれは、建築を幾分なりとも優美な箱に単純化してとらえることを意味していた。
\item 建築の外観は、垂直的要素と水平的要素、およびその繰り返しから構成されるべきである。
\item デザインの工学的特質を示したり、無装飾のものに美しさを与えるために、完璧な技術と繊細な均整が強調されるべきである。
\item 建築とその周囲の環境はすべて、機械時代のデザインの理念を反映する量産工業技術の特質を持つべきである。\\          (Relph 1987:115/180-181)
\end{enumerate}
\end{quote}
まさに「箱」状の建築を推進する理念である。こうした合理主義的建築は、第一次世界大戦後の深刻な住居不足に対処するものとして採用されたようだ。そこでは限られた空間にどれだけ多くの人間を詰め込めるか、が要求されていたのであろう。レルフはこう評している。
\begin{quote}
機械時代のデザイン理念を反映させるという考え方は、機能主義と呼ばれることが多い。しかし実際には、機能主義は個々の建築の目的にはそれほど合致してこなかったし、現在でもそうである。建物は水漏れして、騒音が伝わりやすく、空間は使いにくい。利用者の立場から見ると、モダニズムとは、疑うことを知らず理解力に劣る大衆に押し付けられたエリート主義の前衛的な美的感覚と思われているのが一般的らしい。(Relph 1987:115/181)
\end{quote}
住む側からすれば不便なものであったとしても、管理する側からすれば箱を積み重ねたマンションやアパートは都合がいいのである。方形の箱を積み重ねればデッドスペースは減る。箱のそうした「幾何学的」特徴。おそらく建築工学的にも、そもそも箱は理にかなっているのだろう。とはいえ、私たちは知らず知らずのうちに箱のなかで飼われているのかもしれない。


私たちの多くは箱の中で生まれ、育つだろう。方形の部屋以外で育ったという方がいれば教えてほしい。筆者はふとした瞬間に、自分が空間的な単位として箱型の空間を用いていることに気がつく。しかし、最初に単なる拡がりの意識から空間の認識に移行する際に、最も親しんだ空間がドームであったとしたならばこれは変わっていたのだろうか。あるいはアントニ・ガウディの《バトリョ邸》などの波打った天井の下に生まれていれば。もちろん何の実証性もない意見ではあるが。


箱の家が管理という発想に繋がっているという見方をここまでは取ってきた。しかし、一方で箱としての家には魅惑もまた備わっている。こちらは箱の持つ内部の力に関わるものだろう。もちろん普段街を歩くときには箱の外側しか見ることができない。しかし、バシュラールも言う通り、想像力は箱の内部を夢想する。想像してみたことがあるだろうか。巨大な建造物を構成する一つ一つの箱の中で、あまりにも多くの人間が同時に生を営んでいるさまを。他人が作り上げた箱のうちに自ら入り込み、それでも名前をもつ人生を歩んでいく。なんと滑稽なことか。筆者も多分に漏れてはいない。しかしながらそこには一方で強烈なエロティシズムがあるようにも思う。各々の箱に内密性が、そしてその無限性がある。バタイユの言うようにエロティシズムが「過剰」をその特徴とするならば、箱によってもたらされる「無限性」の次元はまさに過剰であり、従って箱はエロティシズムの一つの源泉であるともみなせるだろう。外的世界、つまり公共的世界の要請する規律によって押さえつけられた在り方から溢れ出すものがエロティシズムである。有限に思える日常生活に点在するブラックボックスの中には「過剰」としての内密の無限性が潜んでいるのである。


またバタイユはエロティシズムの目的を「連続性の回復」であるとみなす。他者と分離され、個的存在という非連続な存在へと至らされた人間の連続性へのノスタルジーを満たそうとするものがエロティシズムである。少し長いが名文なので引用しよう。
\begin{quote}
生の根底には、連続から不連続への変化と、不連続から連続への変化とがある。私たちは不連続な存在であって、理解しがたい出来事のなかで孤独に死んでゆく個体なのだ。だが他方で私たちは、失われた連続性へのノスタルジーを持っている。私たちは偶然的で滅びゆく個体なのだが、しかし自分がこの個体性に釘づけにされているという状況が耐えられずにいるのである。私たちは、この滅びゆく個体性が少しでも存続してほしいと不安にかられながら欲してるが、同時にまた、私たちを広く存在へと結び付ける本源的な連続性に対し強迫観念を持ってもいる。私が語るノスタルジーは、私が挙げた基本的事実を認識していようといまいとまったく関係がない。もっとも単純な存在の分化と融合を知らない人であっても、自分が、無数の波に消えていく一つの波のようにこの世界の中に存在していないことで苦悩するということはありうるのだ。それはともかく、このノスタルジーが原因して、すべての人間のなかに三つの形態のエロティシズムが生じているのである。(Bataille 1957:21-22/24-25)
\end{quote}
箱の内部の絶対的な内面性について、バシュラールを検討した際に述べた。この内面性をバタイユの連続性に読み替えることは可能だろうか。箱の内部のエロティシズムは外部なき内面性への、従って連続性へのノスタルジーと重ね合わせることもできるのではないか。例えば、ラブホテルなどに向かって、そのような視線が強烈に向けられてはいまいか? ラブホテルという開きがたい「箱」は、内密性を私たちに特に強烈に意識させる建築物なのである。


ここまで、とりとめのない、ともすれば「てきとう」だと揶揄されかねないような記述を展開してきた。しかし、これほどまでに、私たちの日常には「箱」のイマージュが溢れている。もちろん上で取り上げた主題以外にも数多くの例があるだろう。特にまとめようとは思わない。読者の中に「なんとなく分かる」と言って下さる方がいることを祈ってやまない。


\section{おわりに}
本稿の目的は日常生活の批判であった。そして批判は日常生活からの脱出のための方途であった。もちろん一つの側面に過ぎないが、日常生活には筆者が「箱の外の秩序」と呼んだものが張り巡らされている。一方で箱の中には秩序によっては汲みつくせない、あるいは秩序からあふれ出る内密の無限性があった。私たちは普段あまり箱について、そして箱のイマージュについて深く反省することはない。しかし私たちは実のところ、有限な日常生活の彼岸を身近な「箱」のイマージュのうちに蔵しているのだ。「箱」を考えることは、私たちを日常からほんの少しずらしてくれるかもしれない。これが本稿の結論である。


この文章のなかに何か「反響」するような言葉やアイデアがあれば、ぜひあなたも日常生活におけるイマージュについて考えてみてほしい。「箱」でも、もちろん他の何かでもいいだろう。例えば「重力」。例えば「穴」。どうだろうか。もしそれを筆者に伝えてくれるようなことがあれば、それは望外の喜びである。一緒に議論しましょう。

\section*{参考文献}
\hangindent=1zw
\noindent PE:Bachelard, Gaston. 1961.\textsl{ La poétique de l'espace}, Les Presses universitaires de France, 3e édition, 1961, 215 pp.(Première édition, 1957)(『空間の詩学』岩村行雄訳、筑摩書房、2002)

\hangindent=1zw
\noindent Bataille, Georges. 1957. \textsl{L'érotisme}, LES ÉDITIONS DE MINUIT.(『エロティシズム』酒井健訳、筑摩書房、2004)

\hangindent=1zw
\noindent Debord, Guy. 1992. \textsl{La société du spectacle}, Gllimard. (Première édition, 1967)(『スペクタクルの社会』木下誠訳、筑摩書房、2003)

\hangindent=1zw
\noindent Relph, Edward. 1987. \textsl{The Modern Urban Landscape}, Groom Helm.(『都市景観の\rensuji{20}世紀』高野岳彦・神谷浩夫・岩瀬寛之訳、筑摩書房、2013)

\hangindent=1zw
\noindent IMR:Sartre, Jean-Paul. 2005. \textsl{L'imaginaire}, Paris, Gllimard.(Première édition, 1940)(サルトル全集第十二巻『想像力の問題』平井啓之訳、人文書院、1955)

\hangindent=1zw
\noindent Tuan, Yi-Fu. 1977. Space and Place:\textsl{The Perspective of Experience}, University of Minnesota Press.(『空間の経験』山本浩訳、筑摩書房、1993)

\hangindent=1zw
\noindent アンテルナシオナル・シチュアシオニスト1『状況の構築へ\tbaselineshift =2.5pt ------\tbaselineshift =4.0ptシチュアシオニスト・インターナショナルの創設』木下誠監訳、インパクト出版会、1994

\hangindent=1zw
\noindent 金森修 1996. 現代思想の冒険者たち 第五巻『バシュラール\tbaselineshift =2.5pt ------\tbaselineshift =4.0pt科学と詩』講談社

\hangindent=1zw
\noindent 湯浅博雄 1997. 現代思想の冒険者たち 第十一巻『バタイユ\tbaselineshift =2.5pt ------\tbaselineshift =4.0pt消尽』講談社

\hangindent=1zw
\noindent 浅井雅志 2012. 「猥褻・過剰・エロティシズム\tbaselineshift =2.5pt ------\tbaselineshift =4.0ptロレンス、サド、バタイユの性観念\tbaselineshift =2.5pt ------\tbaselineshift =4.0pt」(松山大学『言語文化研究』32:336-366)

\end{document}

