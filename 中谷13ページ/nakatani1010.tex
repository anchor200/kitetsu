%\documentclass[10pt,a4j]{utjarticle}
\documentclass[b5j,twoside,twocolumn]{utarticle}
%\documentclass[b5j,twoside]{utarticle}
%\documentclass[b5j,twoside,twocolumn]{utbook}
\setlength{\columnsep}{2zw}
\usepackage{bxpapersize}
\usepackage{pxrubrica}
\rubysetup{<hj>}
\usepackage{endnotes}
\usepackage{multicol}
\usepackage{plext}
\renewcommand{\theendnote}{[後注\arabic{endnote}]}
\renewcommand{\thefootnote}{\arabic{footnote}}
\usepackage{pxftnright}
\usepackage{fancyhdr}
\setlength{\topmargin}{5mm} % ページ上部余白の設定(182mm x 257mmから計算)。
\addtolength{\topmargin}{-1in} % 初期設定の1インチ分を引いておく。
\setlength{\oddsidemargin}{21mm} % 同、奇数ページ左。
\addtolength{\oddsidemargin}{-1in}
\setlength{\evensidemargin}{17mm} % 同、偶数ページ左。
\addtolength{\evensidemargin}{-1in}
\setlength{\footskip}{-5mm}
%\setlength{\marginparwidth}{23mm}
%\setlength{\marginparsep}{5mm}
\setlength{\textwidth}{225mm} % 文書領域の幅(上下)。縦書と横書でパラメータ(width / height)の向きが変わる。
%\setlength{\textheight}{150mm} % 文書領域の幅(左右)
\makeatletter
\def\@cite#1#2{\rensuji{[{#1\if@tempswa , #2\fi}]}}%%
\def\@biblabel#1{\rensuji{[#1]}}%%%
\makeatother
\usepackage{enumerate}
\usepackage{braket}
\usepackage{url}
\usepackage[dvipdfmx]{graphicx}
\usepackage{float}
\usepackage{amsmath,amssymb}
\newcommand{\relmiddle}[1]{\mathrel{}\middle#1\mathrel{}}
\usepackage{ascmac}
\usepackage{okumacro}
\usepackage{marginnote}
%\usepackage[top=15truemm,bottom=15truemm,left=20truemm,right=20truemm]{geometry}
\usepackage{cleveref}
\usepackage{plext}
\usepackage{pxrubrica}
\usepackage{amsmath}
\usepackage{fancybox}
\usepackage[dvipdfmx]{graphicx}
\usepackage{cancel}
\setcounter{tocdepth}{3}

%\renewcommand{\labelenumi}{(\Alph{enumi})}
\usepackage {scalefnt}
\makeatletter
\@definecounter{yakuchu}
\@addtoreset{yakuchu}{document}% <--- depende on class file
\def\yakuchu{%
\@ifnextchar[\@xfootnote %]
{\stepcounter{yakuchu}%
\protected@xdef\@thefnmark{\theyakuchu}%
\@footnotemark\@footnotetext}}
\def\yakuchutext{%
\@ifnextchar [\@xfootnotenext %]
{\protected@xdef\@thefnmark{\theyakuchu}%
\@footnotetext}}
\def\yakuchumark{%
\@ifnextchar[\@xfootnotemark %]
{\stepcounter{yakuchu}%
\protected@xdef\@thefnmark{\theyakuchu}%
\@footnotemark}}
\makeatother

\usepackage{atbegshi,etoolbox}

\newcounter{newfoot}
\patchcmd{\footnotetext}{\thempfn}{\thenewfoot}{}{}

\newcommand{\evenfootnote}[1]{%
  \ifodd\value{page}%
    \footnotemark%
    \AtBeginShipoutNext{%
      \stepcounter{newfoot}\footnotetext{#1}%
    }%
  \else%
    \stepcounter{newfoot}\footnote{#1}%
  \fi%
}


\pagestyle{fancy}

\title{獅子王宣誓}
\author{中谷拓也}
\date{\vspace{-5mm}}
\setcounter{page}{101}

\begin{document}
\maketitle

\setlength{\footskip}{-2mm}
\lhead[]{【小説】}
\chead[]{}
\rhead[獅子王宣誓]{}
\lfoot[]{\thepage{}}
\cfoot[]{}
\rfoot[\thepage{}]{}

\let\yakuchu=\endnote
\renewcommand{\footnoterule}{\noindent\rule{100mm}{0.3mm}\vskip2mm}
%\tableofcontents
\thispagestyle{fancy}

「\tbaselineshift=2.5pt------\tbaselineshift=4.0pt今ここに集ってくださる皆さま一人一人の祈り、そして平穏への感謝をここに込めて、祭壇に送ります。先代の兄から一人目の弟へ戴冠を受け継ぎます。そして末代の先まで皆さまの高らかな精神のもつ希望、そして永劫の秩序となることを信じて、私たちの王家はいつまでも人々が安寧に過ごすことを切に願っております。\tbaselineshift=2.5pt------\tbaselineshift=4.0pt」


ぼくは宣誓式の最中ずっと立っていただけなので時間の流れを長く感じていた。父の演説は本当に長かった。今日の式は予行みたいなものなのに多くの人々が集まって、会場の真ん中にいる人も壁際にいる人もそこに集った人々は全て一様にゆっくりと流れていく時間をただ父の言葉にじっと耳を傾けてやり過ごしていた。ぼくはもちろん緊張するべきなのだろうが実際には思ったほど体はこわばっていなかった。それよりもぼくの横に並んでいる親族たちがどう立っているのか肌で感じていた。弟は序盤は神妙になって聞いていたが、今はうつらうつらとしているのが分かった。徐々に眠たそうになっている弟を母は横からずっと肩をたたいたりして手際よく目を開かせた。その度になんとか背筋を伸ばしてしっかりするのであるが、より睡魔に参ってしまった叔父とその妻が中盤あたりから首を下げているのが分かっていた。従兄弟の兄さんや姉さんには案外何も起こらなかった。そこから先にはぼくの又従兄弟となる男女がいるのだがもうぼくにはどうなっているのかは分からない。なんせ後ろにもぼくからだと遠い遠い親戚が何人も並んでいて顔も知らない大世帯になっているから傍から右の壁を見ると変な見せ物みたいだった。


父はあのあと、ぼくの兄と司教の話は明日の朝になること、そのときには催しを盛大に行うのでどうか参加してしめやかに本番の宣誓を迎えてもらいたいということ、そしてさっきのぼくの予行宣誓とともにどうか明日の式が終わってからも親密にかかわりあっていただきたいということ、そして今朝がある名のある職人の最期であったのでまた葬送の儀式について説明しようということを話して、ようやく退場の令を出した。後ろの大扉が開いて会場から出ていく者もいれば、ぼくたち家族のところに向かって歩いてくる者もいた。後ろの親戚たちはもう挨拶を交わしていた。


ぼくの列の端の席から女の人が一人扉に向かって歩いていくのが分かった。マイペースな足取りで人混みの中に入っていくのを眺めながら、どうして会場外に出たのかが分からず、ぼくはその人のことは同じ学区に通っていてよく知っていたので、無意識的にその人のことを考えていた。ぼくと同世代の人たちがたむろって何か話し合っている。\\
「あの人また一人で行くんや」\\
「\tbaselineshift=2.5pt------\tbaselineshift=4.0ptやっぱり何かあるな、そうとしか考えられない」\\
「まあ、あの人、基本的にプライベートで動いているから」


ぼくもその類いの話題には参加したかったが、結局何もしなかった。


群衆の中から祖父と孫らしき二人がぼくたちのところに近づいてきたが、誰だか分らなかった。ぼくが少し戸惑ったのが顔に出たのかもしれない、母は先に気の利いた挨拶をした。祖父が会釈してぼくより背の高い孫も少し顔でお辞儀すると、孫がこっちに顔を向けた。ぼくはその顔に何となくなじみを感じたが、果たして誰だったのかよく気付けなかった。


「兄さまは今日はおられないんでしょうか」


祖父は少し残念がった様子でしゃがれた声を出して母に話した。\\
「夕方に帰ってくるんですよ」


ふとぼくは二人のうちの子どもの方の顔を眺めているとはっと思い出したので、それが顔に表れてしまい相手にも感づかれたらしく、\\
「\tbaselineshift=2.5pt------\tbaselineshift=4.0pt久しぶりだね、二ヶ月ぶりだよ」\\
と近づいてきた。彼は薄毛の少年みたいな感じのあるぼくと同教室にいた一つ上の先輩である。昔に見た彼の顔写真と同じような雰囲気が感じ取れた。彼は正面の弟に目を向けた。ぼくはその人に関してあることをひらめいた。\\
「あぁ、かわいいね、こんにちは」


弟は五歳にしてかなり体が大きく、頭は彼の肘のあたりまであった。彼は弟の肩を撫でているが年を聞くと驚くのかもしれない。祖父の方は兄が今日帰ってくるのだと分かるとほっとしたようで、母と楽しげにしていた。\\
「あの、こちらが例の遺品であります\tbaselineshift=2.5pt------\tbaselineshift=4.0pt」


祖父が渡したのは職人が最後に使っていた加湿器だった。母は祖母に渡しますと言って箱を持って行った。なすりつけようとしているのだろうかと思った。ぼくは何も話す気になれなかった。


しかし先輩はある愛嬌を持っているのを知っていたので、彼に外に出ようと誘ったら承諾してくれた。そういえば二ヶ月前の式の時にその人と会ったきりだった。彼と扉を出て建物の棟をつなぐ渡り廊下を歩いた。今回の廊下は上り下りもカーブもないまっすぐな廊下だった。その人は特に何もしゃべらなかった。前に会ったときただおめでとうとぼくに言ってくれたのを思い出した。天井が下がった廊下の入り口の前だった。今はぼくは、やはり何も話さずにいるのが奇妙なので、明日どう振る舞うべきか少し不安だと彼に話した。会話に間があいた。\\
「俺はきみを尊敬するけどね。よく受け入れたもんだなあ。大丈夫?」


さあ、とぼくは実質何もわからなかったので首を傾げて、もしかしたら余計なことをしゃべって空気を乱してしまったかもしれないと思った。内心お互いに敏感なのは分かっていたから、ぼくはその人が少し反応に困ったかもしれないと思った。ただ漠然とした不安を持っていたのは確かで、ぼくが受け入れて新しくこの島国の主になったところでたんに兄の補欠みたいな受け取れられ方をするのではないかと勝手に思っていた。現実は実際にそうなのだろうとぼくはこの繰り返しに繰り返した思考回路をまた一瞬で巻き戻して心を閉ざした。\\
「先生がきみに会いたいって言ってたから顔合わせたら」


ぼくは頷いた。大玄関から外に出た。薄い雲がかかった空がまぶしい気がした。彼は引き戸の枠を踏み越えながら肩を上げて軽く深呼吸した。ぼくも外に足を踏み入れると、階段の下に咲いた赤紫に輝くアザミを見つけてどぎまぎしたが、それからゆっくりと街路を眺めた。


さっきまでとは違って人々はそれぞれの目的地に向かって歩いていた。ぼくには一切関係ない目的を持って歩いていた。ぼくは何だかほっとした。後ろを振り向くとその人は口角を上げて上を眺めていた。


彼を河川敷まで連れてってあげると、案の定ぼくの又従兄弟の女の人がいた。今その女の人と今ぼくが連れてきた先輩が目を合わせたらどうなるかぼくは確認してみたかった。そう思いながら、何だか見てはいけないような気がしていた。隣のその人は咳払いをして、\\
「じゃあ、これで...」\\
と言って顔を見せずにその女の人の方向へ歩いていった。ぼくはもうこの光景は何度も妄想して見ていたが現実に視認したのはこれで二回目で、その時にやはり付き合っていたのかと気付いて二人の白い肌が目に映えた。二人は後ろを振り向いて立ったまま道路わきに咲いたコスモスを眺めていた。ぼくはずるいと思った。さっきしゃべった人たちが知っているのは女だけが片思いしているのだけだ。彼らは満更でもない様子で語り合っているのだ。ぼくはその人が歩いてくるのに女の人が気付く前に戻っていった。玄関には大きな荷物を持った学者たちが到着していた。\\



城は堀に囲まれて島の中央にある。城壁はなく教会を兼ねた大広間がある中央の館の横に塔が二つあるが後ろにより大きい頑丈なガラス張りの建物があり、その中に居館と沐浴場がある。城下町を中心にして街が広がっており、平野も山もあるので都会風のビルディング街もあれば田舎いなかしたところもある。市場は城下町の内部にもあるが普通は百貨店やレストランなどを上級民が利用していて、城下町外にはコンビニ、クリーニング、理髪店、本屋などなんでもそろっていて、電車や地下鉄といったものも通っている。ただし電車は宙に浮いた線路を通って大きな軌道を描きながら走っており、大抵は各地域のいくつかとビル街のところにしか発着しないので、地下鉄で行き来する人の方が多い。城の建物から城下町につながる跳ね橋を使う人は、今日みたいな特別な儀式がないかぎり使われず、また城下町の中と外では全く空気が違う。飛行機も船も少なく、海沿いの住民が漁をする時以外はほとんどこの島は閉ざされていて諸外国とのやり取りもめったにない。諸外国といっても灘の向こうの小都ぐらいの規模であるだけなのだが。東の海の向こうにはがいる。\\



研究班の一人が海の向こうの小都市で起こっている事件と、発掘してきたものがあるといって父と話していた。\\
「父様、こちらがその班から届いたものであります」


その人ははがきの束のようなものを渡した。父は一枚目を見て目をほんの少し細め、縦に横に一枚ずつ眺めていた。そして突然はっとした感じでわきの廊下に戻っていくと、もう何人か周りにたむろしていて父に付いていったのが分かった。\\
「兄ちゃん、あれ、あの写真は」


ぼくは確かにあれらは写真のようだと納得して、弟の両肩に手をのせた。弟はそれには反応せずぼんやりとしたあくびをした。居間に連れて行くと静かで弟は即座にソファに寝転んだ。ぼくは父がもらったものが何なのか見ておくべきかと感じて、なんとなく躊躇しながらもう一度廊下に戻った。


座敷にはもう何人も人が入っていて、皆がそれらの写真を渡しあい何度も手に取って目で確認して父を囲んで色々と喋っている。ぼくは何かしら事の些末だけでも把握しないとその場に来た意味がないので隣に座っている客から写真を取ってもらった。セピア色の岩みたいな暗い写真の中央に黒い人型をしたタール人形らしきものが写っているのが分かる。\\
「そいつが呪いの根源だとさ。きみも面倒な立場についたね」


そう話しながら座布団も渡してくれた。ぼくはこれが本物の物質なのかも判然としなかったものの、もうこれが何を示しているのかははっきりしていた。ぼくはおそらく必然的にこれを壊しに行ってもらうか、あるいは自分の手で締め上げねばならないのである。ここの島を出て海を渡り、大陸かどこかに祀られているこの人形を壊さなければ、各地で天変地異を起こしている呪いが明日ぼくが王となったときに降りかかるというのはもう何度か聞かされていた。実際に各地で起きていることが写真の束になって父の手元にようやくやってきたのだということに気付いて座布団を足元に敷こうと立ち上がって周りを眺めると、今この部屋にいる人々が手に持っている写真のそれぞれに写っているものが何か察しがついた。そばにいる二人の大人の会話を聞こえた。\\
「先の王が亡くなってからばかな息子を斥けて新しい皇帝が出たんだと聞いたんだが、それがどうも隣小国の王の私生児らしい。あぁもうむちゃくちゃでこの街はだめなんだろうね。」\\
「前代の父親から駄目だったんだよ。湖で遊ばせて長男だったかを死なせるなんていかれた因果、呪いも何も関係なかっただろうよ」


ぼくはその写真も見せてもらった。城砦の窓の一端が割れているのがあって、それ以外は何も分からない。呪いは人々の精神を蝕むのだと聞いていた。ぼくは父の視線を感じてゆっくりと目を向けた。だがぼくは目線が合った瞬間白い障子に目線をそらして怖じ気を隠そうと試みた。父を中心にあたりが急に静まり返った気がしたからだ。だが庭先に初蝉の声が聞こえ、ぼくがさとっていた架空の軽い沈黙は打ち切られたらしく、何事もなかったかのように議論が続いた。


するとすぐに母が入ってきて目が合った。


「今日はその子と何時に会いに行くの?\tbaselineshift=2.5pt------\tbaselineshift=4.0pt兄さんの迎えはどうする?」\\
昼から会いに行くつもりだったがそれまで時間があったのでそれまでどう過ごすか決めずにうやむやにしていたのもあって、ぼくは唐突に聞かれたので戸惑ったふりをして黙った。\\
「兄さんが駅に帰ってくるのは遅くなるかもしれないから、その子と先生のところにも挨拶しに会いに行くようにしなさい」


ぼくはうなずいて部屋をあとにした。それにしても父と母はぼくのことについてこの後どんな話をするだろうかと気になったし、ぼくの不安定な性格についても言いふらすのだろうかと妄想すると、その妄想にぼくの頭が占領されて廊下を歩くのもままならないぐらいだった。ぼくはとりあえず居間に戻って弟の様子を見たら、兄を迎えに行く時間までどう過ごすか考えた。


兄はいわゆる一時帰郷をしてくるのである。塔のちょうど真横から向こうにあるビル街で、二ヶ月前から広告代理店の営業員として働き始めた。昔からしかしはきはきとした兄の内部には弟であるぼくと似た性根があったのをぼく自身は知っていた。その曲がった性格はいつの間にか消え失せていたようだが、その節目については思い当たることがある。


冬の初め、野良猫が地面に伏しているのが見つかった。目撃者は何人もいた。夜の事故死だった。轢いてしまった本人に罪はなかったが、ぼくは母が泣いている兄を代弁するかのように彼に色々と訴えかけていたのをぼくは覚えている。兄は島中のあらゆる地域に棲んでいる猫の顔も暗記していたほどなので、怒りを抑えきれなかったのかもしれない。ぼくはずっと顔をうつむかせている兄を斜め後ろからじっと眺めた。水仙の花で供養した。やはり兄は顔を手でおさえていた。


それまで言葉を失っていた祖母は、何かを黙々と編むようになった。それが何日も続いたのでぼくと並んだ花火職人が聞いた。\\
「何をずっと、そうして\tbaselineshift=2.5pt------\tbaselineshift=4.0pt」


祖母はいきなり編み棒を投げ出して、雑誌を読み始めた。その後ぼくとその職人は外に出て菜の花畑の横を歩いた。その人が話しかけたのかも分からないつぶやきをした。\\
「生き物をいたわるというのは、良い心を持ったお前の兄さんだね」


その人はまさに職人という名にふさわしい気質だった。もう何年も火薬と玉を触って、どこか世間に脇目も振らない超然とした雰囲気があった。そうでありながらその職人は何やら若々しい対話能力を会得していて、語尾の調子や目配せの仕方が時たまうまくて兄とほとんど変わらないような気がした。ぼくと対面して色々と話してくれるのよりも、隣に座っている彼の友人にどうやら人間関係で悩みを打ち明けられていた時、その職人は説得に必要なときに目を合わせて相手が納得してうなずいたら優しくまばたきを返してまたよどみなく話していた。ぼくは宴会場の隅の席からテーブルを一つ挟んだ正面の席に二人がいるのに気付いてからちらちら様子見したのを今でも思い出せる。もしかしたら二人ともぼくの視線に気が付いていたからそんなそぶりを見せていただけなのかもしれないが。


ぼくはただ一点その花火職人に対して気に入らない思いを抱いていて、それはその職人がぼくよりも兄と親しんでいる様子であることをなおさらぼくに強く意識させた事柄なのだが、その人はおそらく意図的にぼくによく言葉をよくしなさいと言ってきた。人と付き合ううえでまず一番に重要視しないといけないのは言葉を大事にすることだと言った。それはぼくのことを思いやって言ってくれているのだとは誰でも分かることだ。だがぼくは、言葉の選びようによっては相手に一生ものの深い傷を与えることになるということはよく承知しているが、自分の発言をその都度制御せねばならないというのはかなり面倒だし他人が果たしてそんなことを自覚しながら喋っているのか不思議でならないことが多かった。その上ぼくには屈折した感情があった。ぼくは善人にしても悪人にしても人間はみな何かしら一つ装いの欠点を自分で持っていてそれを演技しながら他人にそれを承認してもらっているのを実感しており、ぼくと同年代の子ども達はそれぞれ他人が魅力を感じて愛おしく思いそうな悪い性格を獲得しようと試みているのだと思っていた。ぼくも遅れずに何かを得ようと思った。それこそが人と仲良く付き合うための必要条件であり、この機会を逃したらもう周りに喋りかけてくれる人がいなくなる気がしていたからだ。だからぼくは言葉に気をつけなさいというその助言はずっと納得がいかなかった。もしそれを受け入れたら一生涯寡黙にいなければならないと自分で自分の首を絞めた。なのにふとした瞬間、それはぼくが何日も人と口をきいていないことを悟ったとき、他人同士が楽しそうに話しているのを見るとぼくは会話の仕方と同時に心も失ってしまってもう生きていないのかもしれないと今更本気になって考えていた。皆の眼は笑っていた。良い言葉を失ったぼくの眼は今、もう死んでしまったのかもしれない。


だがぼくは正当防衛としてこんな投げやりな思考をした。それは結局は主に祖母に向けられた。今でさえ自分が面倒をかけた花火職人の加湿器をどうするかずっと父や母に毒づいているではないか。普段の生活でもぼくや父が朝おはようと言っても祖母の気分次第では返事があったりなかったりと中途半端なやりとりをするぐらいの存在が身近にいるではないか。やや差別的に言うがぼくはあの人と同じ空気を吸うのに抵抗してもいいくらい、接し合うのを拒んだ。ぼくがある意味憎悪の目線を向けるのには流石に感付いているだろうが、祖母はぶつぶつと数度念仏を唱えて、おそらく兄に向けて夏物のセーターを編み始めた。


父はどういうことやら熱心にテレビに張り付いている。テレビの中にいる人は個性が輝いていて彼らは鮎釣りとその調理をしていた。釣った役者は焼いた魚の身にかぶりついた。白い鮎の眼と口がその人の顔と一緒に写っていて、瞳孔を失ったその眼がこっちを向いている気がした。\\
「今年も鮎ー、食べようかなあ」


ぼくの心境とは打ってかわってこんなことを父が呟いたのを聞いて、父の旧友もよく鮎をとっていて、父は水辺まで一緒に付き添い帰ったらそれを分け合ってもらっていたというのを聞いたのを思い出した。


編み物を投げ出した祖母も誘って買い物に行く父にぼくはついていくことにした。祖母は兄に来て欲しがったが眠そうな弟も連れて百貨店に出かけた。外は暑かった。ぼくはいつもその地下階の傍の道を歩いていて冷気がその大きな白い扉から漏れ出てくるのを感じていた。実際巨大な百貨店だったので中は奇妙なほどに涼しく、高級そうな焼いた鮎も売られてあった。\\
「漁人にあまり大きいものを持ってこられると、彼らとの値踏みがなかなか面倒なもので」


父は店員と交渉したが祖母が割り込んだ。\\
「養殖なんか?」


ぼくらも地上まで上がって岸辺まで歩くことになった。百貨店の外縁を歩いていると、大きな機械音が後ろの方から聞こえてきた。振り返って上を仰ぐと壁の一部に大きな筒が連なってくっついていた。\\
「空調の排気を出してるんだよ」


巨大建造物の一部の寒いほどの涼しさはあれらのみが熱風を集中して排出することで賄っているという風に気付くとぼくは何やらおぞましい気分になった。ぼくは時々後ろを振り向いて確認しながら歩いた。


黒いコンクリートがむき出しの防波堤の前に露店が出ていて天然鮎と書かれたのぼりがあった。ようやく三尾買って売人が釣銭を渡そうとしたとき、特段大きな硬貨の柄が輝いたのに弟が敏感に反応した。\\
「あ、五百円玉、もつ」


腕を伸ばしてその小銭を受け取ってしまうともうきかなかったので、手に握らせてやって落とさないように目を配りながら横にならんでコンクリートの波止場を歩いた。ぼくは急に慎重になった。目を凝らすと正面の白い漁船の向こうのテトラポットの上で何かが日差しの中でうごめいている。\\
「とんぼや、兄ちゃんとんぼ」


後ろで父たちが何やら兄さんのことで団欒と会話しているのを尻目にぼくは弟が握る左手が緩まないように、足がつっかかって落としてしまわないようにずっと気に掛けていた。弟がぼくに気付かせようと繋いでいる腕を上下に振り回すのに過剰に反応しないようにした。潮波は船から放り出された汚れた網にぶつかり白い泡をうかべて足元のすぐそこにあった。船頭の横を通りこして飛び舞う虫に近付くと弟はひるんでやや手の握りを弱めたようだった。ぼくは大丈夫な気がして首をもたげて遠く横を向いて見やると海は眠るように静かだった。東の向こうに蝦夷がいるとは思えなかった。\\



駅に着くと、大勢の人混みの中から兄が出て来た。弟が大きな声で呼び喜んで手を振った。特に祖母と父は大喜びした。ぼくも久しぶりに会えて懐かしい気持ちがした。ぼくは会話が途切れるのを待ってゆっくりと改札に向かった。後ろを振り向いて兄とその家族が帰っていくのを見送りながら、ポケットからまだ残っている定期を取り出した。


改札の音は無機質だった。残りの期間が表示されてがたんと鳴ったこの機械が時間をすり減らしたようだった。乗り場は下にあり、行き先と方面の表示された、上に突き出した看板から下は明るくなっていた。降りてみるとすでに両側に車両が着いている。中に乗ると淡い空気に包まれて二人掛けの椅子が大きく見えた。少し歩いたところで窓際に詰めて座ると外に池があり、鉄網の向こうで蓮華が咲いていた。電車が出発してから序盤に三回くらいトンネルを抜けると車内は明るくなったり暗くなったりした。ぼくはこれから会いに行く友人のことを回顧していた。講義棟で彼は一人で勉強していた。ぼくもたまに夕闇が深くなるまで付き添って、手洗いから帰ってくるとよく眠ってしまっていたので、起こしてやるとなぜかおやすみと言うのだ。彼の横にある部屋の窓から赤い電灯の明かりが点滅しているのが見えた。ぼくは優しい気分になって、そのまま帰った。そういうことがあったのを思い出していた。またトンネルを抜けるともうそこはギラギラと光る大都会みたいだった。


目的の駅に着いて階段を降りると小さい窓がいくつも付いていた。少し小さめの駅なのだろう、それでも改札まで歩くのに時間がかかったような気がした。改札がまた鳴って、追加料金が引かれていた。すぐそこで彼が待っていた。\\
「\tbaselineshift=2.5pt------\tbaselineshift=4.0pt行こうか。今がちょうど見頃なんだ」


正面の外の広場に出ると、中央の噴水を取り囲んで満開のツツジが出迎えた。よくもまあこんなに植えたものだ。彼は円形の広場を右に突き抜けて地下鉄の方に降りた。彼はどこに行きたいと言ったので、まず古本屋に行こうと伝えた。地下の構内はかなり広かったが、人がいっぱいだった。やけに大きい鞄を背負っているなと彼から言われた。確かに今なんでこんなに大きいリュックで来てしまったのかぼくは考えてしまった。


改札の近くで駅員にどのホームに行けばいいのか聞くと曖昧に返事を返された。彼は礼をして先に行った。ぼくも付いていこうと改札を抜けたが、すぐに後ろから謝る声とともに訂正された。エスカレーターが分かれて扇状に広がっていて、斜面はえらく勾配が低かった。ぼくたちはそれに乗ってゆっくりとホームまでのぼった。その通路はいつの間にか周りに星空が映し出されていた。照明が全てプラネタリウムになったのにもかかわらずぼくたちの足元の黒い階段は明るく見えた。それはホームに出てからもそうで、中央に藤棚が置かれているくらいのスペースに余裕があるのに、すぐそこに星空が迫っているせいで圧迫感があった。


地下鉄の車両に入り込むと、彼は本を読み始めたのでぼくも鞄から同じく本を出した。黒い鞄を膝にのせた男が大きなあくびをした。ぼくは吊革をしっかりと握ってもう片方の手で本を持って読み始めた。電車が出発して次の駅に停まった。


暗い地下鉄の正面のホームにも電車が入ってきて停まったので明るくなった。車両は一つの劇場の箱のように、人の乗り降りを演じた。乗客は順番に入ってきて吊革を握りそれぞれの役柄で静止した。出発音が鳴ってもぼくはちらちら前を向いたので相手側も視線を感じたかもしれない。
車体がいきなり傾いたのにつられて体ごと前に倒れて、顔に近づいた古本の匂いが鼻をかすめた。\\



しばらく暗い地下の中を通っていたせいか、電車が地上に昇った瞬間はふっと視界が明るくなり照明が消えたような気がした。ゆっくりと電車は空中に浮いていった。しばらくぼくたちは窓の下に見える細い線路を眺めていた。モノレールふうの列車は放物線を描きながら市街地に向かってまっすぐに少しずつ落ちていっているのが分かるが、地上に達するまでにはまだまだ遠く時間がかかりそうだ。ぼくには街の中心部を遠くに見て時間を過ごすのはじれったくて不安だった。なんでも窓を通して見た街の建物の全てが一つ一つ小さすぎて何かレンズから覗いているのではないかと錯覚するくらいなのだ。彼は今は本を読みながらゆったりとしている。ぼくは街に着いたらその中でどう振る舞うべきなのか、そもそも彼と一緒に街に向かっている意味や目的はなんなのかと思索したあげく、もやもやとした感情に閉ざされてしまった。今ぼくがいるのは列車の中で、でもぼくの意識の中では到達してしまったあとの街がもう手が届くところにあり、そのぎりぎりの一線を超えてしまうとぼくの現実と妄想はすりかわってしまいそうだ。世界が豹変するような気がした。ぼくはただその時だけ精神が不安定だった。


色々に頭を抱えて悩みながら考えていると、急に車内が揺れたのに気付き、間もなくけたたましい音が中で響いた。乗客たちの携帯のアラーム音であり、多分地震が起きたのだろうと思った。彼も先に気付いて立っていて、ぼくを見て言った。\\
「地震なんだとしたら様子が変だよ。気付いたんなら早く\tbaselineshift=2.5pt------\tbaselineshift=4.0pt」


喋っているのを聞いているとまたかなり激しく上下に揺れ、喚き声があちこちで上がった。地震が起きているというのは分かるが現実には思考が追い付けなかった。空中につながったまま線路が曲がりくねって列車が落ちるのもあり得なくはないほどだった。地上はもっと大きくうねって動いているかも知れない。だめだ、今日は変な日になってしまうやつだなと思った。色々な人と視線が合った。


電車はすでに停車していて、アナウンスが飛び交った。線路を横に拡張するから地上まで歩いてくれということだった。横に延びた通路の到着店はビル街と城の間の町となるらしい。ぼくは彼のほうを向いて、電車の扉が開いたのでとりあえずあきらめて下まで降りようということで合意した。


何人も人が歩いていた。行列が作られている様はまるで先祖の供養の儀式のようだった。そして何歩も何歩も歩いた。下に着くまで二時間くらいかかった。他の電車からも降りて線路を歩いている人が見えた。彼はこう言った。\\
「お前は何も気にしなくてい。...何のプレッシャーも感じなくていいから」


黒い人形がはっきり意識に浮かんだが、実際には電車を降りた時から自覚し始めていたのかもしれなかった。ぼくはうなずくふりをして首を垂れた。彼はぼくを見て労わるつもりでそう言ってくれたのかもしれなかったのに、また視線をそらしてしまったと思った。地上に降りてからもぼくらは無言で歩き続けた。アスファルトの地面はひび割れて液状化している部分もあるらしく、到着地点に今朝行った河川敷が選ばれたのが幸いだった。大建造物の気配を感じてふと上を向いてみたら川の向こうに観覧車があった。ぼくは白のゴンドラをじっと見つめてみたが、動く気配がなかったので今はやはり止まっているのだろうと推測した。いきなりカラフルな物体が現れたので、歩行者はずっとそこを眺めていた。彼もそれに眺め入っていた。皆がその観覧車に心を託していた。地震があったのは確かだが、そこまでの甚大さがあるわけでもないのかもしれなかった。


古本屋がもうすぐそこにあった。ぼくは無事だった私小説を買った。店員のおじさんは何も言わなかった。棚が一つ崩れ落ちていて、彼は手伝おうかといったがとくに何もしなくていいと言われた。ぼくたちは外に出てまた歩いた。しばらくすると別の本屋があり、若い人たちが箱を荷車に乗せて運んでいた。真っ新な本が入荷されている瞬間は初めて見た。\\
「なんで今日やってるんだろう」


ぼくも正直そこは引っかかったが曲がり角まで来たのでずっと向こうを見てみるとトラックが溝にはまっていた。そこから手作業で積み下ろしているみたいだった。ぼくは健気な感じがして協力したくなった。彼も同じで一緒に申し出たら、もうやっとこれが最後の新書なので大丈夫だとのことだった。その人たちによると案外この地域は揺れが少なく済み、近くのレストランも食器の割れなどがなくて普通に営業しているという。\\
「縁起がいいから\tbaselineshift=2.5pt------\tbaselineshift=4.0pt」


ぼくは少し抵抗したが彼が行こうと言って聞かなかった。彼がそういう理由は分かってはいたものの、ぼくは彼がお金を払えるのかが心配だった。だがとりあえずは置いておいて、着いてしまったのでその二階まで上がった。定食屋といった風体でテーブルに花が置かれていた。白い造花だ。ぼくが飲み物と食器とを取りに行って戻ってくると、若い女の店員とテーブルに残った彼が笑って話しているのが見えた。ぼくはちらと木造の壁にはめ込まれた窓の外の風景を見た。見える建物はずば抜けて高いビルでそれ以外は団地すらない住宅地なので、ベージュの外壁に大きく落書きがされているその建物たちは海にそびえているようだった。ぼくは彼らがうらやましかった。ぼくの注文の親子丼を聞いたらその店員はすぐに戻って行ってしまったので、ぼくは少し顔に笑みが残っている彼を見るとテーブルをひっくり返したくなった。


レストランを出てしばらく歩くと城の塔の目の前の堀まで着いた。突然彼が水着を買って泳ごうと言った。すぐ近くにスポーツウェアの店があるので跳ね橋を渡る前にそっちに行こうと言われた。そしていそいそと半パンを買った。赤い模様の入った水着だけを買ってすぐに帰るのはぼくは少し恥ずかしかった。


沐浴場の中は特に人もいなくて少なかった。一階の温泉のエリアだけの話かもしれないと二人は考え、ガラス越しに見ると混んでいなさそうなので、エレベーターに乗って普段はいものこ洗い状態になる七階の白いボタンを押し、円形のプールに向かった。幕を抜けると案外と人がいた。彼がこんな状況下でもよく来れたものだなと近くの人に話しかけた。\\
「いやあ、そんなに揺れもなかったし、局所的なものみたいなんですよ」


ぼくたちは心底安堵したが、彼はまだ胸をなでおろしている感じでもないようだった。彼はそのまま水に入って泳いでいたが、ぼくは少し足をつけただけだ。泳ぐのは苦手でその苦手さあまりに、先に話した先輩とここに練習しに来たことがある。白い肌になじまず野球をしていたその人はぼくと同様かなり泳ぐのが下手らしいと聞いていた。ぼくは泳ぎの練習の順番が回ってきたらとりあえず泳いだのだが、コースロープに腕をひっかけたり水中で足をつったりして散々だった。先輩は足でかいでも進まない感じと言っていた。今も結局泳ぎもせずに旧友の彼と温泉に入った。入ると曇った窓の向こうに新緑が茂っていてしとしとと雨が降っていた。体を洗う時になってシャワーを掴む前に、ぼくは礼を言いかけようとしてうつむいた。


よく見ると、水栓についた水垢が白い斑点を作っている。\\
「\tbaselineshift=2.5pt------\tbaselineshift=4.0pt。」


横を振り向くと、彼は腕についたシャワーの水がまだらな模様を作るのを見て遊んでいた。\\



先生のもとに二人で会いに行った。ぼくは校内で最後に会ったときのことをよく覚えていた。その時にはあの先輩が最初にその人と話していた。ぼくは図書館の二階の吹き抜けの上から本棚の先に先生と先輩が立っているのを見つけて、まるで探し物をやっと見つけたかのように嬉しくなって階段を降りた。図書館の出口の先の渡り廊下のすぐ前で話をした。廊下は洞窟の穴のように暗くなっていた。今は先生の家にはもう一人別の先生も来ていて、袢纏を羽織っていた。


先生はまた同じ話をした。\\
「きみの兄さんは筆まめな人だったんだよ。だから今先生には分かるんだぞ。最近手紙が届いていないことからも分かるんだ。ストレスを抱えがちだったから、学級日誌にその愚痴を書き込んだことがあっただろう。」


先生の家の玄関の前の柱で羽化したところの蝉が雨に打たれて地面に落ちていた。真っ白な体に緑色の筋をつくって、抜け殻の横で足を動かしていた。\\
「そういう態度でいて、そういう調子でずっとやってきているから、まあ今の仕事も大してそんなに頑張っていないんだろうね。やらないといけないということは昔よりしっかりと分かっているよ。でもきみもそんな宙ぶらりんな状態でやっていってほしくないね。」


ぼくは瀕死の蝉を見ていると昨日の鮎を思い出した。川の苔の味がしたのも思い出した。放ったらかしにしている先生の顔を見ながら、話を聞きながら、その蝉について想像を巡らせていた。\\
もう一人の先生に旧友が質問した。\\
「それでこの減価償却というのが分からなくて...この用語集にしかないんですけど」
その先生はテレビを横目にそれをちらりと見て、わざとらしく首を傾げてまたテレビに目線を戻した。\\



別れを告げると彼はなぜか走って帰っていった。ぼくは何度か振り向いて確かめたが夕暮れが空気を青くしていたのではっきりと見えなくなっていた。ぼくは沐浴場を出て城の周りを歩いて居館の入り口まで帰った。十五分ほどかかる距離で、右にはガラスしか見えず左を向いても城下町が堀の向こうに見て取れるだけだ。


明日のことがぼくのこれまでの人生のうちでもっとも重大な行事であることはずっと前から無意識的に分かっていたつもりだった。そのせいでいつの間にかぼくの精神は相当虐げられていた。いま帰途についているこの時間も、明日訪れてくるのであろう時間も、ぼくの長い長い生活の一部を均等に割り振って流れているというのを、ぼくは信じられなくて、でもいやいやながら自覚してしまった。


帰り始めた当初は、帰るとまたしばらく自堕落な時間を過ごしているのだろうなどと思っていたが、やはり明日の事の重大さに感づいてしまってからにはぼくはもう発想を転換している。その重大な行事が来るまでの時間、ぼくは案外大人しくしているのかもしれない。いつもよりも随分といい時間を過ごしているのかもしれない。本当はその時が来るまで何も分からないはずのくせに、いまぼくがいる空間のなかでぼくは色々と空想してやり過ごそうとした。帰ってもぼくは荷物を部屋に置くまでずっともやもやとしていた。


ぼくはいつの間にか兄のいる部屋に向かっていた。ドアをたたいて返事を確かめると、いいから入ってと言われた。兄はビデオプレイヤーで実況を見ていた。ぼくは急にむなしくなって、脇腹にそっと手を寄せた。\\
「すぐれた人だよ。何年も前からずっと、やってることは同じで。声に愛着があってどうしても見てしまうんだよなあ。ほら見てみなよ」


感想が貼られていた。地震が起きて周囲がめちゃくちゃなままだがこの人がいつも通りの動きを見せるので安心したと言っている。ぼくはもう眼に涙をためているが、流してしまわないように耐えた。\\
「ひとは変わらないものを見ると慰められるんだよ。こんなふうに\tbaselineshift=2.5pt------\tbaselineshift=4.0pt」


兄は涙を手で拭いながら隠しているぼくを見て頭をなでてくれた。兄はもう地震があってそれが黒い人形の影響かも知れないという事情は知ってくれているのだ。
ぼくはしばらくして落ち着いた。兄も少し低い口調で言った。\\
「\tbaselineshift=2.5pt------\tbaselineshift=4.0pt多分、今の職場でやっていけるような気がする。同部署の人たちも仲がいいし、環境がいい。これ以上何か求めて変えてしまったら、もう甘んじる能力を失ってしまうかもしれないんだ。本当に。最初そこにある古い時計が読みづらくてその度にイライラして妙だったんだけど、その時計もこれまでこの職場に入ってきたり辞めていったりした人たちを見てきたんだろうなあ、って勝手に思ってるけど、このままで大丈夫だろうと思うよ」


ぼくは兄の部屋の壁に取り付けられた大きなデジタル時計を見て、彼におやすみと言って部屋をあとにした。ぼくは兄が昔まで使っていた書斎に入って、ある本を手に取った。明かりの下でそれを開くと、左上の端に長い暗号みたいなものが付されていて、中央に大きな絵がある。しかもただの絵ではないようなのだった。その巨大な絵の正面で人が歩いているのが分かるからだ。その絵は鳥の頭みたいなものや炎をまとった頭蓋骨みたいなものを擁していた。ぼくは不思議な気分になった。ただ久しぶりに確かめたような気がした。


ぼくは書斎からぼくの部屋まで歩いていた。闇の中を泳ぐようだった。全く腹が空かなかった。脳に釘を刺されて夢を見ているようだった。気付くとぼくは部屋の扉の前に座っていた。ぼくは単に明日が不安で眠れないのだろうと思っていた。だが布団にもぐっても全く眠れなかった。その代わりに色々と過去のことがぼくの頭をめぐるのが分かった。


ぼくの家族と叔父さんの家族で山の中まで遠く行ったことがあるのを思い出した。秋の朝の空気は澄んで涼しく重く感じられた。出た時はまだ暗かったので弟が不満な様子でどこに行くのか尋ねると、ただある宗教団体が運営している美術館に絵画を見に行こうということらしかった。どうやら展示の期限が迫っているらしく人混みの多い展覧会で見かけたある風景画がぼくの感受性に深く訴えかけていた。「夕星」と題がついていた。夜空に白い星が浮かんでいる。隣の絵の正面からでもその大きな絵に眺め入っていたりする人の眼はどれも美しかった。ぼくは自然に対して敏感にものごとを感じがちなので特にその絵を見た途端に思い付いた芸術の仕組みに興奮を覚えた。あの幻想的な絵も含めてすべて、人間が人間の死の悲しみをなだめるために殊勝になって芸術を創るのだろうと思った。


冬になるとある話がぼくのもとに舞い込んできた。兄ともう一人の男の人が雪の中城の横にある楓の樹の一本を斧で切ったということだった。なぜ兄がそんなことをしているのかは分からなかった。職人に言われた言葉がその時から引っかかっていたが、多分そのもう一人の男の人と張り合っていただけなのではないかと思った。そんな話を本で読んだことがあった。


土曜の朝は幸せだった。初夏の風が雨で湿った道路を伝わって、冷たく感じるくらいに爽やかだった。道上に手ぶらの男がただ一人歩いていて、揺れる並木を見ながら軽い足取りで歩いていると穏やかな気分になった。ぼくが外に出ていたのはただ例の花火職人に会って本を返すためだった。来るのは昼でいいと伝えられていたので一人で講義棟に寄って何もせずくつろいでいた。窓の景色から感じ取った外の空気は気温が高くて蒸し暑そうな雰囲気がしたが、本当はじゅうぶんに良い気候だった。そんな訳もあってぼくの不安な気持ちは鈍くなり、用事を済ませばすぐに帰れることに満足した。ぼくは一時的に歓喜を感じていた。こういう感情には付随するものがあるのをぼくは知っていた。
ふっと蛍の光の旋律が頭に思い浮かんだ。ぼくはこの曲に付随してくる一種のトラウマをまた覚えずにいられなかった。というのも、二ヶ月前にあった旧友たちとの最後の集まりに出られなかったのだ。というよりも、出なかったという表現の方がぼくの罪の意識を表すのには妥当なのかもしれない。件の蛍の光は当時の感傷を慰めてくれたと同時にぼくに重い何かを背負わせた唯一の曲だ。


式は誰にとっても眠そうだった。ぼくなんて式の内容には集中せず鍵のないロッカーの箱から財布が奪われやしないかとずっと気に掛けていたほどだ。その代わりに出来るだけ式に参加しているふうに演技をした。偉い人が何か話しているときぼくはうんうんとうなずいたり目線を適度に揺らしたりして何かを示そうとした。周囲からやたら胡散臭いと思われてももうぼくはぼくの存在を誇張することに精一杯な気持ちで、多少の不名誉について気にかけるのは放棄して終始そんなことに気を張り詰めていた。これは今から省みてみたら分かるが完璧なぼくのエゴイズムの表れだったのだ。曲を歌っているときはもう感情の頂点にいて、終わった後は疲れて皆が散り散りにならないうちにぼくは何も深く考えすぎないように気を付けながらふらっと会場を出た。そうやって歩いていたらいつの間にか頭の中に蛍の光の旋律がこびりついていた。道路わきに出ると後ろが開いた車が目に入って、積み込まれた段ボール箱がむなしく放置されている。ぼくはひどく感傷にふけって、夢現で呪われたかのような重い足取りで革靴を鳴らして帰った。式の最中にぼくは無限の目線にさらされていたはずなのに個になるとかえってしんどくて、道を歩く一瞬一瞬が積み重なっていくのを感じていた。ぼくは同じ時間の流れを感じられなかった。ぼくが置いてきぼりにした大量の時間が過去の時空からどっとぼくの胸の奥に押し寄せてきた。ぼくは怒りを覚えずにはいられなかった。


過去の感情がそのまま再現されていた。それでぼくはしばらく眠れなかった。\\



ふと気付くともう朝になっていた。だがすぐに眠気が襲い、目が開いたままでぼくは苦しかった。ぼくは清貧な功徳心をもってしてここまでやってきたつもりでいた。今日はしめやかに式を済ませようと思うと、窓から日差しが差し込んできたので、ブラインドを閉めて日差しが入らないようにした。


それでしばらく後、ぼくはむしろすがすがしくて居間に入ると、ある報告が入った。学者の研究班からだった。写真には黒い人形がひもで縛りつけられ、腕の先や首がとれてばらばらになっているのが写っていた。昨日会ったぼくの旧友が夕方、港から島を出たらしかった。ぼくは彼がまた帰ってきたら、彼を祝福しようと言った。それには全員が賛同した。そのあとは会場で踊りを見た。ぼくは異国の人を招き入れたことは何も考えないようにしていたが、彼らの古い踊りを見ていると深く心に残る気がした。女の人は回りながら舞い、その中央で男の人が勇壮とした踊りを披露してくれた。面白い演目だった。そしてぼくの宣誓がなされた。\tbaselineshift=2.5pt------\tbaselineshift=4.0pt\\


こうして彼は新たなる王となったのである。


\end{document}
